\documentclass[12pt, oneside]{article}
\usepackage[letterpaper, margin=1in]{geometry}
\usepackage[english]{babel}
\usepackage[utf8]{inputenc}
\usepackage{amsmath}
\usepackage{amsfonts}
\usepackage{amssymb}
\usepackage{tikz}
%\usepackage{tkz-fct}
\usepackage{pgfplots}
\pgfplotsset{width=10cm,compat=1.9}
\usepgfplotslibrary{statistics}
\usepackage{pgfplotstable}
%\usepackage{venndiagram}

\usepackage{fancyhdr}
\pagestyle{fancy}
\fancyhf{}
\rhead{\thepage \\Name: \hspace{1.5in}}
\lhead{BECA / Dr. Huson / 11.1 IB Math SL\\* 8 May 2018 \\*\textbf{IB Questionbank: Sequences, logarithms\\*
}}

\renewcommand{\headrulewidth}{0pt}

\begin{document}


\subsection*{\\Sequences \& series}


\begin{enumerate}



\item The first three terms of a geometric sequence are $u_1=1.2$, $u_2=3$, and $u_3=7.5$.
\begin{enumerate}
    \item Find the value of $r$.
        \begin{flushright}[2]\end{flushright}
    \item Find the value of $S_6$.
        \begin{flushright}[2]\end{flushright}
    \item Find the least value of $n$
such that $S_n>300$. 
        \begin{flushright}[3]\end{flushright}
\end{enumerate}

\item Three consecutive terms of a geometric sequence are $x-2$, 6, and $x+7$.\\
Find the possible values of $x$.
    \begin{flushright}[6]\end{flushright}


\item Find the value of each of the following, giving your answer as an integer.
\begin{enumerate}
    \item $\log_6 36$.
        \begin{flushright}[2]\end{flushright}
    \item $\log_6 4 + \log_6 9$.
        \begin{flushright}[2]\end{flushright}
    \item $\log_6 2 - \log_6 12$.
        \begin{flushright}[3]\end{flushright}
\end{enumerate}

\item Solve $\log_2 x + \log_2 (x-2) = 3$, for $x>2$.
    \begin{flushright}[7]\end{flushright}

\item Let $f(x)= e^{x+3}$.
\begin{enumerate}
    \item
    \begin{enumerate}
    \item Show that $f^{-1}(x) = \ln x - 3$.
        \begin{flushright}[3]\end{flushright}
    \item Write down the domain of $f^{-1}$.
    \end{enumerate}
    \item Solve the equation $f^{-1}(x) = \ln \frac{1}{x}$.
        \begin{flushright}[4]\end{flushright}
\end{enumerate}

\item Solve the equation $e^x = 4 \sin x$, for $0 \leq x \leq 2 \pi$.
    \begin{flushright}[5]\end{flushright}


\end{enumerate}
\end{document}}