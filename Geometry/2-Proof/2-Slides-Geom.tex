\documentclass{beamer}
\usepackage{geometry}
\usepackage[english]{babel}
\usepackage[utf8]{inputenc}
\usepackage{amsmath}
\usepackage{amsfonts}
\usepackage{amssymb}
\usepackage{tikz}
\usetikzlibrary{quotes, angles}
\usepackage{graphicx}

%\usepackage{pgfplots}
%\pgfplotsset{width=10cm,compat=1.9}
%\usepackage{pgfplotstable}

\setlength{\headheight}{26pt}%doesn't seem to fix warning

\usepackage{fancyhdr}
\pagestyle{fancy}
\fancyhf{}

%\rhead{\small{10 September 2018}}
\lhead{\small{BECA / Dr. Huson / Geometry Unit 1}}

\renewcommand{\headrulewidth}{0pt}

\title{Mathematics Class Slides}
\subtitle{Bronx Early College Academy}
\author{Chris Huson}
\date{9 October 2018}

\begin{document}
\frame{\titlepage}
\section[Outline]{}
\frame{\tableofcontents}

\section{Project criteria}
  \frame
  {
    \frametitle{GQ: How do we present mathematical work?}
    \framesubtitle{CCSS: HSG.CO.D.12 Congruence, Make geometric constructions}

    Complete binder: \alert{Due Friday}\\
    Exam 1 + corrections; exam 2 (optional corrections); 5 best construction:\\
    Equilateral triangle, Congruent segment \& angles, bisected segment \& angle
      \begin{block}{Criteria for construction projects}
      \begin{enumerate}
          \item Complete and correct construction
          \item Steps written with proper notation
          \item Layout: GQ title, date on left; first \& last name on right
          \item Precise, elegant, mathematical aesthetic
      \end{enumerate}
      \end{block}
    Grading policy: full credit 20, minus 2 points for each missing\\[5pt]
  }

\section{Notetaking criteria}
  \frame
  {
    \frametitle{GQ: How do we organize our mathematical notes?}
    \framesubtitle{CCSS: HSG.CO.A.1 Know precise geometric definitions}

    \begin{block}{Criteria for notebook project grade (20 points)}
    \begin{enumerate}
      \item Your name and "Geometry" on cover
      \item Toward front: math.huson.com, husonbeca@gmail.com, 917-648-5632, Deltamath teacher ID: 546068
      \item Labeled composition book out during class; GQ, date each day
      \item Definitions, postulates, constructions, \& theorems
      \item Combination of symbols, diagrams, text (best: your own words)
      \item Examples, but not practice problem sets
    \end{enumerate}
    \end{block}
    Grading policy: daily tracker, pop notebook checks
  }

\section{2.1 Drui: Induction, pattens. Monday 15 October}
  \frame
  {
    \frametitle{GQ: How do we reason logically?}
    \framesubtitle{CCSS: HSG.GPE.B.7 Compute areas and perimeters using the distance formula \hspace{\stretch{1}} \alert{2-1}}

    \begin{block}{Do Now: Area practice. Given the polygon with vertices $M(4,0), A(8,0), T(8,4), H(4,4)$}
    \begin{enumerate}
        \item Sketch $MATH$. What kind of polygon is it?
        \item Find the area and perimeter of $MATH$.
        \item Spicy: A circle is inscribed in the polygon, centered at $C(6,2)$ and touching each side in one spot. Find the area and perimeter of circle $C$.
    \end{enumerate}
    \end{block}
    2-1 Inductive logic  pp. 82-84\\
    Classwork problems 6-30 odds p. 85\\
    \vspace{0.5cm}
    Homework: Perimeter \& area practice
  }

\section{2.2 Drui: Deltamath. Tuesday 16 October}
  \frame
  {
    \frametitle{GQ: How do we use geometric notation?}
    \framesubtitle{CCSS: HSG.CO.D.12 Congruence, Make geometric constructions \hspace{\stretch{1}} \alert{2-2}}

    Deltamath practice\\ \bigskip
    Homework: Complete deltamath (10pm deadline)
  }

\section{2.3 Drui: Induction, logic. Wednesday 17 October}
  \frame
  {
    \frametitle{GQ: How do we apply the equilateral triangle construction?}
    \framesubtitle{CCSS: HSG.CO.D.12 Congruence, Make geometric constructions \hspace{\stretch{1}} \alert{2-3}}

    \begin{block}{Do Now:}
      \begin{center}
      \begin{tikzpicture}
        %\draw [->, thick] (0,0)--(5,5);
        \draw [<-, thick] (-2,0)--(3,0)--(2,1.5)--(1,0);
        \draw [fill] (-1,0) circle [radius=0.05] node[below]{$R$};
        \draw [fill] (1,0) circle [radius=0.05] node[below]{$S$};
        \draw [fill] (2,1.5) circle [radius=0.05] node[right]{$U$};
        \draw [fill] (3,0) circle [radius=0.05] node[right]{$T$};
      \end{tikzpicture}
      \end{center}
      \begin{enumerate}
        \item Given $m\angle RSU = 115^\circ$. Find $m\angle TSU$
        \item Given $S$ bisects $\overline{RT}$, $RS=\frac{1}{5} (x+8)$ and $ST = x$. Find $RT$.
    \end{enumerate}
    \end{block}
    Equilateral triangle construction applications, Engage workbook\\
    \vspace{0.25cm}
    Homework: Engage workbook
  }

\section{2.4 Drui: Conditional statements, logic. Thursday 18 October}
  \frame
  {
    \frametitle{GQ: How do we reason logically?}
    \framesubtitle{CCSS: HSG.CO.C.9 Prove geometric theorems \hspace{\stretch{1}} \alert{2-4}}

    \begin{block}{Do Now: Euclidean constructions}
    \begin{enumerate}
        \item Construct a perpendicular to a line through a given point
        \item Duplicate a given line segment
        \item Bisect a given angle
    \end{enumerate}
    \end{block}
    New construction: Duplicate an angle\\
    2-2 Conditional statements, logic  pp. 89-92\\
    Classwork problems 5-24 odds p. 93\\
    \vspace{0.5cm}
    Homework: Engage workbook Lesson 3 Problem Set p. S.17. \\
    Spicy: Engage workbook Lesson 2 Challenge 1, 2 p. S.8, S.9
  }

\section{2.5 Drui: Converse, contrapositive, definitions. Friday 19 October}
  \frame
  {
    \frametitle{GQ: How do we reason logically?}
    \framesubtitle{CCSS: HSG.CO.C.9 Prove geometric theorems \hspace{\stretch{1}} \alert{2-5}}

    \begin{block}{Do Now: Sketch and label the figure shown}
      \begin{center}
      \begin{tikzpicture}
        \draw [->, thick] (3,0)--(4,0);
        \draw [<-, thick] (-2,0)--(3,0)--(2,1.5)--(1,0);
        \draw [fill] (-1,0) circle [radius=0.05] node[above]{$R$};
        \draw [fill] (1,0) circle [radius=0.05] node[above left]{$S$};
        \draw [fill] (2,1.5) circle [radius=0.05] node[right]{$U$};
        \draw [fill] (3,0) circle [radius=0.05] node[above right]{$T$};
      \end{tikzpicture}
      \end{center}
      \begin{enumerate}
      \item Name two opposite rays
      \item Given $m\angle TSU = 55^\circ$. Find $m\angle RSU$
      \item $S$ bisects $\overline{RT}$, $RT=\frac{1}{2} (3x+15)$ and $ST = x+3$. Find $RS$.
    \end{enumerate}
    \end{block}
    2-2 Conditional statements, logic  pp. 89-92\\
    Classwork problems 5-24 odds p. 93\\
    %\vspace{0.5cm}
    Homework: Engage workbook Lesson 4 Problem Set p. S.22-23 \\
    Spicy: \#3 p. S.24
  }

\section{2.6 Drui: Deductive logic, two column proofs. Monday 22 October}
  \frame
  {
    \frametitle{GQ: How do we use deductive logic?}
    \framesubtitle{CCSS: HSG.CO.C.9 Prove geometric theorems \hspace{\stretch{1}} \alert{2-6}}

    \begin{block}{Do Now: Area practice.}
    \begin{enumerate}
        \item Find the area of rectangle with length 3.5 and width 7.1.
        \item Find the width of rectangle with length 17.5 and area 84.
        \item Spicy: Find the dimensions of a rectangle with area 84 having length five greater than its width.
        \item Given an example of the distributive property.
    \end{enumerate}
    \end{block}
    2-5 Congruence, addition proofs  pp. 113-116\\
    Classwork problems 5-13 p. 117\\
    \vspace{0.5cm}
    Homework: Engage workbook Lesson 5 Problem Set p. S.29-30
  }

\section{2.7 Drui: Deltamath. Tuesday 23 October}
  \frame
  {
    \frametitle{GQ: How do we calculate area and perimeter?}
    \framesubtitle{CCSS: HSG.CO.D.12 Congruence, Make geometric constructions \hspace{\stretch{1}} \alert{2.7}}

    Deltamath practice\\ \bigskip
    Homework: Complete deltamath (10pm deadline)\\
    Engage workbook Lesson 6 Problem Set p. S.37
  }

\section{2.8 Drui: 2-column addition proofs. Wednesday 24 October}
  \frame
  {
    \frametitle{GQ: How do we use deductive logic?}
    \framesubtitle{CCSS: HSG.CO.C.9 Prove geometric theorems \hspace{\stretch{1}} \alert{2-8}}

    \begin{block}{Do Now: Handout review and practice.}
    \end{block}
    Lesson: 2-6 Congruence, addition proofs  pp. 120\\
    Classwork problems 5-24 odds p. 124\\
    \vspace{0.5cm}
    Homework: Pre-test review packet
  }

\section{2.9 Drui: Review. Thursday 25 October}
  \frame
  {
  \frametitle{GQ: How do we apply the properties of angle pairs?}
  \framesubtitle{CCSS: HSG.CO.D.12 Congruence, Make geometric constructions \hspace{\stretch{1}} \alert{2-9}}

  \begin{block}{Do Now: Handout angle calculation problems review and practice.}
    %\begin{enumerate}
        %\item
    %\end{enumerate}
  \end{block}
  Lesson: Pretest review of constructions, angle properties, logic terminology, algebraic methods (textbook through p. 105)\\
  Students work packet problems on board\\
  \vspace{0.5cm}
  Homework: Study for \alert{exam tomorrow}
}

\section{2.10 Drui: Test. Friday 26 October}
        \frame
        {
          \frametitle{GQ: How do we use the tools of geometry?}
          \framesubtitle{CCSS: HSG.CO.A.1 Know precise geometric definitions \hspace{\stretch{1}} \alert{2-10}}

          \begin{block}{Do Now: (Test)}
          %\begin{enumerate}
              %\item Use your assigned laptop number
          %\end{enumerate}
          \end{block}
          Test\\
          \vspace{1cm}
          Homework: Angle measure algebra problems
        }

\section{2.11 Drui: Addition proofs, transversals. Monday 29 October}
  \frame
  {
    \frametitle{GQ: How do we name the angles of a transversal?}
    \framesubtitle{CCSS: HSG.CO.C.9 Prove geometric theorems \hspace{\stretch{1}} \alert{2-11}}

    \begin{block}{Do Now: Vertical angle proof applications}
    \begin{enumerate}
        \item Lesson check \#1, 6, 12 p. 124 from textbook
        \item Spicy: \#33, 34 p. 127
        \item Spicy: Given $\angle ABC$, construct duplicate $\angle CDE$
        \begin{center}
        \begin{tikzpicture}
          \draw [<->, thick] (2,2)--(0,0)--(8,0);
          \draw [->, thick, dashed] (4,0)--(6,2) node[below right]{$E$};
          \draw [fill] (1.5,1.5) circle [radius=0.05] node[above left]{$A$};
          \draw [fill] (0,0) circle [radius=0.05] node[above left]{$B$};
          \draw [fill] (4,0) circle [radius=0.05] node[above left]{$D$};
          \draw [fill] (7,0) circle [radius=0.05] node[above right]{$C$};
        \end{tikzpicture}
        \end{center}
    \end{enumerate}
    \end{block}
    Transversal and corresponding angles pp. 140-142\\
    Classwork problems 17-23 p. 144\\
    \vspace{0.15cm}
    Homework: Handout transversal practice and median construction
  }

\section{2.12 Drui: Deltamath. Tuesday 30 October}
  \frame
  {
    \frametitle{GQ: How do we construct the centroid?}
    \framesubtitle{CCSS: HSG.CO.D.12 Congruence, Make geometric constructions \hspace{\stretch{1}} \alert{2.12}}

    Deltamath practice: triangle centers, transversal practice\\ \bigskip
    Homework: Complete deltamath (10pm deadline)\\
    Graph midpoint practice
  }

\section{2.13 Drui: Transversals. Wednesday 31 October}
  \frame
  {
    \frametitle{GQ: How do we compare the angles of a transversal?}
    \framesubtitle{CCSS: HSG.CO.C.9 Prove geometric theorems \hspace{\stretch{1}} \alert{2-13}}

    \begin{block}{Do Now: Given two parallel lines shown, $m\angle 5=110$. Find all other angle measures.}
    \begin{enumerate}
      \begin{center}
      \begin{tikzpicture}[scale=0.8]
        \draw [<->, thick] (1,2)--(9,2);
        \draw [<->, thick] (0,0)--(8,0);
        \draw [<->, thick] (4,-1)--(5.5,3);
        \node at (4.5,0.3) [left]{$5$};
        \node at (4.5,0.3) [right]{$6$};
        \node at (4.3,-0.3) [left]{$7$};
        \node at (4.3,-0.3) [right]{$8$};
        \node at (5.2,2) [above left]{$1$};
        \node at (5.2,2) [above right]{$2$};
        \node at (5,2) [below left]{$3$};
        \node at (5,2) [below right]{$4$};
      \end{tikzpicture}
      \end{center}    \end{enumerate}
    \end{block}
    Transversal angle theorems pp. 148-152\\
    Classwork problems 7-9, 12-17 p. 153\\
    \vspace{0.2cm}
    Homework: Triangle center project\\
    Engage workbook lesson 7 classwork p. S38-39 mild, spicy p. S40
  }

\section{2.13 Project: Triangle centers paper, Wednesday 31 October}
  \frame
  {
    \frametitle{GQ: How do we construct the centroid, circumcenter, incenter, and orthocenter?}
    \framesubtitle{CCSS: HSG.CO.C.9 Prove geometric theorems \hspace{\stretch{1}} \alert{2-13}}

    \begin{block}{Construction project: Triangle centers}
    \begin{enumerate}
        \item Circumcenter: perpendicular bisectors
        \item Incenter: angle bisectors
        \item Orthocenter: altitudes (perpendiculars through vertices)
        \item Centroid: medians (midpoint to opposite vertices)
    \end{enumerate}
    \end{block}
    We will have time at Kipps Bay Center. Due Monday
    }

\section{2.14 Drui: Transversals. Monday 5 November}
  \frame
  {
    \frametitle{GQ: How do we set up a geometry problem?}
    \framesubtitle{CCSS: HSG.CO.C.9 Prove geometric theorems \hspace{\stretch{1}} \alert{2-14 Monday Nov 5}}

    \begin{block}{Do Now: Formulating geometric situations, handout}
      \begin{enumerate}
          \item When are two angles congruent? Two line segments?
          \item When are angles supplementary or complementary?
          \item What theorems justify the answers?
      \end{enumerate}
    \end{block}
    Triangle center project due today (math.huson.com list)\\
    Exam review\\
    \vspace{0.2cm}
    Homework: test corrections due Wednesday\\
    Trimester final exam Thursday
    }
\section{2.14+ Drui: Transversals. Wednesday 7 November}
  \frame
  {
    \frametitle{GQ: How do we construct triangle centers?}
    \framesubtitle{CCSS: HSG.CO.C.9 Prove geometric theorems \hspace{\stretch{1}} \alert{2-14}}

    Triangle center project time\\
    Exam review\\
    Kipps Bay youth center
  }

\section{2.15 Drui: Trimester exam. Thursday 8 November}
  \frame
  {
    \frametitle{GQ: How do we set up a geometry problem?}
    \framesubtitle{CCSS: HSG.CO.C.9 Prove geometric theorems \hspace{\stretch{1}} \alert{2-15 Thursday Nov 8}}
    Trimester final exam
  }

\section{2.16 Drui: Transversals. Friday 9 November}
  \frame
  {
    \frametitle{GQ: How do we quantify slope?}
    \framesubtitle{CCSS: HSG.CO.C.9 Prove geometric theorems \hspace{\stretch{1}} \alert{2-16 Friday Nov 9}}

    \begin{block}{Do Now: Justify the congruence statements}
    \end{block}
    \emph{iff} means ``if and only if," i.e. both statement and converse\\[0.2cm]
    Theorems: \\
    The sum of a triangle internal angle measures is $180^\circ$ \\
    Different lines have equal slopes \emph{iff} they are parallel\\
    Lines are $\perp$ \emph{iff} the product of their slopes is $-1$\\[0.2cm]
    Homework: Triangle and slope practice, handout
    }
\end{document}
