\documentclass{beamer}
\usepackage{geometry}
\usepackage[english]{babel}
\usepackage[utf8]{inputenc}
\usepackage{amsmath}
\usepackage{amsfonts}
\usepackage{amssymb}
\usepackage{tikz}
\usetikzlibrary{quotes, angles}
\usepackage{graphicx}

%\usepackage{pgfplots}
%\pgfplotsset{width=10cm,compat=1.9}
%\usepackage{pgfplotstable}

\setlength{\headheight}{26pt}%doesn't seem to fix warning

\usepackage{fancyhdr}
\pagestyle{fancy}
\fancyhf{}

%\rhead{\small{10 September 2018}}
\lhead{\small{BECA / Dr. Huson / Geometry Unit 1}}

\renewcommand{\headrulewidth}{0pt}

\title{Mathematics Class Slides}
\subtitle{Bronx Early College Academy}
\author{Chris Huson}
\date{24 September 2018}

\begin{document}
\frame{\titlepage}
\section[Outline]{}
\frame{\tableofcontents}

\section{1b.0 Project criteria}
  \frame
  {
    \frametitle{GQ: How do we present mathematical work?}
    \framesubtitle{CCSS: HSG.CO.D.12 Congruence, Make geometric constructions \hspace{\stretch{1}} \alert{1b.0}}

    \begin{block}{Criteria for construction projects}
    \begin{enumerate}
        \item Complete and correct construction
        \item Steps written with proper notation
        \item Layout: GQ title, date on left; first, last name on right
        \item Precise, elegant, mathematical aesthetic
    \end{enumerate}
    \end{block}
    Grading policy: full credit (20 out of 20) or redo\\[5pt]
    Constructions: \alert{Due Friday}\\
    Congruent segment \& angles, bisected segment \& angle \\
    (collect exams and projects in classroom binder)
  }

\section{1b.0 Notetaking criteria}
  \frame
  {
    \frametitle{GQ: How do we organize our mathematical notes?}
    \framesubtitle{CCSS: HSG.CO.A.1 Know precise geometric definitions \hspace{\stretch{1}} \alert{1b.0}}

    \begin{block}{Criteria for notebook project grade (20 points)}
    \begin{enumerate}
        \item Labeled composition book out during class; GQ, date each day
        \item Definitions, postulates, constructions, \& theorems
        \item Combination of symbols, diagrams, text (best if in your own words)
        \item Examples, but not practice problem sets
    \end{enumerate}
    \end{block}
    Grading policy: daily tracker, pop notebook checks (Wednesday)
  }

  \section{1b.1 Drui}
    \frame
    {
      \frametitle{GQ: How do we classify angle pairs?}
      \framesubtitle{CCSS: HSG.CO.A.1 Know precise geometric definitions \hspace{\stretch{1}} \alert{1b.1}}

      \begin{block}{Do Now: Practice and review}
      \begin{enumerate}
          \item Segment addition problem
          \item Equilateral triangle construction
          \item Segment replication
      \end{enumerate}
      \end{block}
      Construct a congruent line segment \\
      1-5 Exploring Angle Pairs pp. 34-37\\
      Classwork problems 7-26 odds pp. 38\\
      Review Friday's Exam.\\
      \vspace{0.5cm}
      Homework: Exam corrections due Wednesday\\
    }

\section{1b.2 Drui: Deltamath. Tuesday 25 Sept}
  \frame
  {
    \frametitle{GQ: How do we use geometric notation?}
    \framesubtitle{CCSS: HSG.CO.A.1 Know precise geometric definitions \hspace{\stretch{1}} \alert{1b.2}}

    \begin{block}{Do Now: Laptop setup}
    \begin{enumerate}
        \item Use your assigned laptop number
        \item ``Lids down" for group focus
        \item Return laptops to proper slot number, charging cable
    \end{enumerate}
    \end{block}
    Deltamath geometry notation; \\Challenge Geogebra construction\\
    \vspace{1cm}
    Homework: Angle measure, algebra problems
  }

\section{1b.3 Drui: Vertical angles. Wednesday Sept 26}
  \frame
  {
    \frametitle{GQ: How do we classify angle pairs?}
    \framesubtitle{CCSS: HSG.CO.A.1 Know precise geometric definitions \hspace{\stretch{1}} \alert{1b.3}}

    \begin{block}{Do Now: Construction practice and review}
    \begin{enumerate}
        \item Given $\overline{AB}$, construct a congruent line segment
        \item Given $\overline{DE}$, construct an equilateral triangle
    \end{enumerate}
    \end{block}
    1-5 Exploring Angle Pairs pp. 34-37\\
    Classwork problems 7-26 odds pp. 38\\
    Construct a perpendicular bisector \\
    \vspace{0.5cm}
    Homework: Angle pair practice
  }

\section{1b.4 Drui: Construct perpendicular bisector. Thursday Sept 27}
  \frame
  {
    \frametitle{GQ: How do we classify angle pairs?}
    \framesubtitle{CCSS: HSG.CO.A.1 Know precise geometric definitions \hspace{\stretch{1}} \alert{1b.4}}

    \begin{block}{Do Now: Angle pair practice. Show steps, including the check.}
    \begin{enumerate}
        \item Given two supplementary angles: $m \angle 1 = 50$, $m \angle 2 = x$.\\ Find $x$.
        \item Given two complementary angles: $m \angle 1 = x+10$, $m \angle 2 = x+20$. Find $m \angle 1$.
        \item Given two vertical angles: $m \angle 1 = 3x+10$, $m \angle 2 = 55$.\\ Find $x$.
    \end{enumerate}
    \end{block}
    1-5 Exploring Angle Pairs pp. 34-37\\
    Classwork problems 8-30 evens pp. 38-39\\
    Construct a perpendicular bisector \\
    \vspace{0.2cm}
    Homework: Angle pair practice
  }

\section{1b.5 Drui: Constuct angle bisector. Friday Sept 28}
  \frame
  {
    \frametitle{GQ: How do we do classical constructions?}
    \framesubtitle{CCSS: HSG.CO.D.12 Congruence, Make geometric constructions \hspace{\stretch{1}} \alert{1b.5}}
    Do Now: Angle pair practice
    \begin{block}{Constructions due today: Complete, correct, precise, elegant}
      Standard header. (you may combine constructions on same page)
      \begin{enumerate}
          \item Equilateral triangle (you may combine on same page)
          \item Congruent segments
          \item Perpendicular bisector
          \item New: Angle bisector
          \item Spicy: Flower design p. 42
      \end{enumerate}
    \end{block}
    Classwork problems 3-25 odds pp. 41\\
    \vspace{0.5cm}
    Homework: Angle pair practice
  }

\section{1b.6 Drui: Midpoints. Monday Oct1}
  \frame
  {
    \frametitle{GQ: How do we calculate the midpoint of a segment?}
    \framesubtitle{CCSS: HSG.CO.A.1 Know precise geometric definitions \hspace{\stretch{1}} \alert{1b.6}}

    \begin{block}{Do Now: Angle pair practice. Show steps, including the check.}
      %\begin{enumerate}
          %\item
      %\end{enumerate}
    \end{block}
    Binder project: Exam, corrections; best of each construction
    Review test corrections problems, handout\\
    Deltamath review, Skedula check (HW)\\
    Midpoint calculations: average method, vector method
    \vspace{0.5cm}
    Homework: Pretest, Skedula check in notebook
  }

  \frame
    {
      \frametitle{GQ: How do we calculate the midpoint of a segment?}
      \framesubtitle{CCSS: HSG.CO.A.1 Know precise geometric definitions \hspace{\stretch{1}} \alert{1b.6}}

      \begin{block}{DeltaMath review}
        \begin{enumerate}
            \item Credit for working on DeltaMath after school: Guadalupe, Crismeiris, Janessa, Janicksa, Jesus
            \item Finding slope graphically
            \item Constructing an Equilateral Triangle
        \end{enumerate}
      \end{block}
      Hint: Start tasks in order\\
      Complete only the required number of a problem type\\
      If making multiple errors, move to next task type\\
      \vspace{0.5cm}
      Let me know if you can not access at home
    }

\section{1b.7 Drui: Deltamath. Tuesday Oct 2}
  \frame
  {
    \frametitle{GQ: How do we practice our geometry skills?}
    \framesubtitle{CCSS: HSG.CO.A.1 Know precise geometric definitions \hspace{\stretch{1}} \alert{1b.7}}

    \begin{block}{Do Now: Laptop setup}
    \begin{enumerate}
        \item Use your assigned laptop number
        \item ``Lids down" for group focus
        \item Return laptops to proper slot number, charging cable
    \end{enumerate}
    \end{block}
    Deltamath geometry notation; \\Challenge Geogebra construction\\
    \vspace{1cm}
    Homework: Complete Deltamath problems
  }

\section{1b.8 Drui: Midpoints. Wednesday Oct 3}
  \frame
  {
    \frametitle{GQ: How do we calculate the midpoint of a segment?}
    \framesubtitle{CCSS: HSG.CO.A.1 Know precise geometric definitions \hspace{\stretch{1}} \alert{1b.8}}

    \begin{block}{Do Now: Mixed Review \#39-47 p. 48. Loose leaf paper}
      %\begin{enumerate}
          %\item
      %\end{enumerate}
    \end{block}
    Binder project: Exam, corrections; best of each construction\\
    Review test corrections problems, handouts\\
    Midpoint calculations p. 50-51, \#6-21 p. 54\\
    \vspace{0.5cm}
    Homework: Pretest, Skedula check in notebook
  }

\section{1b.9 Drui: Review. Thursday Oct 4}
\frame
{
  \frametitle{GQ: How do we review for a test?}
  \framesubtitle{CCSS: HSG.CO.A.1 Know precise geometric definitions \hspace{\stretch{1}} \alert{1b.9}}

  \begin{block}{Do Now: Mixed Review \#65-68 p. 56. Loose leaf paper}
    %\begin{enumerate}
        %\item 
    %\end{enumerate}
  \end{block}
  Pretest review\\
  \vspace{0.5cm}
  Homework: Study for \alert{exam tomorrow}
}
  %Study for \alert{exam tomorrow}

\section{1b.10 Drui: Test. Friday Oct 5}
        \frame
        {
          \frametitle{GQ: How do we use the tools of geometry?}
          \framesubtitle{CCSS: HSG.CO.A.1 Know precise geometric definitions \hspace{\stretch{1}} \alert{1.10}}

          \begin{block}{Do Now: (Test)}
          %\begin{enumerate}
              %\item Use your assigned laptop number
          %\end{enumerate}
          \end{block}
          Test\\
          \vspace{1cm}
          Homework: Angle measure algebra problems
        }

\end{document}



  \frame
  {
    \frametitle{A step-by-step guide to solving geometry problems}
    \framesubtitle{Segment addition postulate \hspace{\stretch{1}} \alert{1.4}}
    Given $\overline{ABC}$, $AB=3x-7$, $BC=x+5$, $AC=14$. Find ${AB}$.\\[0.5in]
       \begin{tikzpicture}
        \draw [-, thick] (0,0)--(7,0);
        \draw [fill] (0,0) circle [radius=0.05] node[below]{$A$};
        \draw [fill] (3,0) circle [radius=0.05] node[below]{$B$};
        \draw [fill] (7,0) circle [radius=0.05] node[below]{$C$};
      \end{tikzpicture} \vspace{1cm}
  \begin{enumerate}
      \item<2-> Sketch and label the situation\\
      \item<2-> Write a geometric equation\\
      \item<2-> Substitute algebraic values\\
      \item<2-> Solve for the unknown\\
      \item<2-> Answer the question\\
      \item<2-> Check your answer
    \end{enumerate}
  }
