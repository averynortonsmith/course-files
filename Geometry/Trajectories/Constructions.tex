\documentclass[12pt, oneside]{article}
\usepackage[letterpaper, margin=1in, headsep=0.5in]{geometry}
\usepackage[english]{babel}
\usepackage[utf8]{inputenc}
\usepackage{amsmath}
\usepackage{amsfonts}
\usepackage{amssymb}
\usepackage{tikz}
\usetikzlibrary{quotes, angles}
\usepackage{graphicx}
%\usepackage{pgfplots}
%\pgfplotsset{width=10cm,compat=1.9}
%\usepgfplotslibrary{statistics}
%\usepackage{pgfplotstable}
%\usepackage{tkz-fct}
%\usepackage{venndiagram}

\usepackage{fancyhdr}
\pagestyle{fancy}
\fancyhf{}
\rhead{\thepage \\Name: \hspace{1.5in}.\\}
\lhead{BECA / Dr. Huson / Geometry 10th Grade\\* Learning trajectory: Constructions}

\renewcommand{\headrulewidth}{0pt}

\begin{document}
\subsubsection*{Classical constructions}
\begin{enumerate}
  \item Elementary, single constuctions
  \begin{enumerate}
    \item Equilateral Triangle
    \item Duplicate a line segment
    \item Perpendicular (bisector, through a point on/off the line)
    \item Bisect an angle
    \item Duplicate an angle
    \end{enumerate}
  \item Triangle centers (perpendicular, bisectors, altitudes, medians)
  \item Hexagon and square inscribed in a circle
  \item Triangle midline
  \end{enumerate}

\begin{enumerate}
\subsubsection*{Equilateral triangle}
  \item Given

\subsubsection*{Triangle centers}

  \item Construct a perpendicular to $\overline{AB}$ though $C$.\\
    %\hspace{1cm} Given the line  $l$ and point $P$.
    \vspace{2cm}
    \begin{center}
    \begin{tikzpicture}
      \draw [<->, thick] (0,0)--(11,0)--(6,4)--cycle;
      \draw [fill] (0,0) circle [radius=0.05] node[left]{$A$};
      \draw [fill] (11,0) circle [radius=0.05] node[right]{$B$};
      \draw [fill] (6,4) circle [radius=0.05] node[above right]{$C$};
    \end{tikzpicture}
    \end{center}

  \item Construct the midpoint $M$ of $\overline{BC}$ by using the perpendicular bisector construction. Draw $\overline{AM}$, a \emph{median} of $\triangle ABC$.\\
  Spicy: Construct the other two medians, and hence, the centroid.
    \vspace{1cm}
    \begin{center}
    \begin{tikzpicture}
      \draw [<->, thick] (0,0)--(7,0)--(6,4)--cycle;
      \draw [fill] (0,0) circle [radius=0.05] node[left]{$A$};
      \draw [fill] (7,0) circle [radius=0.05] node[right]{$B$};
      \draw [fill] (6,4) circle [radius=0.05] node[above right]{$C$};
    \end{tikzpicture}
  \end{center} \vspace{1.5cm}


  \end{enumerate}
\end{document}
