\documentclass[12pt, oneside]{article}
\usepackage[letterpaper, margin=1in, headsep=0.5in]{geometry}
\usepackage[english]{babel}
\usepackage[utf8]{inputenc}
\usepackage{amsmath}
\usepackage{amsfonts}
\usepackage{amssymb}
\usepackage{tikz}
\usetikzlibrary{quotes, angles}
\usepackage{graphicx}
%\usepackage{pgfplots}
%\pgfplotsset{width=10cm,compat=1.9}
%\usepgfplotslibrary{statistics}
%\usepackage{pgfplotstable}
%\usepackage{tkz-fct}
%\usepackage{venndiagram}

\usepackage{fancyhdr}
\pagestyle{fancy}
\fancyhf{}
\rhead{\thepage \\Name: \hspace{1.5in}.\\}
\lhead{BECA / Dr. Huson / Geometry 10th Grade\\* Learning trajectory: Area, perimeter, volume}

\renewcommand{\headrulewidth}{0pt}

\begin{document}
\subsubsection*{Area, perimeter, volume}
  \begin{enumerate}
  \item Rectangle, square area and perimeter
  \item Circle area and circumference
  \item Sector areas, arc length
  \item Solve for parameter versus calculate result
  \item Compound shapes
  \item Distance on the coordinate plane
  \begin{enumerate}
    \item Plotting, labeling points, etc.
    \item Horizontal \& vertical distances
    \item Pythagorean formula
    \item Radicals, \pi and rounding
    \end{enumerate}
  \item Triangle area, perimeter (formula sheet)
  \item Volume: prism, cylinder, cone
  \item Surface area
  \item Scaling shapes (eg. rectangle, triangles)
  \end{enumerate}

  \begin{enumerate}
    \subsubsection*{Basic shapes}
    \item


  \end{enumerate}

\end{document}
