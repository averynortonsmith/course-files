\documentclass[12pt, twoside]{article}
\usepackage[letterpaper, margin=1in, headsep=0.5in]{geometry}
\usepackage[english]{babel}
\usepackage[utf8]{inputenc}
\usepackage{amsmath}
\usepackage{amsfonts}
\usepackage{amssymb}
\usepackage{tikz}
%\usetikzlibrary{quotes, angles}

\usepackage{graphicx}
\usepackage{enumitem}
\usepackage{multicol}

\usepackage{fancyhdr}
\pagestyle{fancy}
\fancyhf{}
\renewcommand{\headrulewidth}{0pt} % disable the underline of the header

\fancyhead[R]{\thepage}
\fancyhead[L]{BECA / Dr. Huson / 10th Grade Geometry\\* Learning trajectory: Transversals and parallel lines}

\begin{document}
\subsubsection*{Transversals and parallel lines}
  \begin{enumerate}
  \item Corresponding angles
  \item Alternate interior angles
  %\item Situations and configurations
%  \begin{enumerate}
%    \item Triangle $180^\circ$ sum of internal angles
%    \item Parallelograms in both directions
%    \item Triangle midlines
%    \end{enumerate}
  \end{enumerate}

  \begin{enumerate}
    %\subsubsection*{Corresponding angles}
    \item Given two parallel lines and a transversal, as shown. Apply the theorem ``If a transversal intersects two parallel lines, then corresponding angles are congruent."
      \begin{center}
      \begin{tikzpicture}
        \draw [<->, thick] (1,2)--(9,2);
        \draw [<->, thick] (0,0)--(8,0);
        \draw [<->, thick] (4,-1)--(5.5,3);
        \node at (4.5,0.3) [left]{$5$};
        \node at (4.5,0.3) [right]{$6$};
        \node at (4.3,-0.3) [left]{$7$};
        \node at (4.3,-0.3) [right]{$8$};
        \node at (5.2,2) [above left]{$1$};
        \node at (5.2,2) [above right]{$2$};
        \node at (5,2) [below left]{$3$};
        \node at (5,2) [below right]{$4$};
      \end{tikzpicture}
      \end{center}
      \begin{enumerate}
        \item State the angle corresponding with $\angle 8$. \bigskip
        \item Given $m\angle 6 = 81^\circ$ and $m\angle 2 = 3x^\circ$. Find $x$. \bigskip
        \item Given $m\angle 5 = 99^\circ$. Find $m\angle 3$. \bigskip
        \item In a proof, what reason would justify $\angle 4 \cong \angle 5$? \rule{6cm}{0.15mm}
      \end{enumerate}

    \item Given two parallel lines and a transversal, as shown. $m\angle 2=5x$ and $m\angle 5=6x+15$. Find $m\angle 5$.
        \begin{center}
        \begin{tikzpicture}
          \draw [<->, thick] (1,2)--(9,2);
          \draw [<->, thick] (0,0)--(8,0);
          \draw [<->, thick] (4,-1)--(5.5,3);
          \node at (4.5,0.3) [left]{$5$};
          %\node at (4.5,0.3) [right]{$6$};
          %\node at (4.3,-0.3) [left]{$7$};
          %\node at (4.3,-0.3) [right]{$8$};
          %\node at (5.2,2) [above left]{$1$};
          \node at (5.2,2) [above right]{$2$};
          \node at (5,2) [below left]{$3$};
          %\node at (5,2) [below right]{$4$};
        \end{tikzpicture}
        \end{center}

\newpage
    \item Spicy: Construct a duplicate of $\angle ABC$, with $A$ as the vertex and with one leg parallel to $\overleftrightarrow{BC}$, as shown by the dotted line. (Leave all construction marks.)
    \vspace{2cm}
      \begin{center}
      \begin{tikzpicture}[scale=1.4]
        \draw [<->, thick] (-1,0)--(0,0)--(4,0);
        \draw [<->, thick] (-1,-1)--(5,5);
        \draw [->, thick, dashed] (2.5,2.5)--(5, 2.5); %node[below right]{$E$};
        \draw [fill] (2.5,2.5) circle [radius=0.05] node[above left]{$A$};
        \draw [fill] (0,0) circle [radius=0.05] node[above left]{$B$};
        \draw [fill] (3,0) circle [radius=0.05] node[above left]{$C$};
        %\draw [fill] (7,0) circle [radius=0.05] node[above right]{$C$};
      \end{tikzpicture}
      \end{center}
      \vspace{1cm}
      Explain why the constructed leg is parallel to $\overleftrightarrow{BC}$.

  \end{enumerate}

\end{document}
