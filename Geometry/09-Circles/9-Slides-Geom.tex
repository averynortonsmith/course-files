\documentclass{beamer}
\usepackage{geometry}
\usepackage[english]{babel}
\usepackage[utf8]{inputenc}
\usepackage{amsmath}
\usepackage{amsfonts}
\usepackage{amssymb}
\usepackage{tikz}
\usetikzlibrary{quotes, angles}
\usepackage{graphicx}

%\usepackage{pgfplots}
%\pgfplotsset{width=10cm,compat=1.9}
%\usepackage{pgfplotstable}

\setlength{\headheight}{26pt}%doesn't seem to fix warning

\usepackage{fancyhdr}
\pagestyle{fancy}
\fancyhf{}

%\rhead{\small{24 March 2019}}
\lhead{\small{BECA / Dr. Huson / Geometry - Unit 9 Angle Relationships}}

\renewcommand{\headrulewidth}{0pt}

\title{10th Grade Geometry - Unit 9 Angle Relationships}
\subtitle{Bronx Early College Academy}
\author{Christopher J. Huson PhD}
\date{25 March 2019}

\begin{document}
\frame{\titlepage}
\section[Outline]{}
\frame{\tableofcontents}


\section{9.1 Internal \& external triangle angles, transversals Monday 25 March}
  \frame
  {
    \frametitle{GQ: How do we name the angles of a transversal intersecting parallels?}
    \framesubtitle{CCSS: HSG.CO.D.12 Congruence, geometric constructions \hfill \alert{9.2 Monday 25 March}}

    \begin{block}{Do Now Triangle angles handout}
      \begin{enumerate}
        \item Internal \& external angles sums of a triangle
        \item Solving algebra embedded in geometry situations
        \item Vertical, complementary, and supplementary relationships
      \end{enumerate}
    \end{block}
    Lesson: Common angular situations: intersections, triangles, transversals\\
    Homework: Practice problems handout
  }

\section{9.2 Geogebra - Common angular situations: triangles, transversals Tuesday 26 March}
  \frame
  {
    \frametitle{GQ: How do we use technology to explore geometry?}
    \framesubtitle{CCSS: MP5 Use appropriate tools strategically: dynamic geometry software \hfill \alert{9.2 Tuesday 26 March}}

    \begin{block}{Project: Polygon sum of internal angles theorem}
      \begin{enumerate}
        \item Write a paper illustrating why a polygon's internal angles sum to $S=(n-2)180^\circ$, where $n$ is the number of sides
        \item Spicy: Use color \& line variations for clarity (not decoration)
        \item Construct in Geogebra, compile in Word: add heading \& title, text, and formulas using Microsoft's equation editor
        \item Email me: Last-Title.pdf, with subject line \& message
        \item Rubric: correct, aesthetics, MLA \& email standards
      \end{enumerate}
    \end{block}
    SAT tomorrow, \alert{Test corrections due Thursday}\\
    Homework: Complete project (due by 10:00 pm), problem set
  }

\section{9.3 Isosceles triangles Thursday 28 March}
  \frame
  {
    \frametitle{GQ: How do we analyze isosceles triangles in situations?}
    \framesubtitle{CCSS: HSG.CO.D.12 Congruence, geometric constructions \hfill \alert{9.3 Thursday 28 March}}

      \begin{block}{Do Now Triangle angles handout}
        \begin{enumerate}
          \item Internal \& external angles sums of a triangle
          \item Vertical, complementary, and supplementary relationships
          \item Solving algebra embedded in geometry situations
        \end{enumerate}
      \end{block}
    Lesson: Isosceles base angle theorem, circle radii\\
    Homework: Practice problems handout
  }

\section{9.4 Circle angles Friday 29 March}
  \frame
  {
    \frametitle{GQ: How do we name the angles within a circle?}
    \framesubtitle{CCSS: HSG.CO.D.12 Congruence, geometric constructions \hfill \alert{9.4 Friday 29 March}}

      \begin{block}{Do Now Quiz: Isosceles triangles handout}
        \begin{enumerate}
          \item Isosceles base angle theorem
          \item External angle theorem applications
          \item Dilation \& similar triangle review
          \item Solving algebra embedded in geometry situations
        \end{enumerate}
      \end{block}
    Lesson: Central angle measures, included half-angle theorem\\
    Assessment: \alert{Mock Regents Tuesday}\\
    Homework: Practice problems handout
  }

\section{9.5 Prep for Mock Regents Monday 1 April}
\frame
{
  \frametitle{GQ: How do we calculate the angles of secant and chord intersections?}
  \framesubtitle{CCSS: HSG.CO.D.12 Congruence, geometric constructions \hfill \alert{9.5 Monday 1 April}}

    \begin{block}{Do Now: Inscribed and central angles handout}
      \begin{enumerate}
        \item Central angle measures
        \item Included (half-angle) theorem
      \end{enumerate}
    \end{block}
  Lesson: Chord and secant angle versus arc meaure theorems\\
  Homework: Study for \alert{Mock Regents tomorrow}
}

\section{9.6 Sector areas, arc lengths Thursday 4 April}
  \frame
  {
    \frametitle{GQ: How do we calculate the measures of part of a circle?}
    \framesubtitle{CCSS: HSG.CO.D.12 Congruence, geometric constructions \hfill \alert{9.6 Thursday 4 April}}

    \begin{block}{Do Now: Angle relationships handout}
      \begin{enumerate}
        \item Area calculations
        \item Circle angle measures
        \item Scale factor
      \end{enumerate}
    \end{block}
    Homework review: Circle angle formulas\\
    Lesson: Portions of a circle, sector areas \& arc lengths\\[0.5cm]
    Homework: Practice problems handout
  }

\section{9.7 Volume situations Friday 5 April}
  \frame
  {
    \frametitle{GQ: How do we apply angle relationships?}
    \framesubtitle{CCSS: HSG.CO.D.12 Congruence, geometric constructions \hfill \alert{9.7 Friday 5 April}}

    \begin{block}{Do Now: Angle relationships handout}
      \begin{enumerate}
        \item Area calculations
        \item Circle angle measures
        \item Scale factor
      \end{enumerate}
    \end{block}
    Lesson: Portions of a circle, sector areas \& arc lengths\\
    Substitute coverage for 10.1 \\[0.5cm]
    Homework: Practice problems handout
  }

\section{9.8 Sectors, arcs, and chords Monday 8 April}
  \frame
  {
    \frametitle{GQ: How do we apply angle relationships?}
    \framesubtitle{CCSS: HSG.CO.D.12 Congruence, geometric constructions \hfill \alert{9.8 Monday 8 April}}

    \begin{block}{Do Now Similar triangle handout}
      \begin{enumerate}
        \item Naming corresponding relationships
        \item Determining equal ratios (to scale factor)
        \item Applying similarity relationships in situations
      \end{enumerate}
    \end{block}
    Lesson: Substitute coverage\\
    Homework: Practice problems handout
  }

\section{9.9 Deltamath practice Tuesday 9 April}
  \frame
  {
    \frametitle{GQ: How do we apply angle relationships?}
    \framesubtitle{CCSS: HSG.CO.D.12 Congruence, geometric constructions \hfill \alert{9.9 Tuesday 9 April}}

    \begin{block}{Do Now: Deltamath problem set}
      \begin{enumerate}
        \item Naming corresponding relationships
        \item Determining equal ratios (to scale factor)
        \item Applying similarity relationships in situations
      \end{enumerate}
    \end{block}
    Lesson: Dilation, proportion, and similarity\\
    Homework: Complete Deltamath assignment
  }

\section{9.10 Compound shape areas and volumes Wednesday 10 April}
  \frame
  {
    \frametitle{GQ: How do we combine simple shapes?}
    \framesubtitle{CCSS: HSG.CO.D.12 Congruence, geometric constructions \hfill \alert{9.10 Wednesday 10 April}}

    \begin{block}{Do Now: Sectors, secants, and chords handout}
      \begin{enumerate}
        \item Naming corresponding relationships
        \item Determining equal ratios (to scale factor)
        \item Applying similarity relationships in situations
        \item Secant and chord angle relationships to arc measures
      \end{enumerate}
    \end{block}
    Lesson: Compound shape areas and volumes\\
    Assessment: Circles quiz Friday\\
    Homework: Practice problems handout
  }

\section{9.11 Compound shape areas and volumes Thursday 11 April}
  \frame
  {
    \frametitle{GQ: How do we combine simple shapes?}
    \framesubtitle{CCSS: HSG.CO.D.12 Congruence, geometric constructions \hfill \alert{9.11 Thursday 11 April}}

    \begin{block}{Do Now: Sectors, secants, and chords handout}
      \begin{enumerate}
        \item Naming corresponding relationships
        \item Determining equal ratios (to scale factor)
        \item Applying similarity relationships in situations
        \item Secant and chord angle relationships to arc measures
      \end{enumerate}
    \end{block}
    Lesson: Compound shape areas and volumes\\
    Assessment: Circles quiz \alert{tomorrow}\\
    Homework: Practice problems handout
  }

\section{9.12 Compound shape areas and volumes Friday 12 April}
  \frame
  {
    \frametitle{GQ: How do we combine simple shapes?}
    \framesubtitle{CCSS: HSG.CO.D.12 Congruence, geometric constructions \hfill \alert{9.12 Friday 12 April}}

    \begin{block}{Do Now: Sectors, secants, and chords handout}
      \begin{enumerate}
        \item Naming corresponding relationships
        \item Determining equal ratios (to scale factor)
        \item Applying similarity relationships in situations
        \item Secant and chord angle relationships to arc measures
      \end{enumerate}
    \end{block}
    Lesson: Compound shape areas and volumes\\
    Assessment: Circles quiz\\
    Homework: Practice problems handout
  }

\end{document}
