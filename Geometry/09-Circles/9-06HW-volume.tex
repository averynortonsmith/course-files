\documentclass[12pt, twoside]{article}
\usepackage[letterpaper, margin=1in, headsep=0.5in]{geometry}
\usepackage[english]{babel}
\usepackage[utf8]{inputenc}
\usepackage{amsmath}
\usepackage{amsfonts}
\usepackage{amssymb}
\usepackage{tikz}
%\usetikzlibrary{quotes, angles}

\usepackage{graphicx}
\usepackage{enumitem}
\usepackage{multicol}

\usepackage{fancyhdr}
\pagestyle{fancy}
\fancyhf{}
\renewcommand{\headrulewidth}{0pt} % disable the underline of the header

\fancyhead[RE]{\thepage}
\fancyhead[RO]{\thepage \\ Name: \hspace{3cm}}
\fancyhead[L]{BECA / Dr. Huson / 10th Grade Geometry\\* 4 April 2019}

\begin{document}
\subsubsection*{Homework: Area and volume calculations}
 \begin{enumerate}

 \item Circle $O$ has a diameter $AB=10$, as shown.
       \begin{center}
       \begin{tikzpicture}[scale=.6]
         \draw (0,0) circle[radius=5];
         \draw [thick]
         (0:5) node[right] {$A$}--(180:5) node[left] {$B$};
         \draw (0,0)--(60:5) node[above right] {$C$};
         \fill (0,0) circle[radius=0.1] node[below]{$O$};
         %\draw (75:1.8) node[above] {$C$};
         %\draw (290:5) node[below] {$D$};
       \end{tikzpicture}
     \end{center}
     \begin{enumerate}
       \item Find the area of circle $O$. \vspace{2.5cm}
       \item Find the perimeter of the semi-circle with diameter $\overline{AB}$, including the length of the diameter. \vspace{2.5cm}
       \item Given $m\angle AOC=60^\circ$. Find the area of the sector $AOC$. \vspace{2.5cm}
       \item Find the perimeter of the sector $AOC$.
     \end{enumerate}

\newpage

\item Find the volume of a pyramid ($V=\frac{1}{3}Bh$) having a height of 14.5 inches and with a square base having side lengths of 15 inches. Express your result to the \emph{nearest cubic inch}. \vspace{5cm}

\item Find the volume of a hemisphere with a radius of 6 inches, to the \emph{nearest whole cubic inch}. (The formula for the volume of a \emph{sphere} is $V=\frac{4}{3}\pi r^3$)  \vspace{5cm}

\item Given a rectangle with area 60, width $x$, and length $x+7$.
  \begin{enumerate}
    \item Find $x$. \vspace{4cm}
    \item Find the perimeter of the rectangle.
  \end{enumerate}

\end{enumerate}
\end{document}
