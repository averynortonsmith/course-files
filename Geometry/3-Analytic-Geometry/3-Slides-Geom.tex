\documentclass{beamer}
\usepackage{geometry}
\usepackage[english]{babel}
\usepackage[utf8]{inputenc}
\usepackage{amsmath}
\usepackage{amsfonts}
\usepackage{amssymb}
\usepackage{tikz}
\usetikzlibrary{quotes, angles}
\usepackage{graphicx}

%\usepackage{pgfplots}
%\pgfplotsset{width=10cm,compat=1.9}
%\usepackage{pgfplotstable}

\setlength{\headheight}{26pt}%doesn't seem to fix warning

\usepackage{fancyhdr}
\pagestyle{fancy}
\fancyhf{}

%\rhead{\small{13 November 2018}}
\lhead{\small{BECA / Dr. Huson / Geometry Unit 3}}

\renewcommand{\headrulewidth}{0pt}

\title{Mathematics Class Slides}
\subtitle{Bronx Early College Academy}
\author{Chris Huson}
\date{13 November 2018}

\begin{document}
\frame{\titlepage}
\section[Outline]{}
\frame{\tableofcontents}

\section{Project criteria}
  \frame
  {
    \frametitle{GQ: How do we present mathematical work?}
    \framesubtitle{CCSS: HSG.CO.D.12 Congruence, Make geometric constructions}

    Complete binder - \alert{project grade}\\
    Exams \& corrections\\
    Best examples of each basic construction:\\
    Equilateral $\triangle$, $\cong$ segment \& $\angle$s, bisected segment \& $\angle$, $\perp$s \\
    $\triangle$ concurrencies, compound constructions
      \begin{block}{Criteria for construction projects}
      \begin{enumerate}
          \item Complete and correct construction
          \item Steps written with proper notation
          \item Layout: GQ title, date on left; first \& last name on right
          \item Precise, elegant, mathematical aesthetic
      \end{enumerate}
      \end{block}
    %Grading policy: full credit 20, minus 2 points for each missing\\[5pt]
  }

\section{Notetaking criteria}
  \frame
  {
    \frametitle{GQ: How do we organize our mathematical notes?}
    \framesubtitle{CCSS: HSG.CO.A.1 Know precise geometric definitions}

    \begin{block}{Criteria for notebook project grade (20 points)}
    \begin{enumerate}
      \item Your name and "Geometry" on cover
      \item Toward front: math.huson.com, husonbeca@gmail.com, 917-648-5632, Deltamath teacher ID: 546068
      \item Labeled composition book out during class; GQ, date each day
      \item Definitions, postulates, constructions, \& theorems
      \item Combination of symbols, diagrams, text (best: your own words)
      \item Examples, but not practice problem sets
    \end{enumerate}
    \end{block}
    Grading policy: daily tracker, pop notebook checks
  }

\section{2.13 Project: Triangle centers project, Wednesday 31 October}
  \frame
  {
    \frametitle{GQ: How do we construct the centroid, circumcenter, incenter, and orthocenter?}
    \framesubtitle{CCSS: HSG.CO.C.9 Prove geometric theorems \hspace{\stretch{1}} \alert{2-13}}

    \begin{block}{Construction project: Triangle centers}
    \begin{enumerate}
        \item Circumcenter: perpendicular bisectors
        \item Incenter: angle bisectors
        \item Orthocenter: altitudes (perpendiculars through vertices)
        \item Centroid: medians (midpoint to opposite vertices)
    \end{enumerate}
    \end{block}
    Was due Monday November 5th
    }

\section{3.1 Drui: Deltamath. Tuesday 16 October}
  \frame
  {
    \frametitle{GQ: How do we use slope in geometry?}
    \framesubtitle{CCSS: HSG.CO.D.12 Congruence, Make geometric constructions \hspace{\stretch{1}} \alert{3-1 Tuesday Nov 13}}

    \begin{block}{Today's class assignments, in order}
    \begin{enumerate}
        \item Triangle center project (over due)
        \item Write a binder checklist: exams, constructions, projects
        \item Deltamath practice: slope, parallels, perpendiculars, $\triangle$ sums
    \end{enumerate}
    \end{block}

    Notebook check \\ \bigskip
    Test corrections due Friday\\ \bigskip
    Homework: Complete deltamath (10pm deadline)
  }


\section{3.2 Drui: Isosceles. Wednesday 14 November}
  \frame
  {
    \frametitle{GQ: How do we use isosceles triangles?}
    \framesubtitle{CCSS: HSG.CO.C.9 Prove geometric theorems \hspace{\stretch{1}} \alert{3-2 Wednesday Nov 14}}

    \begin{block}{Do Now: Sketch $\triangle ABC$, $A(-2,-1), B(2,-1), C(2,2)$}
      \begin{enumerate}
          \item Find the slope of $\overleftrightarrow{AC}$
          \item Find the lengths $AB, BC, AC$
          \item Given $m\angle A=37$, $m\angle B=90$. Find $m\angle C$
      \end{enumerate}
    \end{block}
    Theorems: \\
    A triangle is isosceles \emph{iff} it has two congruent base angles \\
    Radii of a circle, and congruent circles, are congruent\\[0.2cm]
    Homework: Triangle and slope practice, handout
    }

\section{3.2 Drui: Isosceles. Wednesday 14 November}
  \frame
  {
    \frametitle{The isosceles base angle theorem.}
    %\framesubtitle{CCSS: HSG.CO.C.9 Prove geometric theorems \hspace{\stretch{1}} \alert{3-2 Wednesday Nov 14}}

  Given $\triangle ABC$. $\overline{AC} \cong \overline{BC}$ \emph{iff} $\angle A \cong \angle B$.\\[0.5cm]
  \begin{columns}
    \column{5cm}
    \begin{tikzpicture}[scale=0.7]
      \draw [thick](0,0)--(4,0)--(2,6)--(0,0);
      \draw [fill] (0,0) circle [radius=0.05] node[below]{$A$};
      \draw [fill] (4,0) circle [radius=0.05] node[below]{$B$};
      \draw [fill] (2,6) circle [radius=0.05] node[above right]{$C$};
      \draw [color=blue] (0,0) ++(0.75,0) arc [start angle=0, end angle=70, radius=0.75];
      \draw [color=blue] (4,0) ++(-0.22, 0.73) arc [start angle=110, end angle=180, radius=0.75];
      \draw [thick] (0.8,3.1)--(1.2,2.9); %tick mark
      \draw [thick] (2.8,2.9)--(3.2,3.1); %tick mark
      %\node [right] at (3.25,2.5){$x+7$};
      %\node [left] at (0.75,2.5){$2x+1$};
    \end{tikzpicture}
    \column{4cm}
      The two congruent angles are the \emph{base} angles. The third angle is the \emph{vertex} angle.
  \end{columns}
  }

  \section{3.3 Drui: Isosceles. Thursday 15 November}
    \frame
    {
      \frametitle{GQ: How do we calculate the area of a parallelogram?}
      \framesubtitle{CCSS: HSG.GPE.B.7 Use coordinates to compute perimeters \& areas of polygons \hspace{\stretch{1}} \alert{3-3 Thursday 15 November}}

      \begin{block}{Do Now: $\triangle$ center construction handout}
        \begin{enumerate}
            \item Altitude, orthocenter, spicy: hexagon
        \end{enumerate}
      \end{block}
      Lesson: \\
      The area of a parallelogram equals base times height. $A=b \times h$ \\[0.2cm]
      Aassessment: \\
      Isosceles triangle  and circle radii\\[0.2cm]
      Homework: Area and distance review, handout
    }

  \section{3.4 Drui: Isosceles. Friday 16 November}
    \frame
    {
      \frametitle{GQ: How do we calculate the area of a parallelogram?}
      \framesubtitle{CCSS: HSG.GPE.B.7 Use coordinates to compute perimeters \& areas of polygons \hspace{\stretch{1}} \alert{3-4 Friday 16 November}}

      \begin{block}{Do Now: Given parallelogram $SNOW$ with $S(2,1),N(7,1),O(10,5),W(5,5)$, handout}
      \end{block}
      Lesson: \\
      Applying distance, midpoint, slope , and angle congruence formulas to parallelograms\\[0.5cm]
      Homework: Linear functions review, handout
    }

    \section{3.4 Drui: Isosceles. Friday 16 November}
      \frame
      {
        \frametitle{Features of parallelograms (and rhombuses)}
        %\framesubtitle{CCSS: HSG.GPE.B.7 Use coordinates to compute perimeters \& areas of polygons \hspace{\stretch{1}} \alert{3-4}}

      Parallelogram $SNOW$ with $S(2,1),N(7,1),O(10,5),W(5,5)$\\[0.5cm]
        \begin{tikzpicture}[scale=1.2]
          \draw [thick] (2,1)--(7,1)--(10,5)--(5,5)--(2,1);
          \draw [dashed] (2,1)--(10,5);
          \draw [dashed] (7,1)--(5,5);
          \draw [fill] (2,1) circle [radius=0.05] node[below]{$S$};
          \draw [fill] (7,1) circle [radius=0.05] node[below]{$N$};
          \draw [fill] (10,5) circle [radius=0.05] node[above right]{$O$};
          \draw [fill] (5,5) circle [radius=0.05] node[above left]{$W$};
          \draw [fill] (6,3) circle [radius=0.05] node[right]{$P$};
        \end{tikzpicture}
      }

    \section{3.5 Drui: Triangle external angle theorem. Monday 19 November}
      \frame
      {
        \frametitle{GQ: How do we write the equation of a line?}
        \framesubtitle{CCSS: HSG.GPE.B.7 Use coordinates to compute perimeters \& areas of polygons \hspace{\stretch{1}} \alert{3-5 Monday 19 November}}

        \begin{block}{Do Now: Textbook triangle problems}
          Exercises \#25-29 p. 209
        \end{block}
        Lesson: \\
        Triangle external angle theorem \\
        Slope-intercept form of the equation of a line\\
        Exercises: Chapter Test questions p.211 \\[0.5cm]
        Homework: Linear functions review, handout
      }


    \section{3.6 Drui: Deltamath. Tuesday 20 November}
      \frame
      {
        \frametitle{GQ: How do we use trigonometric ratios?}
        \framesubtitle{CCSS: HSG.CO.D.12 Congruence, geometric constructions \hspace{\stretch{1}} \alert{3-6 Tuesday 20 November}}

        \begin{block}{Write in your notebook: Trig ratios, "SOH-CAH-TOA"}
        \begin{enumerate}
            \item sine, SOH: $\displaystyle \sin x = \frac{\text{opposite}} {\text{hypotenuse}}$
            \item cosine, CAH: $\displaystyle \cos x = \frac{\text{adjacent}} {\text{hypotenuse}}$
            \item tangent, TOA: $\displaystyle \tan x = \frac{\text{opposite}} {\text{adjacent}}$
        \end{enumerate}
        \end{block}
        Classwork priorities
        \begin{enumerate}
            \item  Triangle center project
            \item  Deltamath practice: trig, parallels, perpendiculars, $\triangle$ sums
        \end{enumerate}
        Homework: Complete deltamath (10pm deadline)
      }


    \section{3.7 Drui: Trig. Wednesday 26 November}
      \frame
      {
        \frametitle{GQ: How do we use trig ratios to solve problems?}
        \framesubtitle{CCSS: HSG.SRT.C.8 Use trig ratios to solve problems \hspace{\stretch{1}} \alert{3-7 Wednesday 19 November}}

        \begin{block}{Do Now: Trig problems, handout}
          %\begin{enumerate}
            %\item Given $A(2,1),B(6,4)$. Find the length $AB$ and the slope $m_{\overline{AB}}$
          %\end{enumerate}
        \end{block}
        Lesson: \\
        Right triangle trigonomy examples\\[0.5cm]
        Homework: Review packet
      }

\end{document}

Lesson: \\
Right triangle example proof \\
Perpendicular bisector example\\
Triangle midline proof\\[0.5cm]
