\documentclass{beamer}
\usepackage{geometry}
\usepackage[english]{babel}
\usepackage[utf8]{inputenc}
\usepackage{amsmath}
\usepackage{amsfonts}
\usepackage{amssymb}
\usepackage{tikz}
\usetikzlibrary{quotes, angles}
\usepackage{graphicx}

%\usepackage{pgfplots}
%\pgfplotsset{width=10cm,compat=1.9}
%\usepackage{pgfplotstable}

\setlength{\headheight}{26pt}%doesn't seem to fix warning

\usepackage{fancyhdr}
\pagestyle{fancy}
\fancyhf{}

%\rhead{\small{10 September 2018}}
\lhead{\small{BECA / Dr. Huson / Geometry Unit 1}}

\renewcommand{\headrulewidth}{0pt}

\title{Mathematics Class Slides}
\subtitle{Bronx Early College Academy}
\author{Chris Huson}
\date{24 September 2018}

\begin{document}
\frame{\titlepage}
\section[Outline]{}
\frame{\tableofcontents}

\section{1b.0 Project criteria}
  \frame
  {
    \frametitle{GQ: How do we present mathematical work?}
    \framesubtitle{CCSS: HSG.CO.D.12 Congruence, Make geometric constructions \hspace{\stretch{1}} \alert{1b.0}}

    \begin{block}{Criteria for construction projects}
    \begin{enumerate}
        \item Complete and correct construction
        \item Steps written with proper notation
        \item Layout: GQ title, date on left; first, last name on right
        \item Precise, elegant, mathematical aesthetic
    \end{enumerate}
    \end{block}
    Grading policy: full credit (20 out of 20) or redo\\[5pt]
    Constructions: \alert{Due Friday}\\
    Congruent segment \& angles, bisected segment \& angle \\
    \alert{Due Friday}
    (collect exams and projects in classroom binder)
  }

\section{1b.0 Notetaking criteria}
  \frame
  {
    \frametitle{GQ: How do we organize our mathematical notes?}
    \framesubtitle{CCSS: HSG.CO.A.1 Know precise geometric definitions \hspace{\stretch{1}} \alert{1b.0}}

    \begin{block}{Criteria for notebook project grade (20 points)}
    \begin{enumerate}
        \item Labeled composition book out during class; GQ, date each day
        \item Definitions, postulates, constructions, \& theorems
        \item Combination of symbols, diagrams, text (best if in your own words)
        \item Examples, but not practice problem sets
    \end{enumerate}
    \end{block}
    Grading policy: daily tracker, pop notebook checks (Wednesday)
  }

  \section{1b.1 Drui}
    \frame
    {
      \frametitle{GQ: How do we classify angle pairs?}
      \framesubtitle{CCSS: HSG.CO.A.1 Know precise geometric definitions \hspace{\stretch{1}} \alert{1b.1}}

      \begin{block}{Do Now: Practice and review}
      \begin{enumerate}
          \item Segment addition problem
          \item Equilateral triangle construction
          \item Segment replication
      \end{enumerate}
      \end{block}
      Construct a congruent line segment \\
      1-5 Exploring Angle Pairs pp. 34-37\\
      Classwork problems 7-26 odds pp. 38\\
      Review Friday's Exam. Corrections due Wednesday\\
      \vspace{0.5cm}
      Homework: Angle measure algebra problems
    }


\section{1b.2 Drui} %Tuesday 25 Sept - Deltamath
  \frame
  {
    \frametitle{GQ: How do we use geometric notation?}
    \framesubtitle{CCSS: HSG.CO.A.1 Know precise geometric definitions \hspace{\stretch{1}} \alert{1b.2}}

    \begin{block}{Do Now: Laptop setup}
    \begin{enumerate}
        \item Use your assigned laptop number
        \item ``Lids down" for group focus
        \item Return laptops to proper slot number, charging cable
    \end{enumerate}
    \end{block}
    Deltamath geometry notation; \\Challenge Geogebra construction\\
    \vspace{1cm}
    Homework: Angle measure, algebra problems
  }

  \section{1.10 Drui Friday Sept 21}
    \frame
    {
      \frametitle{GQ: How do we use the tools of geometry?}
      \framesubtitle{CCSS: HSG.CO.A.1 Know precise geometric definitions \hspace{\stretch{1}} \alert{1.10}}

      \begin{block}{Do Now: (Test)}
      %\begin{enumerate}
          %\item Use your assigned laptop number
      %\end{enumerate}
      \end{block}
      Test\\
      \vspace{1cm}
      Homework: Angle measure algebra problems
    }

\end{document}


  \frame
  {
    \frametitle{A step-by-step guide to solving geometry problems}
    \framesubtitle{Segment addition postulate \hspace{\stretch{1}} \alert{1.4}}
    Given $\overline{ABC}$, $AB=3x-7$, $BC=x+5$, $AC=14$. Find ${AB}$.\\[0.5in]
       \begin{tikzpicture}
        \draw [-, thick] (0,0)--(7,0);
        \draw [fill] (0,0) circle [radius=0.05] node[below]{$A$};
        \draw [fill] (3,0) circle [radius=0.05] node[below]{$B$};
        \draw [fill] (7,0) circle [radius=0.05] node[below]{$C$};
      \end{tikzpicture} \vspace{1cm}
  \begin{enumerate}
      \item<2-> Sketch and label the situation\\
      \item<2-> Write a geometric equation\\
      \item<2-> Substitute algebraic values\\
      \item<2-> Solve for the unknown\\
      \item<2-> Answer the question\\
      \item<2-> Check your answer
    \end{enumerate}
  }
