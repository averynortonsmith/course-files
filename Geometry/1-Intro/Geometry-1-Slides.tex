\documentclass{beamer}
\usepackage{geometry}
\usepackage[english]{babel}
\usepackage[utf8]{inputenc}
\usepackage{amsmath}
\usepackage{amsfonts}
\usepackage{amssymb}
\usepackage{tikz}
\usepackage{graphicx}
\usepackage{venndiagram}

%\usepackage{pgfplots}
%\pgfplotsset{width=10cm,compat=1.9}
%\usepackage{pgfplotstable}

\setlength{\headheight}{26pt}%doesn't seem to fix warning

\usepackage{fancyhdr}
\pagestyle{fancy}
\fancyhf{}

%\rhead{\small{5 September 2018}}
\lhead{\small{BECA / Dr. Huson / Geometry Unit 1}}

%\vspace{1cm}

\renewcommand{\headrulewidth}{0pt}


\title{Mathematics Class Slides}
\subtitle{Bronx Early College Academy}
\author{Chris Huson}
\date{5-21 September 2018}

\begin{document}

\frame{\titlepage}

%\section[Outline]{}
%\frame{\tableofcontents}

  \section{1.1 Drui}
  \frame
  {
    \frametitle{GQ: How do we define the basic elements of geometry?}
    \framesubtitle{CCSS: HSF.IF.C.7 Analyze functions \qquad \qquad \qquad \alert{1.1}}
    Welcome back to school
    \begin{block}{Do Now Handout: Algebra skills check}
    \begin{enumerate}
        \item Assigned seating: \alert{Without saying a word (!),} arrange yourself alphabetically by last name, left to right, front to back.
        \item Take out notebooks (or blank paper) \& calculator
        \item Complete signin sheet \emph{in order by last name}\\*
    \end{enumerate}
    \end{block}
    Lesson: Definitions: point, line, plane, ray, segment, end point, colinear, coplanar, congruent, distance or length, angle, vertex \\%*[5pt]
    Segment addition postulate (classwork handout)\\
    Early finishers: How should the papers be handed in to be in order?\\
    Homework: Problem set 1-1 Vocabulary and terminology
  }
%Prepare copies of formula sheets

  \section{1.2 Drui}
  \frame
  {
    \frametitle{GQ: How do we construct geometric figures?}
    \framesubtitle{CCSS: HSG.CO.D.12 Congruence, Make geometric constructions \qquad \alert{1.2}}

    \begin{block}{Do Now: Problems 75-80 pg 19}
    %\begin{enumerate}
        %\item
    %\end{enumerate}
    \end{block}
    Homework review\\
    Riddle on page 3\\
    Lesson: Opposite rays, intersection, p. 13-15\\
    Project: Introduction to compass use\\
    Calculator deposits \$20
    \\%*[5pt]
    Homework: Algebra skills assessment
  }

  \section{1.3 Drui}
  \frame
  {
    \frametitle{GQ: How do we construct an equilateral triangle?}
    \framesubtitle{CCSS: HSG.CO.D.12 Congruence, Make geometric constructions \qquad \alert{1.3}}

    \begin{block}{Do Now Quiz}
    \begin{enumerate}
        \item Notation and terminology
    \end{enumerate}
    \end{block}
    Lesson: Circle notation; ``Sketch", ``draw", ``construct"; ``Given"\\[5pt]
    Euclid's first construction
    \begin{enumerate}
        \item Steps in the construction
        \item Logic: Why does it work?
        \item Assessment criteria: precision, correct \& complete, elegance or beauty
    \end{enumerate}
    \\%*[5pt]
    Homework: Geometry and algebra practice\\
    Due: notebook, folder, compass, ruler, protractor, calculator
  }

  \section{1.4 Drui}
  \frame
  {
    \frametitle{GQ: How do we graph quadratics?}
    \framesubtitle{CCSS: HSG.CO.D.12 Congruence, Make geometric constructions  \qquad \alert{1.4}}

    \begin{block}{Do Now: Factoring}
    \begin{enumerate}
        \item
        \item
    \end{enumerate}
    \end{block}
    Lesson: $x-$ and $y-$intercepts, vertex, axis of symmetry, discriminant, odd, even functions
    \\%*[5pt]
    Homework: Quadratics graphing
  }

  \frame
  {
    \frametitle{GQ: How do we graph quadratics?}
    \framesubtitle{CCSS: HSG.CO.D.12 Congruence, Make geometric constructions  \qquad \alert{1.4}}

    \begin{block}{Symmetry of functions, graphically and algebraically}
    \begin{enumerate}
        \item Definition: a function is \emph{even} if it can be reflected onto itself across the $y-$axix. Equivalently, iff $f(x)=f(-x)$
        \item Definition: a function is \emph{odd} if it can be reflected onto itself across the origin. Equivalently, iff $f(x)=-f(-x)$
    \end{enumerate}
    \end{block}
    %graph examples
  }

  \frame
  {
    \frametitle{How do we graph quadratics?}
    \framesubtitle{CCSS: HSG.CO.D.12 Congruence, Make geometric constructions  \qquad \alert{1.4}}

    \begin{block}{Consider the function $f(x)=-x^2+2x+3$}
    \begin{enumerate}
        \item Factor $f$ and state its zeros.
        \item Restate $f$ in vertex form. Write down the vertex as an ordered pair.
        \item Over what intervals is the function increasing, decreasing, and neither?
        \item If $f(x)$ represents the height of a diver over the domain $0 \leq x \leq 3$, interpret $f(0)$, the vertex, and $f(3)$
        \item What does the "slope" of the curve represent?
    \end{enumerate}
    \end{block}
  }


  \section{1.5 Drui}
  \frame
  {
    \frametitle{GQ: How do we calculate rates in context?}
    \framesubtitle{CCSS: HSG.CO.D.12 Congruence, Make geometric constructions  \qquad \alert{1.5}}

    \begin{block}{Do Now: Sketch the function $f(x)=x^3-9x$}
      \begin{enumerate}
      \item Either factor it by hand or use a graphing calculator .
      \item "Sketch" is an IB "command term" meaning roughly show the key relationships.
      \item Label the intersections, maximum, and minimum.
      \item Add an axis caption of increasing ("++++") and decreasing ("- - - -") intervals
      \end{enumerate}
   \end{block}
    Lesson: Rates of change problem applications, graphing\\%*[5pt]
    Homework: Motion problems
  }

  \end{document}
