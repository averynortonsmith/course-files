\documentclass{beamer}
\usepackage{geometry}
\usepackage[english]{babel}
\usepackage[utf8]{inputenc}
\usepackage{amsmath}
\usepackage{amsfonts}
\usepackage{amssymb}
\usepackage{tikz}
\usetikzlibrary{quotes, angles}
\usepackage{graphicx}

%\usepackage{pgfplots}
%\pgfplotsset{width=10cm,compat=1.9}
%\usepackage{pgfplotstable}

\setlength{\headheight}{26pt}%doesn't seem to fix warning

\usepackage{fancyhdr}
\pagestyle{fancy}
\fancyhf{}

%\rhead{\small{24 March 2019}}
\lhead{\small{BECA / Dr. Huson / Geometry - Unit 10 Area \& Volume}}

\renewcommand{\headrulewidth}{0pt}

\title{10th Grade Geometry - Unit 10 Area \& Volume}
\subtitle{Bronx Early College Academy}
\author{Christopher J. Huson PhD}
\date{15 April 2019}

\begin{document}
\frame{\titlepage}
\section[Outline]{}
\frame{\tableofcontents}


\section{10.1 Solids and Their Cross Sections Monday 15 April}
  \frame
  {
    \frametitle{GQ: How do we slice 3-dimensional obects?}
    \framesubtitle{CCSS: HSG.CO.D.12 Congruence, geometric constructions \hfill \alert{10.1 Monday 15 April}}

    \begin{block}{Do Now: Solids handout}
      \begin{enumerate}
        \item Review the handout
        \item Open your notebook to Friday's lesson
        \item Write from memory the formulas for a circle's circumfernce and area
      \end{enumerate}
    \end{block}
    Lesson: Solids and cross sections of 3-dimensional figures\\
    Homework: Practice problems handout
  }

\section{10.2 Geogebra - Area Situations in Sports Tuesday 16 April}
  \frame
  {
    \frametitle{GQ: How do we measure the areas of competitive sports?}
    \framesubtitle{CCSS: MP5 Use appropriate tools strategically: dynamic geometry software \hfill \alert{10.2 Tuesday 16 April}}

    \begin{block}{Project: Quantify the area of a playing field of your choice}
      \begin{enumerate}
        \item Write a paper illustrating the area of part of a playing field.
        \item Spicy: Use color \& line variations for clarity (not decoration)
        \item Construct in Geogebra, compile in Word: add heading \& title, text, and formulas using Microsoft's equation editor
        \item Email me: Last-Title.pdf, with subject line \& message
        \item Rubric: correct, aesthetics, MLA \& email standards
      \end{enumerate}
    \end{block}
    Homework: Complete project (due by 10:00 pm)
  }

\section{10.3 Review for exam Wednesday 17 April}
  \frame
  {
    \frametitle{GQ: How do we slice 3-dimensional obects?}
    \framesubtitle{CCSS: HSG.CO.D.12 Congruence, geometric constructions \hfill \alert{10.3 Wednesday 17 April}}

    \begin{block}{Do Now: Review Exercises p. 189}
      \begin{enumerate}
        \item Review for exam
    \end{enumerate}
    \end{block}
    Lesson: Test review, areas, volumes, density; cumulative review\\[1cm]
    Homework: Study for \alert{exam tomorrow}
  }

\section{10.4 Exam Thursday 18 April}
  \frame
  {
    \frametitle{GQ: How do we slice 3-dimensional obects?}
    \framesubtitle{CCSS: HSG.CO.D.12 Congruence, geometric constructions \hfill \alert{10.4 Thursday 18 April}}

    Lesson: Test \\[1cm]
    Homework: Vacation packet
  }

\section{10.5 Dilation+similarity review Monday 29 April}
  \frame
  {
    \frametitle{GQ: How do we use similarity to calculate triangle lengths?}
    \framesubtitle{CCSS: HSG.CO.D.12 Congruence, geometric constructions \hfill \alert{10.5 Monday 29 April}}
    \begin{block}{Do Now: Similar triangles problem}
      \begin{enumerate}
        \item Use dilation approach to solve for $x$ and $y$
        \item Alternate method: trig $\cos \theta$
    \end{enumerate}
    \end{block}
    Lesson: Review exam results\\[1cm]
    Homework: Test corrections
  }

\section{10.6 Deltamath Volume Practice Tuesday 30 April}
  \frame
  {
    \frametitle{GQ: How do we model 3-dimensional obects?}
    \framesubtitle{CCSS: HSG.CO.D.12 Congruence, geometric constructions \hfill \alert{10.6 Tuesday 30 April}}

    Lesson: Deltamath practice \\[1cm]
    Homework: Complete Deltamath assignment (due Thursday 10pm)
  }

\section{10.7 Dilation+similarity review Monday 3 May}
  \frame
  {
    \frametitle{GQ: How do we model 3-dimensional obects?}
    \framesubtitle{CCSS: HSG.CO.D.12 Congruence, geometric constructions \hfill \alert{10.7 Monday 3 May}}
    \begin{block}{Do Now: Volume warmup with trig ratio review}
      \begin{enumerate}
        \item Using the formula sheet \& calculators
        \item Rounding practice
        \item Simple trigonometry situations: $\sin, \cos, \tan \theta$
    \end{enumerate}
    \end{block}
    Lesson: Volume with density problems\\[1cm]
    Homework: Handout packet mixed volume \& trig review
  }

\end{document}
