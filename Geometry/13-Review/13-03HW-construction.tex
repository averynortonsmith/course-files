\documentclass[12pt, oneside]{article}
\usepackage[letterpaper, margin=1in, headsep=0.5in]{geometry}
\usepackage[english]{babel}
\usepackage[utf8]{inputenc}
\usepackage{amsmath}
\usepackage{amsfonts}
\usepackage{amssymb}
\usepackage{tikz}
\usetikzlibrary{quotes, angles}
\usepackage{graphicx}
%\usepackage{pgfplots}
%\pgfplotsset{width=10cm,compat=1.9}
%\usepgfplotslibrary{statistics}
%\usepackage{pgfplotstable}
%\usepackage{tkz-fct}
%\usepackage{venndiagram}

\usepackage{fancyhdr}
\pagestyle{fancy}
\fancyhf{}
\rhead{\thepage \\Name: \hspace{1.5in}.\\}
\lhead{BECA / Dr. Huson / 10th Grade Geometry\\* 30 May 2019}

\renewcommand{\headrulewidth}{0pt}

\begin{document}
\subsubsection*{Homework: Construction review}
Use only a compass and straightedge for these classical constructions.
  \begin{enumerate}

  \item Construct an equilateral triangle having one side on $\overrightarrow{T}$ with each leg congruent to $\overline{AB}$.
    [Leave all construction marks.]\\
      \vspace{3cm}
      \begin{center}
      \begin{tikzpicture}
        \draw [-, thick] (0,3)--(3,0);
        \draw [->, thick] (4,-3)--(11,-3);
        \draw [fill] (4,-3) circle [radius=0.05] node[above left]{$T$};
        \draw [fill] (0,3) circle [radius=0.05] node[above left]{$A$};
        %\node at (8.5,-0.4){$l$};
        \draw [fill] (3,0) circle [radius=0.05] node[below right]{$B$};
      \end{tikzpicture}
      \end{center}
      %\vspace{2cm}

\newpage
  \item Duplicate a given angle.\\[0.5cm]
  \hspace{1cm} Construct an angle with vertex $R$ and one leg the ray $\overrightarrow{R}$, congruent to $\angle A$. Show all construction marks.\\[0.5cm]
    Spicy: List the steps\\
    \vspace{1cm}
    \begin{center}
    \begin{tikzpicture}
      \draw [<->, thick] (3,6)--(0,0)--(9,0);
      \draw [fill] (0,0) circle [radius=0.05] node[below]{$A$};
      \draw [->, thick] (2,-8)--(10,-8);
      \draw [fill] (2,-8) circle [radius=0.05] node[below]{$R$};
    \end{tikzpicture}
    \end{center}
\newpage

\item Construct a perpendicular to $\overline{AB}$ though $C$.\\
  %\hspace{1cm} Given the line  $l$ and point $P$.
  \vspace{2cm}
  \begin{center}
  \begin{tikzpicture}
    \draw [<->, thick] (0,0)--(11,0)--(6,4)--cycle;
    \draw [fill] (0,0) circle [radius=0.05] node[left]{$A$};
    \draw [fill] (11,0) circle [radius=0.05] node[right]{$B$};
    \draw [fill] (6,4) circle [radius=0.05] node[above right]{$C$};
  \end{tikzpicture}
\end{center} \vspace{5cm}

\newpage
  \item Spicy: Construct the perpendicular bisectors of the legs of a triangle and, hence, the circumcenter.\\
    %\hspace{1cm} Given the line  $l$ and point $P$.
    \vspace{3cm}
    \begin{center}
    \begin{tikzpicture}
      \draw [<->, thick] (0,0)--(9,0)--(7,11)--cycle;
      %\draw [fill] (2,3) circle [radius=0.05] node[right]{$P$};
      %\node at (8.5,-0.4){$l$};
      %\draw [fill] (6,0) circle [radius=0.05] node[below]{$Q$};
    \end{tikzpicture}
    \end{center}

\end{enumerate}
\end{document}
