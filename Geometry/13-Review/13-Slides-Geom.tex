\documentclass{beamer}
\usepackage{geometry}
\usepackage[english]{babel}
\usepackage[utf8]{inputenc}
\usepackage{amsmath}
\usepackage{amsfonts}
\usepackage{amssymb}
\usepackage{tikz}
\usetikzlibrary{quotes, angles}
\usepackage{graphicx}
\usepackage{multicol}
%\usepackage{pgfplots}
%\pgfplotsset{width=10cm,compat=1.9}
%\usepackage{pgfplotstable}

\setlength{\headheight}{26pt}%doesn't seem to fix warning

\usepackage{fancyhdr}
\pagestyle{fancy}
\fancyhf{}

%\rhead{\small{24 March 2019}}
\lhead{\small{BECA / Dr. Huson / Geometry - Unit 13: Regents Review}}

\renewcommand{\headrulewidth}{0pt}

\title{10th Grade Geometry - Unit 13: Regents Review}
\subtitle{Bronx Early College Academy}
\author{Christopher J. Huson PhD}
\date{27 May 2019}

\begin{document}
\frame{\titlepage}
\section[Outline]{}
\frame{\tableofcontents}


\section{13.1 Scale \& applications of dilation Tuesday 28 May}
  \frame
  {
    \frametitle{GQ: How do we use scale factors?}
    \framesubtitle{CCSS: HSG.CO.D.12 Congruence, geometric constructions \hfill \alert{13.1 Tuesday 28 May}}

    \begin{block}{Do Now: Handout}
      \begin{enumerate}
        \item Using scale factors
        \item Real world situations
      \end{enumerate}
    \end{block}
    Guest teacher, Mr. Segal. Applications of scale factors in finance.\\[0.25cm]
    Homework: Problem set, test corrections due Thursday
  }

  \frame
  {
    \frametitle{GQ: How do we use scale factors?}
    \framesubtitle{Triangle similarity: for your notebook}
      Given  \[\triangle ABC \sim \triangle DEF \]
      Equivalently \[ \triangle ABC \rightarrow \triangle DEF \]
      Complete the three line segment correspondences, three scale factor ratios, \& three dilations.\\[0.5cm]
        \begin{multicols}{3}
          \renewcommand{\baselinestretch}{1.5}
          \begin{enumerate}
            \item $\overline{AB} \rightarrow \overline{DE} $
            \item $\overline{BC} \rightarrow$
            \item $\overline{AC} \rightarrow$
          \end{enumerate}
          \begin{enumerate}
            \item $k= \frac{DE}{AB}$
            \item $k=$
            \item $k=$
          \end{enumerate}
          \begin{enumerate}
            \item $DE= k \times AB$
            \item $EF= k \times $
            \item $DF= k \times $
          \end{enumerate}
        \end{multicols}

  What happens if $k=1$?
  }

\section{13.2 Similarity review Wednesday 29 May}
  \frame
  {
    \frametitle{GQ: How do we use scale factors?}
    \framesubtitle{CCSS: HSG.CO.D.12 Congruence, geometric constructions \hfill \alert{13.2 Wednesday 29 May}}

    \begin{block}{Do Now: Quadrilateral properties}
      \begin{enumerate}
        \item Given a list of features, identify the applicable quadrilateral
        \item Early finishers: Triangle congruency proofs
      \end{enumerate}
    \end{block}
    ASA proof of a parallelogram's congruent triangles, implications\\
    Pretest packet: volume, trig, analytic geometry
  }

\section{13.3 Constructions Thursday 30 May}
  \frame
  {
    \frametitle{GQ: How do we use scale factors?}
    \framesubtitle{CCSS: HSG.CO.D.12 Congruence, geometric constructions \hfill \alert{13.3 Thursday 30 May}}

    \begin{block}{Do Now: Similarity transformation}
      \begin{enumerate}
        \item List what segments map to what segments
        \item Find $k$ as a ratio, apply it to each length.
      \end{enumerate}
    \end{block}
    Binder check\\
    Classical constructions using compass \& straightedge\\[0.5cm]
    Homework: Problem set
  }

\section{13.4 Constructions, Similarity quiz Friday 31 May}
  \frame
  {
    \frametitle{GQ: How do we use scale factors?}
    \framesubtitle{CCSS: HSG.CO.D.12 Congruence, geometric constructions \hfill \alert{13.4 Friday 31 May}}

    Do Now handout: Constructions, dilation problems\\
    Classwork review
    \begin{block}{Assessment: Exit quiz (\& Binder check)}
      \begin{enumerate}
        \item Angle bisector, perpendiculars constructions
        \item Dilation situations
        \item Similarity proof situations
    \end{enumerate}
    \end{block}
    Homework: Review packet; \\
    Quiz corrections due Wednesday (pick up Monday)\\
    \alert{Monday Regents review after History exam, Melrose Library}
  }

  \section{13.5 Slope review Wednesday 5 June}
    \frame
    {
      \frametitle{GQ: How do we apply slope calculations?}
      \framesubtitle{CCSS: HSG.CO.D.12 Congruence, geometric constructions \hfill \alert{13.5 Wednesday 5 June}}

      \begin{block}{Do Now: Handout}
        \begin{enumerate}
          \item Duplicate a line segment \& an angle
          \item Parallel \& perpendicular slopes
        \end{enumerate}
      \end{block}
      Circle equations\\[0.5cm]
      Assessment: Exit quiz covering slope applications\\
      Review packet: Triangles, parallels, circle equations, intersections
    }

    \frame
    {
      \frametitle{Equations of circles in different forms}
      \framesubtitle{Use algebra, the distributive property}

      \begin{block}{Take notes in your notebook}
        \begin{enumerate}
          \item State the center \& radius of $x^2+(y-1)^2=25$
          \item Write the equation of a circle centered at $(2,3)$ with $r=3$
          \item True or false: $(x-2)^2=x^2-4x+4$?
          \item Is $x^2-4x+y^2 = 5$ a circle?
          \item Which equation represents a circle with center $(2,3)$ \& $r=5$?
          \begin{enumerate}
            \item $x^2-4x+y^2-6y = 25$ \vspace{0.25cm}
            \item $x^2-4x+y^2-6y = 12$
          \end{enumerate}
        \end{enumerate}
      \end{block}
      }

  \section{13.6 Similarity review Friday 7 June}
    \frame
    {
      \frametitle{GQ: How do we calculate the measure of angles?}
      \framesubtitle{CCSS: HSG.CO.D.12 Congruence, geometric constructions \hfill \alert{13.6 Friday 7 June}}

      \begin{block}{Do Now: Angle measures}
        \begin{enumerate}
          \item Vertical, supplementary, complementary angles
          \item Triangle internal \& external angles theorems
          \item Parallel lines with transversals
        \end{enumerate}
      \end{block}
      Assessment: Exit quiz covering angle measure situations\\
      \alert{It is your responsibility to complete projects and check Pupilpath}\\
      Review packet: Transformations
    }

  \section{13.7 Transformations review Monday 10 June}
    \frame
    {
      \frametitle{GQ: How do we transform objects to their image?}
      \framesubtitle{CCSS: HSG.CO.D.12 Congruence, geometric constructions \hfill \alert{13.7 Monday 10 June}}

      \begin{block}{Do Now: Transformations}
        \begin{enumerate}
          \item Transformation, reflection, rotation, dilation
          \item Symmetry \& transformations onto itself
          \item Segment partitions by a ratio
          \item Cross sections
        \end{enumerate}
      \end{block}
      Assessment: Exit quiz covering transformations\\
      Review packet: Transformations
    }

  \section{13.8 Distance review Tuesday 11 June}
    \frame
    {
      \frametitle{GQ: How do we calculate distances?}
      \framesubtitle{CCSS: HSG.CO.D.12 Congruence, geometric constructions \hfill \alert{13.8 Tuesday 11 June}}

      \begin{block}{Do Now: Distance}
        \begin{enumerate}
          \item Pythagorean formula
          \item Using distance in proof
          \item Midpoint, midpoint extension
          \item Slant length situations
        \end{enumerate}
      \end{block}
      Assessment: Exit quiz covering distance situations\\
      Review packet: Circles \& trigonometry
    }
\end{document}
