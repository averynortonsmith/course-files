\documentclass{beamer}
\usepackage{geometry}
\usepackage[english]{babel}
\usepackage[utf8]{inputenc}
\usepackage{amsmath}
\usepackage{amsfonts}
\usepackage{amssymb}
\usepackage{tikz}
\usetikzlibrary{quotes, angles}
\usepackage{graphicx}

%\usepackage{pgfplots}
%\pgfplotsset{width=10cm,compat=1.9}
%\usepackage{pgfplotstable}

\setlength{\headheight}{26pt}%doesn't seem to fix warning

\usepackage{fancyhdr}
\pagestyle{fancy}
\fancyhf{}

%\rhead{\small{2 January 2019}}
\lhead{\small{BECA / Dr. Huson / Geometry - Unit 7 Analytic Geometry}}

\renewcommand{\headrulewidth}{0pt}

\title{10th Grade Geometry - Unit 6: Similarity}
\subtitle{Bronx Early College Academy}
\author{Christopher J. Huson PhD}
\date{29 January 2019}

\begin{document}
\frame{\titlepage}
\section[Outline]{}
\frame{\tableofcontents}


\section{7.1 Laptops - Geogebra. Tuesday 28 January}
  \frame
  {
    \frametitle{GQ: How do we model with digital tools?}
    \framesubtitle{CCSS: HSG.CO.D.12 Congruence, geometric constructions \hspace{\stretch{1}} \alert{7.1 Tuesday 18 January}}

    \begin{block}{Do Now: Regents review and reflection}
      \begin{itemize}
        \item Results: 10 passing scores, 4 college ready
        \item Top score 75; average 53
        \item 70\% earned free response points
      \end{itemize}
    \end{block}
    GeoGebra Geometry App\\
    Enter \alert{P9PNZ} on www.geogebra.org/groups to join class\\
    Set up account using your real name.\\
    Beginner Tutorials with Lesson Ideas\\
    Author: Tim Brzezinski\\[0.5cm]
    Homework: Complete Geogebra
  }

\section{7.2 Linear equations. Wednesday 29 January}
  \frame
  {
    \frametitle{GQ: How do we use functions and equations to represent objects on the coordinate plane?}
    \framesubtitle{CCSS: HSG.CO.D.12 Congruence, geometric constructions \hspace{\stretch{1}} \alert{7.2 Wednesday 29 January}}

    \begin{block}{Do Now: Handout}
      \begin{enumerate}
        \item Dilation, plotting equations
      \end{enumerate}
    \end{block}
    Function notation, slope-intercept, standard, \& point slope forms of linear equations\\[0.5cm]
    Homework: Handout review of linear equations and functions
  }

\section{7.3 Linear equations. Thursday 31 January}
  \frame
  {
    \frametitle{GQ: How do we use functions and equations to represent objects on the coordinate plane?}
    \framesubtitle{CCSS: HSG.CO.D.12 Congruence, geometric constructions \hspace{\stretch{1}} \alert{7.3 Wednesday 29 January}}

    \begin{block}{Do Now: Handout}
      \begin{enumerate}
        \item Dilation, plotting equations
      \end{enumerate}
    \end{block}
    Function notation, slope-intercept, standard, \& point slope forms of linear equations\\[0.5cm]
    Homework: Handout review of linear equations and functions
  }


\section{7.4 Linear equations. Wednesday 29 January}
  \frame
  {
    \frametitle{GQ: How do we use functions and equations to represent objects on the coordinate plane?}
    \framesubtitle{CCSS: HSG.CO.D.12 Congruence, geometric constructions \hspace{\stretch{1}} \alert{7.4 Wednesday 29 January}}

    \begin{block}{Do Now: Handout}
      \begin{enumerate}
        \item Dilation, plotting equations
      \end{enumerate}
    \end{block}
    Function notation, slope-intercept, standard, \& point slope forms of linear equations\\[0.5cm]
    Homework: Handout review of linear equations and functions
  }




\end{document}
