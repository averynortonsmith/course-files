\documentclass[12pt, oneside]{article}
\usepackage[letterpaper, margin=1in, headsep=0.5in]{geometry}
\usepackage[english]{babel}
\usepackage[utf8]{inputenc}
\usepackage{amsmath}
\usepackage{amsfonts}
\usepackage{amssymb}
\usepackage{tikz}
\usetikzlibrary{quotes, angles}
%\usepackage{pgfplots}
%\pgfplotsset{width=10cm,compat=1.9}



%\usepackage{tkz-fct} %alternative to pgfplots.
%\usepackage{pgfplots}
%\pgfplotsset{width=10cm,compat=1.9}
%\usepgfplotslibrary{statistics}
%\usepackage{pgfplotstable}
%\usepackage{venndiagram}


\begin{document}

\begin{tikzpicture} %https://tex.stackexchange.com/questions/256863/drawing-and-labeling-angles-in-an-axis-environment

%A triangle is drawn on the Cartesian plane. One side of the triangle is along
%the positive x-axis, and another side of the triangle is drawn in Quadrant II.
\coordinate (A) at (0,0);
\coordinate (B) at (3.5,0);
\coordinate (C) at ({5*cos(120)},{5*sin(120)});
\draw (A) -- (B) -- (C) -- cycle;

\coordinate (D) at (A-|C);% horizontal to (A), vertical to (C)

%The labels for A and B are typeset.
\node[below right] at (A){$A$};
\node[below] at (B){$B$};
%The label for C is typeset.
\node[blue,above left] at (C) {$C$};


%Angles are drawn for $\theta$ and its supplement.
\draw[draw=blue] (A) ++(120:0.4) arc (120:0:0.4)
  node[midway,above right,inner sep=2pt,font={\footnotesize}]{$\theta$};
\draw[draw=blue,dash dot] (A) ++(180:0.6) arc (180:120:0.6)
  node[midway,left,inner sep=2pt,font={\footnotesize}]{$\pi - \theta$};

%A right-angle mark is drawn.
\draw[dash dot] (D) +(0,3mm) -- +(3mm,3mm) -- +(3mm,0);

\draw[dashed] (C) -- (D);
\end{tikzpicture}

\end{document}
