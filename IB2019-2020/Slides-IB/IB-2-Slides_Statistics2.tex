\documentclass{beamer}
\usepackage{geometry}
\usepackage[english]{babel}
\usepackage[utf8]{inputenc}
\usepackage{amsmath}
\usepackage{amsfonts}
\usepackage{amssymb}
\usepackage{tikz}
\usepackage{graphicx}
\usepackage{venndiagram}

%\usepackage{pgfplots}
%\pgfplotsset{width=10cm,compat=1.9}
%\usepackage{pgfplotstable}

\setlength{\headheight}{26pt}%doesn't seem to fix warning

\usepackage{fancyhdr}
\pagestyle{fancy}
\fancyhf{}

%\rhead{\small{5 September 2018}}
\lhead{\small{BECA / Dr. Huson / IB Math Unit 2}}

\renewcommand{\headrulewidth}{0pt}

\title{Mathematics Class Slides}
\subtitle{Bronx Early College Academy}
\author{Chris Huson}
\date{15 October 2019}

\begin{document}
\frame{\titlepage}
\section[Outline]{}
\frame{\tableofcontents}

  \section{2.1 Intro to statistics 11 Oct}
  \frame
  {
    \frametitle{GQ: How do we collect and organize data?}
    \framesubtitle{CCSS: HSF.IF.C.7 Analyze functions \hfill \alert{2.1 Friday 11 Oct}}

    \begin{block}{Do Now Handout: Analyze chart on p 45}
    \begin{enumerate}
        \item Income vs Health of the world's nations
        \item Answer the questions
    \end{enumerate}
    \end{block}
    Lesson: Statistics concepts and vocabulary pp 44-50 \\%*[5pt]
    Homework: Problem set: Organizing data 2A p. 50
  }

  \section{2.2 Deltamath: statistics 15 Oct}
  \frame
  {
    \frametitle{GQ: How do we quantify central tendency and dispersion?}
    \framesubtitle{CCSS: MP5 attend to precision \hfill \alert{2.2 Tuesday 15 Oct}}

    \begin{block}{Using spreadsheets for data, calculations, and display}
      \begin{enumerate}
        \item Boot up laptops, log in to email and Google Drive account
        \item Download Excel ``Simple Calculator'' (explore)
        \item Early finishers: model problem \#5, p. 56
    \end{enumerate}
    \end{block}
    Deltamath practice\\ \smallskip
    Homework: Complete Deltamath problems, 10:00PM deadline
  }

  \section{2.3 Descriptive statistics measures, 17 Oct}
  \frame
  {
    \frametitle{GQ: How do we quantify central tendency and dispersion?}
    \framesubtitle{CCSS: MP5 attend to precision \hfill \alert{2.3 Thursday 17 Oct}}

    \begin{block}{Laptops: Statistical analysis in Excel}
    \begin{enumerate}
        \item Create a one-page report answering problem \#5 p.56
        \item Replicate the ``raw data table'' with modifications for Excel
        \item Use Excel functions for the required statistical calculations
        \item Include text to answer the question with a short justification
        \item Format in Excel (including MLA header) and ``print'' as pdf
        \item Email Excel and pdf to me by 10:00pm today
        \item Early finishers: Deltamath
    \end{enumerate}
    \end{block}
  }

  \section{2.4 Descriptive statistics measures, 18 Oct}
  \frame
  {
    \frametitle{GQ: How do we quantify central tendency and dispersion?}
    \framesubtitle{CCSS: MP5 attend to precision \hfill \alert{2.4 Friday 18 Oct}}

    \begin{block}{Laptops: Statistical analysis with a handheld calculator}
    \begin{enumerate}
        \item Enter in your calculator problem \#5 p.56
        \item Compare results to your Excel report
        \item Prepare to discuss comparison of Excel to a handheld calculator
    \end{enumerate}
    \end{block}
    Review Excel analysis and reporting \\
    Lesson: Continuous data and and using frequency tables p 57\\ \smallskip
    Homework: Practice exercises 2B \& 2C p. 55-56, 58 (class time tomorrow)
  }

  \section{2.5 Frequency tables of continuous variables 21 Oct}
  \frame
  {
    \frametitle{GQ: How do we collect and organize data?}
    \framesubtitle{CCSS: HSF.IF.C.7 Analyze functions \hfill \alert{2.5 Monday 21 Oct}}

    \begin{block}{Do Now: \#13 P2, p 78}
    \begin{enumerate}
        \item Work on loose leaf paper you can turn in
        \item Use a calculator
        \item Use the definition of outlier on page 53
        \item Draw the plot accurately
    \end{enumerate}
    \end{block}
    Exploration paper scoring criterion: Personal Engagement \\
    Review Excel analysis \\
    Lesson: Continuous data and and using frequency tables p 57\\ \smallskip
    Homework: Rework Excel file
  }

  \section{2.6 Revise Excel analysis summary page 22 Oct}
  \frame
  {
    \frametitle{GQ: How do we communicate statistical results?}
    \framesubtitle{CCSS: MP5 attend to precision \hfill \alert{2.6 Tuesday 22 Oct}}

    \begin{block}{Using spreadsheets for data, calculations, and display}
      \begin{enumerate}
        \item Boot up laptops, log in to email and Google Drive account
        \item Download your saved Excel model of Mr. Jones club scores
        \item Complete and improve your analysis
        \item Email the Excel file and a pdf version (1 page)
    \end{enumerate}
    \end{block}
    Deltamath practice\\ \smallskip
    Homework: Complete Deltamath problems, 10:00PM deadline
  }

  \section{2.6 Histograms and box plots 23 Oct}
  \frame
  {
    \frametitle{GQ: How do we display data?}
    \framesubtitle{CCSS: HSF.IF.C.7 Analyze functions \hfill \alert{2.6 Wednesday 23 Oct}}

    \begin{block}{Do Now: \#3 p. 75}
    \begin{enumerate}
        \item Work on loose leaf paper you can turn in
        \item Use a calculator
        \item Explain (in writing) why your answers are estimates
    \end{enumerate}
    \end{block}
    Real world, pseudo real world, \& pure math problems; implications \\
    Birthday data analysis \\
    Lesson: Box plots and histograms p 59-62\\ \smallskip
    Homework: Textbook exercises 2D p. 62
  }

  \section{2.7 Cumulative frequency tables and graphs 24 Oct}
  \frame
  {
    \frametitle{GQ: How do we display and interpret cumulative data?}
    \framesubtitle{CCSS: HSF.IF.C.7 Analyze functions \hfill \alert{2.7 Thursday 24 Oct}}

    \begin{block}{Do Now: \#4, p 75, continued on p. 76}
    \begin{enumerate}
        \item Work on loose leaf paper you can turn in
        \item Use a calculator for calculations and to replicate the plot
        \item Clearly answer parts \#4d.i and \#4d.ii
    \end{enumerate}
    \end{block}
    Exploration paper scoring criterion: Personal Engagement \\
    Review student birthday survey data \& analysis \\
    Lesson: Cumulative frequency tables and graphs p 63-5\\ \smallskip
    Homework: Textbook exercises 2E p. 64-5
  }

  \section{2.8 Revise Excel analysis summary page 25 Oct}
  \frame
  {
    \frametitle{GQ: How do we communicate statistical results?}
    \framesubtitle{CCSS: MP5 attend to precision \hfill \alert{2.8 Friday 25 Oct}}

    \begin{block}{Mini Exploration: What is the best route to school?}
      \begin{enumerate}
        \item Based on Excel model of commuter data (math.huson.com)
        \item Complete statistical calculations and written analysis
        \item Email the Excel file and a pdf version of spreadsheet \& paper\\
        (three attachments)
    \end{enumerate}
    \end{block}
    Homework: Complete your paper, Sunday 10:00PM deadline
  }

  \section{2.9 Cumulative frequency tables and graphs 28 Oct}
  \frame
  {
    \frametitle{GQ: How do we display and interpret cumulative data?}
    \framesubtitle{CCSS: HSF.IF.C.7 Analyze functions \hfill \alert{2.9 Monday 28 Oct}}

    \begin{block}{Do Now: Handout IB exam problems (paper 2, with calculator)}
    \begin{enumerate}
        \item Work on loose leaf paper you can turn in
        \item Frequency distribution (table)
        \item Box plot interpretation
    \end{enumerate}
    \end{block}
    Peer review of draft of subway commute analysis \\
    Lesson: Cumulative frequency tables and graphs, handout\\ \smallskip
    Homework: Statistics exam problems handout
  }

  \section{2.10 Revise Excel analysis summary page 29 Oct}
  \frame
  {
    \frametitle{GQ: How do we communicate statistical results?}
    \framesubtitle{CCSS: MP5 attend to precision \hfill \alert{2.10 Tuesday 29 Oct}}

    \begin{block}{Mini Exploration: What is the best route to school?}
      \begin{enumerate}
        \item Based on Excel model of commuter data (math.huson.com)
        \item Complete written analysis
        \item Email the Excel file and a pdf version of spreadsheet \& paper\\
        (three attachments)
    \end{enumerate}
    \end{block}
    Homework: Complete your paper, today 10:00PM deadline
  }


  \section{2.11 Bivariate data 30 Oct}
  \frame
  {
    \frametitle{GQ: How do we display and interpret bivariate data?}
    \framesubtitle{CCSS: HSF.IF.C.7 Analyze functions \hfill \alert{2.11 Wednesday 30 Oct}}

    \begin{block}{Do Now: Handout IB exam problems (paper 2, with calculator)}
    \begin{enumerate}
        \item Work on loose leaf paper you can turn in
        \item Frequency distribution (table)
        \item Box plot interpretation
    \end{enumerate}
    \end{block}
    Peer review of draft of subway commute analysis \\
    Lesson: Cumulative frequency tables and graphs, handout\\ \smallskip
    Homework: Statistics exam problems handout
  }

  \section{2.11 Exploration project paper schedule Oct}
  \frame
  {
    \frametitle{GQ: How do we employ mathematics to explore a topic?}
    \framesubtitle{CCSS: MP5 attend to precision \hfill \alert{Wednesday 30 Oct}}

    \begin{block}{Exploration: Schedule and deadlines}
      \begin{enumerate}
        \item Topic selection - Monday November 4th
        \item Complete paper for peer review - Friday November 22nd
        \item Complete paper for grade - Wednesday November 28th
        \item Final paper - Wednesday December 20th

    \end{enumerate}
    \end{block}
  }
    \begin{block}{Do Now: Mr. Price's students, page 66}
    \begin{enumerate}
        \item Enter the data in the table on page 66 into your calculator
        \item To graph the data pairs, what values for $x$ and $y$ are needed?
        \item Plot the data on graph paper 
        \item Interpretation the graph
    \end{enumerate}
    \end{block}
    Mind map / brainstorming an exploration topic p. 743 \\
    Lesson: 2.4 Comparing two sets of related quantities\\ \smallskip
    Homework: Make at least one topic mind map (due tomorrow)\\
    Exercises 2F, p. 71-72 (due Friday, share data for extra credit)
  }


\end{document}
https://www.economist.com/graphic-detail/2019/10/25/how-being-second-choice-could-put-elizabeth-warren-on-top?cid1=cust/dailypicks1/n/bl/n/20191028n/owned/n/n/dailypicks1/n/n/NA/333375/n