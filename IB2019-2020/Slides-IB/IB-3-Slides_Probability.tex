\documentclass{beamer}
\usepackage{geometry}
\usepackage[english]{babel}
\usepackage[utf8]{inputenc}
\usepackage{amsmath}
\usepackage{amsfonts}
\usepackage{amssymb}
\usepackage{tikz}
\usepackage{graphicx}
\usepackage{venndiagram}

%\usepackage{pgfplots}
%\pgfplotsset{width=10cm,compat=1.9}
%\usepackage{pgfplotstable}

\setlength{\headheight}{26pt}%doesn't seem to fix warning

\usepackage{fancyhdr}
\pagestyle{fancy}
\fancyhf{}

%\rhead{\small{5 September 2018}}
\lhead{\small{BECA / Dr. Huson / IB Math Unit 3 - Probability}}

\renewcommand{\headrulewidth}{0pt}

\title{Mathematics Class Slides}
\subtitle{Bronx Early College Academy}
\author{Chris Huson}
\date{12 November 2019}

\begin{document}
\frame{\titlepage}
\section[Outline]{}
\frame{\tableofcontents}

\section{3.0 Exploration project paper schedule}
\frame
{
  \frametitle{GQ: How do we employ mathematics to explore a topic?}
  \framesubtitle{CCSS: MP5 attend to precision \hfill \alert{originally Thursday 31 Oct}}
  \begin{block}{Exploration: Schedule and deadlines}
    \begin{enumerate}
      \item Topic selection - Monday November 4th
      \item In class work sessions (you must work at home too)
      \begin{enumerate}
        \item Independent work on introduction, data, mathematics - Nov 11
        \item Complete design of methods, collect data - Nov 15
        \item Apply mathematics, write up methods \& results - Nov 19
        \item Finalize peer review paper, print - Nov 22
      \end{enumerate}
      \item Complete paper for peer review - Friday November 22nd
      \item Complete paper for grade - Friday December 6th
      \item Final paper - Friday January 17th
    \end{enumerate}
    \end{block}
}

\section{3.1 Exploration paper student work time (1) 12 Nov}
\frame
{
  \frametitle{GQ: How do we use mathematics to explore a topic?}
  \framesubtitle{CCSS: MP5 attend to precision \hfill \alert{3.1 Tuesday 12 Nov}}

  \begin{block}{Work on exploration papers}
  \begin{enumerate}
      \item Inputs: what data will you use and how will you get it? 
      \item What mathematics will you apply (find the textbook chapter)
      \item Outputs: What results will you use to answer your aim?
      \item Start drafting and re-drafting your introduction (aim, rationale, personal engagement)
  \end{enumerate}
  \end{block}
  Homework: Develop exploration 
}

\section{3.2 Introduction to probability 13 Nov}
\frame
{
  \frametitle{GQ: How do we use mathematics to explore a topic?}
  \framesubtitle{CCSS: MP5 attend to precision \hfill \alert{3.2 Wednesday 13 Nov}}

  \begin{block}{Do Now Skills check page 205}
  \begin{enumerate}
      \item Treat probabilities as fractions between zero and one 
      \item Use tables to organize data
  \end{enumerate}
  \end{block}
  Afterschool today; parent conferences tomorrow 4:00-7:00, Friday\\ \smallskip
  Lesson: theoretical and experimental probabilities, notation \\
  Sleeping situation\\ \smallskip
  Homework: Read and evaluate sample exploration paper according to criteria pp. 737-740
}

\section{3.3 Probability as fractions 14 Nov}
\frame
{
  \frametitle{GQ: How do we quantify uncertainty?}
  \framesubtitle{CCSS: MP5 attend to precision \hfill \alert{3.3 Thursday 14 Nov}}

  \begin{block}{2.16 Do Now: Dice probability}
  \begin{enumerate}
      \item probability tables (pdf)
  \end{enumerate}
  \end{block}
  Scoring an exploration paper \\
  Lesson: Probability calculations \\ \smallskip
  Homework: Textbook exercises 5A p. 210-211; exploration paper
}

\section{3.4 Exploration paper student work time (2) 15 Nov}
\frame
{
  \frametitle{GQ: How do we use mathematics to explore a topic?}
  \framesubtitle{CCSS: MP5 attend to precision \hfill \alert{3.4 Friday 15 Nov}}

  \begin{block}{Work on exploration papers}
  \begin{enumerate}
      \item Inputs: what data will you use and how will you get it? 
      \item What mathematics will you apply (find the textbook chapter)
      \item Outputs: What results will you use to answer your aim?
      \item Start drafting and re-drafting your introduction (aim, rationale, personal engagement)
  \end{enumerate}
  \end{block}
  Homework: Develop exploration 
}

\section{3.5 Sample spaces \& Venn diagrams 18 Nov}
\frame
{
  \frametitle{GQ: How do we organize the event space for analysis?}
  \framesubtitle{CCSS: MP5 attend to precision \hfill \alert{3.5 Monday 18 Nov}}

  \begin{block}{2.16 Do Now: Express each probability as a fraction}
  \begin{enumerate}
      \item Rolling a twelve with two dice
      \item Drawing an ace from a deck of cards
      \item Having a birthday on a weekday
      \item Two students in a class of 30 have the same birthday
  \end{enumerate}
  \end{block}
  Review homework problems 5A p. 211 \\
  Lesson: Venn diagrams and sample spaces \\ \smallskip
  Homework: Textbook exercises 5B p. 215; exploration paper
}

\section{3.6 Exploration paper student work time (3) 19 Nov}
\frame
{
  \frametitle{GQ: How do we use mathematics to explore a topic?}
  \framesubtitle{CCSS: MP5 attend to precision \hfill \alert{3.6 Tuesday 19 Nov}}

  \begin{block}{Work on exploration papers - quiet, independent work}
  \begin{enumerate}
      \item Organize your inputs or data. Do not worry about formatting it yet. 
      \item Apply mathematics, probably with technology. Use pencil \& paper for equations for now (reference the textbook)
      \item Study your initial results. Write down what you find! Brainstorm, outline, type up descriptions, findings, reflections. Tie back to your aim.
      \item Re-write your introduction (aim, rationale, personal engagement). Draft the conclusion (perhaps rough). 
  \end{enumerate}
  \end{block}
  Homework: Develop exploration 
}

\section{3.7 Frequency table, cumulative graphs review; re-Quiz 20 Nov}
\frame
{
  \frametitle{GQ: How do we display and interpret cumulative data?}
  \framesubtitle{CCSS: MP5 attend to precision \hfill \alert{3.7 Wednesday 20 Nov}}

  \begin{block}{2.16 Do Now: Handout practice of stats problems}
  \begin{enumerate}
      \item Frequency tables, univariate summary statistics
      \item Bivariate data analysis
  \end{enumerate}
  \end{block}
  Review stats quiz problems \\
  Lesson: Cumulative plots \\ \smallskip
  Exit note re-quiz: Univariate summary statistics \\ \smallskip
  Homework: Statistics Pre-exam; Exploration paper
}

\section{3.8 Cumulative graphs, regression review 21 Nov}
\frame
{
  \frametitle{GQ: How do we display and interpret cumulative data?}
  \framesubtitle{CCSS: MP5 attend to precision \hfill \alert{3.8 Thursday 21 Nov}}

  \begin{block}{2.16 Do Now: Handout practice of stats problems}
  \begin{enumerate}
      \item Bivariate data analysis
  \end{enumerate}
  \end{block}
  Review stats Pre-exam problems \\
  Homework: Exploration paper
}


\section{3.9 Exploration paper student work time (4) 22 Nov}
\frame
{
  \frametitle{GQ: How do we use mathematics to explore a topic?}
  \framesubtitle{CCSS: MP5 attend to precision \hfill \alert{3.9 Friday 22 Nov}}
  \begin{block}{Submit exploration papers for peer review - quiet, independent work}
  \begin{enumerate}
      \item Organize and print your inputs or data. Formatting is not critical,but label it clearly (by hand is fine). 
      \item Check mathematics. Include spreadsheets in submission to peer. Pencil \& paper for equations are fine, but organize and write clearly.
      \item Explain the results clearly. Complete descriptions, findings, reflections. Tie back to your aim.
      \item Lock down your introduction (aim, rationale, personal engagement)  conclusion (which must tie back to aim). 
  \end{enumerate}
  \end{block}
  Homework: Study for statistics \alert{exam Monday} \\
  Read peer paper, mark with comments, complete checklist (due Tuesday) 
}

\end{document}

Friday: Desmos graphing of soccer shot angle optimization project 

https://www.economist.com/graphic-detail/2019/10/25/how-being-second-choice-could-put-elizabeth-warren-on-top?cid1=cust/dailypicks1/n/bl/n/20191028n/owned/n/n/dailypicks1/n/n/NA/333375/n