\documentclass{beamer}
\usepackage{geometry}
\usepackage[english]{babel}
\usepackage[utf8]{inputenc}
\usepackage{amsmath}
\usepackage{amsfonts}
\usepackage{amssymb}
\usepackage{tikz}
\usepackage{graphicx}
\usepackage{venndiagram}

%\usepackage{pgfplots}
%\pgfplotsset{width=10cm,compat=1.9}
%\usepackage{pgfplotstable}

\setlength{\headheight}{26pt}%doesn't seem to fix warning

\usepackage{fancyhdr}
\pagestyle{fancy}
\fancyhf{}

%\rhead{\small{5 September 2018}}
\lhead{\small{BECA / Dr. Huson / IB Math Unit 3 - Probability}}

\renewcommand{\headrulewidth}{0pt}

\title{Mathematics Class Slides}
\subtitle{Bronx Early College Academy}
\author{Chris Huson}
\date{12 November 2019}

\begin{document}
\frame{\titlepage}
\section[Outline]{}
\frame{\tableofcontents}


\section{3.1 Introduction to probability 13 Nov}
\frame
{
  \frametitle{GQ: How do we use mathematics to explore a topic?}
  \framesubtitle{CCSS: HSS.CP.A.4 Understand conditional probability \hfill \alert{3.1 Wednesday 13 Nov}}

  \begin{block}{Do Now Skills check page 205}
  \begin{enumerate}
      \item Treat probabilities as fractions between zero and one 
      \item Use tables to organize data
  \end{enumerate}
  \end{block}
  Afterschool today; parent conferences tomorrow 4:00-7:00, Friday\\ \smallskip
  Lesson: theoretical and experimental probabilities, notation \\ \smallskip
  Homework: Read and evaluate sample exploration paper according to criteria pp. 737-740
}

\section{3.2 Probability as fractions 14 Nov}
\frame
{
  \frametitle{GQ: How do we quantify uncertainty?}
  \framesubtitle{CCSS: HSS.CP.A.4 Understand conditional probability \hfill \alert{3.2 Thursday 14 Nov}}

  \begin{block}{2.16 Do Now: Dice probability}
  \begin{enumerate}
      \item probability tables (pdf)
  \end{enumerate}
  \end{block}
  Scoring an exploration paper \\
  Lesson: Probability calculations \\ \smallskip
  Homework: Textbook exercises 5A p. 210-211; exploration paper
}

\section{3.3 Sample spaces \& Venn diagrams 18 Nov}
\frame
{
  \frametitle{GQ: How do we organize the event space for analysis?}
  \framesubtitle{CCSS: HSS.CP.A.4 Understand conditional probability \hfill \alert{3.3 Monday 18 Nov}}

  \begin{block}{2.16 Do Now: Express each probability as a fraction}
  \begin{enumerate}
    \item Rolling a twelve with two dice
    \item Drawing an ace from a deck of cards
    \item Having a birthday on a weekday
    \item Two students in a class of 30 have the same birthday
  \end{enumerate}
  \end{block}
  Review homework problems 5A p. 211 \\
  Lesson: Venn diagrams and sample spaces \\ \smallskip
  Homework: Textbook exercises 5B p. 215; exploration paper
}

\section{3.4 Set operations on Venn diagrams 2 Dec}
\frame
{
  \frametitle{GQ: How do we combine events on a Venn diagram?}
  \framesubtitle{CCSS: HSS.CP.A.4 Understand conditional probability \hfill \alert{3.4 Monday 2 Dec}}

  \begin{block}{Do Now: Using Venn diagrams to organize a situation}
  \begin{enumerate}
    \item Listing sets
    \item The universal set $U$
    \item Items in, or not in, multiple sets
  \end{enumerate}
  \end{block}
  Review homework problems 5B p. 211 \\
  Example email conventions \\
  Lesson: 5.3 Set operations \& Venn diagrams \\ \smallskip
  Homework: Textbook exercises 5C p. 220 \\
  Exploration paper \alert{due Friday}
}

\section{3.5 Using tree diagrams 4 Dec}
\frame
{
  \frametitle{GQ: How do we diagram a situation with a tree?}
  \framesubtitle{CCSS: HSS.CP.A.4 Understand conditional probability \hfill \alert{3.5 Wednesday 4 Dec}}

  \begin{block}{Do Now: Conditional probablity from a matrix}
  \begin{enumerate}
    \item What is the probability of each event overall?
    \item What is the probability conditional on each subset?
  \end{enumerate}
  \end{block}
  Review homework problems 5C p. 220 \\
  Lesson: 5.4 Probability and tree diagrams \\ \smallskip
  Homework: Textbook exercises 5D p. 223-4 \\
  Exploration paper \alert{due Friday}
}

\end{document}
