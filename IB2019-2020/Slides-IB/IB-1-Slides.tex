\documentclass{beamer}
\usepackage{geometry}
\usepackage[english]{babel}
\usepackage[utf8]{inputenc}
\usepackage{amsmath}
\usepackage{amsfonts}
\usepackage{amssymb}
\usepackage{tikz}
\usepackage{graphicx}
\usepackage{venndiagram}

%\usepackage{pgfplots}
%\pgfplotsset{width=10cm,compat=1.9}
%\usepackage{pgfplotstable}

\setlength{\headheight}{26pt}%doesn't seem to fix warning

\usepackage{fancyhdr}
\pagestyle{fancy}
\fancyhf{}

%\rhead{\small{5 September 2018}}
\lhead{\small{BECA / Dr. Huson / 11.1 IB Math Unit 1}}

\renewcommand{\headrulewidth}{0pt}

\title{Mathematics Class Slides}
\subtitle{Bronx Early College Academy}
\author{Chris Huson}
\date{5-21 September 2018}

\begin{document}
\frame{\titlepage}
\section[Outline]{}
\frame{\tableofcontents}

  \section{1.1 First day of IB Mathematics 5 Sept}
  \frame
  {
    \frametitle{GQ: How do we define functions?}
    \framesubtitle{CCSS: HSF.IF.C.7 Analyze functions \hfill \alert{1.1 Thursday 5 Sept}}

    \begin{block}{Do Now Handout: Algebra skills check}
    \begin{enumerate}
        \item Welcome back to school!
        \item Assigned seating: arrange yourself alphabetically by last name, left to right, front to back.
        \item Take out notebooks (or blank paper) \& calculator
        \item Complete handout problem set\\*
    \end{enumerate}
    \end{block}
    Lesson: Linear functions, slope, solving; vertical line test p 4-6 \\%*[5pt]
    Homework: Problem set: Function identification 1A \& 1B p. 6-7
  }
%Prepare copies of formula sheets

  \section{1.2 Function domain and range}
  \frame
  {
    \frametitle{GQ: What are domain and range?}
    \framesubtitle{CCSS: HSF.IF.C.7 Analyze functions \hfill \alert{1.2 Friday 6 Sept}}

    \begin{block}{Do Now: Substitution notation}
    \begin{enumerate}
      \item Handout, IB exam problem
      \item Challenge: %2 points Aug 2017
        Verify the following Pythagorean identity for all values of $x$ and $y$:
        \[(x^2+y^2)^2=(x^2-y^2)^2+(2xy)^2\]
    \end{enumerate}
    \end{block}
    Homework review\\
    Lesson: Domain, range, function review\\[5pt]
    Calculator deposits \$20
    \\[5pt]
    Homework: Polynomial simplification, graphing linear functions\\
    Due: notebook, folder, calculator
  }

  \section{1.3 Precision and significant figures, 9 Sept}
  \frame
  {
    \frametitle{GQ: What is the appropriate precision for a calculation?}
    \framesubtitle{CCSS: MP5 Attend to precision \hfill \alert{1.3 Monday 9 Sept}}

    \begin{block}{Do Now: Textbook chapter warmup, use looseleaf paper}
    \begin{enumerate}
        \item Skills check \#1-3 p. 3
    \end{enumerate}
    \end{block}
    Lesson: Rounding, significant figures, error bars pp. 1-5\\
    Exercise 1A, \#1-2, p. 5
    \\[0.5cm]
    Homework: Calculation and rounding practice
  }

  \section{1.4 Error bounds, 10 Sept}
  \frame
  {
    \frametitle{GQ: How do we measure the bounds of errors?}
    \framesubtitle{CCSS: MP5 attend to precision \hfill \alert{1.4 Tuesday 10 Sept}}

    \begin{block}{Do Now: Calculator practice}
    \begin{enumerate}
        \item Chapter review \#1 p. 39
        \item Pay careful attention to saving calculator values, rather than copying to paper and reentering.
        \item Check your answers in back of book, p. 766
    \end{enumerate}
    \end{block}
    Lesson: Bounds and errors pp. 6-8\\ \bigskip
    Practice exercises 1B p. 8-9\\
    Homework: Function substitution, domain and range
  }

  \section{1.5 Exponents \& scientific notation, 11 Sept}
  \frame
  {
    \frametitle{GQ: How do we write very large or small numbers?}
    \framesubtitle{CCSS: MP5 attend to precision \hfill \alert{1.5 Wednesday 11 Sept}}

    \begin{block}{Do Now: Precision practice}
    \begin{enumerate}
        \item Practice exercises 1B p. 8-9
        \item Pay careful attention to saving calculator values, rather than copying to paper and reentering.
        \item Check your answers in back of book, p. 765
    \end{enumerate}
    \end{block}
    Lesson: Exponents \& scientific notation pp. 9-12\\ \smallskip
    Note exponent rules top of page 11\\ \smallskip
    Homework: Practice exercises 1C p. 12-13
  }


  \section{1.6 Right triangle trigonometry, 12 Sept}
  \frame
  {
    \frametitle{GQ: How do we calculate the side lengths of a right triangle?}
    \framesubtitle{CCSS: MP5 attend to precision \hfill \alert{1.6 Thursday 11 Sept}}

    \begin{block}{Do Now: Precision practice}
    \begin{enumerate}
        \item Chapter review \#2 p. 39
        \item Which will be easier to use, scientific notation or the fully expanded number?
        \item Use proper notation to display your answer clearly
    \end{enumerate}
    \end{block}
    Homework review \\
    Lesson: Right triangle trigonometry pp. 13-15\\ \smallskip
    Angle of elevation and depression page 11\\ \smallskip
    Homework: Practice exercises 1D p. 16-17
  }


  \section{1.7 Sine rule, 13 Sept}
  \frame
  {
    \frametitle{GQ: How do we calculate the side lengths of a non-right triangle?}
    \framesubtitle{CCSS: MP5 attend to precision \hfill \alert{1.7 Friday 13 Sept}}

    \begin{block}{Do Now: Precision practice}
    \begin{enumerate}
        \item Chapter review \#3 p. 39
        \item Learn how to use the calculator to solve an equation. (multiple methods)
    \end{enumerate}
    \end{block}
    Lesson: Non-right triangles and the sine rule pp. 17-21\\ \smallskip
    The ambiguous case page 21\\ \smallskip
    Homework: Practice exercises 1E p. 21-22
  }

  \section{1.8 Sine formula for the area of a triangle, 16 Sept}
  \frame
  {
    \frametitle{GQ: How do we calculate the area of a triangle?}
    \framesubtitle{CCSS: MP5 attend to precision \hfill \alert{1.8 Monday 16 Sept}}

    \begin{block}{Do Now: Precision practice}
    \begin{enumerate}
        \item Chapter review \#4b p. 39
        \item Note that both $\displaystyle \frac{15}{\sin 31}=\frac{13.4}{\sin R}$ and $\displaystyle \frac{\sin 31}{15}=\frac{\sin R}{13.4}$. \\ \bigskip
        Which is easier to solve?
    \end{enumerate}
    \end{block}
    Lesson: Practicing applying the sine rule pp. 17-21\\
    The ambiguous case page 21\\
    The sine formula for the area of a triangle page 22\\ \smallskip
    Homework: Practice exercises 1E p. 21-22
  }

  \section{1.9 Deltamath: scientific notation, trig 17 Sept}
  \frame
  {
    \frametitle{GQ: How do we practice the law of sines?}
    \framesubtitle{CCSS: MP5 attend to precision \hfill \alert{1.9 Tuesday 17 Sept}}

    \begin{block}{Deltamath practice: scientific notation, trig}
      \begin{enumerate}
        \item Laptops, login with Teacher ID \alert{546068}
        \item Do Deltamath sections in order \\
        Practice comes first, then new topics
        \item Work extra problems on the skills you need to practice
    \end{enumerate}
    \end{block}
    New material: The sine formula for the area of a triangle page 22\\
    Radian / degree conversion; law of cosines\\ \smallskip
    Homework: Complete Deltamath problems, 10:00PM deadline
  }

  \section{1.10 Cosine rule, 18 Sept}
  \frame
  {
    \frametitle{GQ: How do we calculate the angles of a triangle?}
    \framesubtitle{CCSS: MP5 attend to precision \hfill \alert{1.10 Wednesday 18 Sept}}

    \begin{block}{Do Now: Precision practice}
    \begin{enumerate}
        \item Chapter review \#6 p. 39
    \end{enumerate}
    \end{block}
    Lesson: The cosine rule pp. 23-24\\
    The sine formula for the area of a triangle page 22\\ \smallskip
    Homework: Practice exercises 1F p. 24-25
  }


  \section{1.11 Sine \& cosine rule practice, 19 Sept}
  \frame
  {
    \frametitle{GQ: How do we ``solve" a triangle?}
    \framesubtitle{CCSS: MP5 attend to precision \hfill \alert{1.11 Thursday 19 Sept}}

    \begin{block}{Do Now: IB exam problems}
    \begin{enumerate}
        \item Applications of the sine and cosine rules
    \end{enumerate}
    \end{block}
    Lesson: The cosine rule pp. 23-24\\
    The sine formula for the area of a triangle page 22\\ \smallskip
    Homework: Study Arc length and area of sector \\
    Oxford textbook pp. 25-27 \\
    Deltamath, practice circle sectors and arc length \\
    Khan Academy, log in and use videos as resource (DrHuson)
  }

  \section{1.12 Circle sectors \& arc length, 20 Sept}
  \frame
  {
    \frametitle{GQ: How do we calculate the angles of a triangle?}
    \framesubtitle{CCSS: MP5 attend to precision \hfill \alert{1.12 Friday 20 Sept}}

    Continue IB exam trig problems \\ \bigskip
    Lesson: The cosine rule pp. 23-24\\
    The sine formula for the area of a triangle page 22\\ \smallskip
    Homework: Complete Khan videos and Deltamath problems if you haven't already. \\
    Practice exercises 1G p. 26-27
  }

  \section{1.13 3-dimensional figures, volume, 23 Sept}
  \frame
  {
    \frametitle{GQ: How do we calculate the volumes of objects?}
    \framesubtitle{CCSS: MP5 attend to precision \hfill \alert{1.13 Monday 23 Sept}}

    \begin{block}{Do Now: Developing inquiry skills, top of page 28}
      \begin{enumerate}
          \item Draw a scale model of the surveying of Mt. Everest on IB centimeter graph paper. (use a protractor)
          \item Determine the height of the mountain by measuring your model
          \item Calculate the height using trig formulas
      \end{enumerate}
      \end{block}
  
    Continue IB exam trig problems \\ \bigskip
    Lesson: Solid geometry terminology, volume formulas\\ \smallskip
    Homework: Practice exercises 1H p. 30-31 \\
    Khan videos and Deltamath problems
  }

  \section{1.14 3-dimensional figures, surface area, nets, 24 Sept}
  \frame
  {
    \frametitle{GQ: How do we calculate the volumes of objects?}
    \framesubtitle{CCSS: MP5 attend to precision \hfill \alert{1.14 Tuesday 24 Sept}}

    \begin{block}{Do Now: Calculate the volume of each object}
      \begin{enumerate}
          \item A sphere with a radius of 15 cm
          \item A circular pond 40 meters in diameter with a depth of 20 centimeters
          \item A cone with a height of 2 feet and diameter of 20 inches
          \item Find the radius of a sphere with volume $215 \pi$
      \end{enumerate}
      \end{block}
    Lesson: Solid geometry surface area, nets\\ \smallskip
    Investigation 7, page 32 \\ \bigskip
    Homework: Practice exercises 1I p. 35-36
  }

  \section{1.15 3-dimensional figures, internal angles, 25 Sept}
  \frame
  {
    \frametitle{GQ: How do we calculate the slant angles and lengths?}
    \framesubtitle{CCSS: MP5 attend to precision \hfill \alert{1.15 Wednesday 25 Sept}}

    \begin{block}{Do Now: Calculate the surface area of each object}
      \begin{enumerate}
          \item A cylinder 4 cm in diameter with a height of 5 millimeters
          \item A sphere with a radius of 15 cm
          \item A pyramid with a height of 40 feet and base 50 feet by 50 feet
          \item Find the radius of a sphere with surface area $72 \pi$
      \end{enumerate}
      \end{block}
    Lesson: Solid geometry slant angles and lengths\\ \smallskip
    Investigation 7, page 32 \\ \bigskip
    Homework: Return and complete exercises 1E p. 21-22
  }

  \section{1.16 3-dimensional figures, internal angles, 26 Sept}
  \frame
  {
    \frametitle{GQ: How do we calculate the slant angles and lengths?}
    \framesubtitle{CCSS: MP5 attend to precision \hfill \alert{1.16 Thursday 26 Sept}}

    \begin{block}{Do Now: Calculate the surface area of a cone}
      \begin{enumerate}
          \item Developing inquiry skills p. 38 (Mt. Everest)
          \item What formula would apply?
       \end{enumerate}
      \end{block}
    Lesson: Solid geometry, Chapter Summary\\ \smallskip
    Homework: Chapter review 1-10 p. 39-40 (revisit problems)
  }


  \section{1.17 review, bounds, 27 Sept}
  \frame
  {
    \frametitle{GQ: How do we calculate the bounds around a value?}
    \framesubtitle{CCSS: MP5 attend to precision \hfill \alert{1.17 Friday 27 Sept}}

    \begin{block}{Do Now Quiz: \href{https://www.smartbmicalculator.com/}{Calculate Body Mass Index} (link)}
      \begin{itemize}
          \item BMI is a measure of a healthy personal weight, $\displaystyle BMI = \frac{w}{h^2}$
          \item $w$ is a person's weight in kilograms and $h$ is height in meters 
          \item Given a height of 170 cm and weight of 77 kg, find the BMI
          \item These measurements are not exact. Assuming the height is between 169-171 cm and weight 76-78 kg, find the bounds of the BMI.
       \end{itemize}
      \end{block}
    Lesson: Solid geometry, Chapter Summary\\ \smallskip
    Homework: Chapter review 11-17 p. 39-40 (revisit problems)
  }

\end{document}
