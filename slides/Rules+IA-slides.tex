\documentclass{beamer}
\usepackage{geometry}
\usepackage[english]{babel}
\usepackage[utf8]{inputenc}
\usepackage{amsmath}
\usepackage{amsfonts}
\usepackage{amssymb}
\usepackage{tikz}
\usepackage{graphicx}
\usepackage{venndiagram}

%\usepackage{pgfplots}
%\pgfplotsset{width=10cm,compat=1.9}
%\usepackage{pgfplotstable}

\setlength{\headheight}{26pt}%doesn't seem to fix warning

\usepackage{fancyhdr}
\pagestyle{fancy}
\fancyhf{}

%\rhead{\small{21 May 2018}}
\lhead{\small{BECA / Dr. Huson / Mathematics}}

%\vspace{1cm}

\renewcommand{\headrulewidth}{0pt}


\title{Mathematics Class Slides}
\subtitle{Bronx Early College Academy}
\author{Chris Huson}
\date{27 August [was 23 May] 2018}

\begin{document}

\frame{\titlepage}

\frame
{
  \frametitle{Communication protocols}
  \framesubtitle{How to send information and common conventions to follow}
\begin{itemize}
      \item Mail (``post"), by messenger: formal, fancy, legally secure
      \item Text (cell phone): brief, informal, immediate, transitory
      \item Email: versatile, threaded, transitory or permanent (insecure)\\
      Subject line, salutations, handle@domain
      \item Attachments: .pdf is universally readable, can't be edited\\
      Microsoft Word, Excel, .ppt can be edited, commented
      \item Link: extended collaboration, commercial paywalls\\
      \emph{If you are not paying for it, you're not the customer; you're the product being sold.}
\end{itemize}
 }


\frame
{
  \frametitle{Extra help}
  %\framesubtitle{Practice writing mathematics according to IB requirements, as per IA criteria.}
\begin{itemize}
      \item Algebra 2 Regents prep\\
      7th period pullout to room 414 (sometimes 1st period)\\
      Twice a week\\
      Alesha, Elisabeth, Nicole, Emelyn, Stephen, Mivian, Joshua\\[20pt]
      \item IB Math, exploration paper, general help\\
      Thursday lunch (usually Mondays \& Fridays too)\\
      Thursday after school room 414 (Johnsen \& Guarnaccia)\\
      Saturday (Guarnaccia)
\end{itemize}
 }

 \frame
 {
   \frametitle{Steps for writing technical papers}
   \framesubtitle{Practice writing mathematics according to IB requirements, as per IA criteria.}
 Proposal
 \begin{enumerate}
     \item Define an ``aim," including success criteria.
     \item Outline paper, especially Method including data collection, graphs, formulas; list references
     \item Draft introduction, including rationale and aim.
     \item Structure data tables, sketch graphs, begin formula and algebra (all handwritten, perhaps spreadsheets or Desmos)
     \item Draft Method section text
 \end{enumerate}
 Method
 \begin{enumerate}
     \item Collect data (survey, search, simulation, etc.)
     \item Work interactively with spreadsheets, graphing software, math
     \item Refine Method section, draft results and discussion.
 \end{enumerate}
 Complete mathematics and paper. Proofread carefully. Rewrite. Receive peer feedback. Rewrite. Submit final draft.
 }


 \frame
 {
   \frametitle{Standards for writing technical papers}
   \framesubtitle{Practice writing mathematics according to IB requirements, as per IA criteria.}
 Criterion C: Personal engagement (0-4 points)
 \begin{enumerate}
     \item Address a personal interest; ``make it your own"
     \item Think independently and/or creatively
     \item Present mathematical ideas in your own way
 \end{enumerate}
 Criterion D: Reflection (0-3 points)
 \begin{enumerate}
     \item Review, analyze, and evaluate the mathematics throughout the paper. Go beyond just describing results
     \item Link to the aims, comment on what has been learned, consider limitations, and compare different mathematical approaches
     \item Consider what's next, discuss the implications of results, strengths and weaknesses of approaches, and consider different perspectives
 \end{enumerate}
 }

 \begin{frame}{Technical writing}
     \framesubtitle{Write a short paper answering the query: \\* "How many subsets can be picked from a group of four students?"}
     \begin{enumerate}
         \item Logical, step-by-step explanation, using an example
         \item Precise terminology, succinct: combination, permutation, order (matters), event, sample space, set, subset, with /without replacement, factorial
         \item Notation: algebra symbols, tables, trees, grids
         \item Summary, big-picture, conceptual idea
         \item Audience: student peers
     \end{enumerate}
 \end{frame}

 \frame
 {
   \frametitle{Standard conventions for mathematical notation}
   \framesubtitle{Practice writing mathematics according to IB requirements, as per exam rubrics.}
 \begin{enumerate}
     \item Use the formula sheet.
     \item Chose the appropriate formula (M1).\\*
     (you do not have to copy the formula)
     \item Substitute values correctly (A1).
     \item Solve, showing key steps (A1).\\
     (skip routine algebra if you like)
     \item Write down the exact solution or copy the calculator display. An ellipsis (\ldots) indicates more digits (A1).
     \item Round to 3 significant digits (use $\approx$)(A1).
 \end{enumerate}
 }

 \frame
 {
   \frametitle{Standard conventions for mathematical notation}
   \framesubtitle{Practice writing mathematics according to IB requirements, as per exam rubrics.}
 Examples of key algebraic techniques
 \begin{enumerate}
     \item Setting a quadratic function $=0$
     \item Converting an exponent to a log
     \item Reading a value from a graph
     \item When writing lists, you may write only the first two and the last terms. For example,
 \[\sum_{k=1}^5 3 \cdot 2.25^k =3 + 6.75+\ldots+76.8867\ldots\]
 \[=135.99609\ldots \approx 136\]
 \end{enumerate}
 }


 \frame
 {
   \frametitle{Topics to cover in Reciprocals paper}
   \framesubtitle{Practice writing mathematics according to IB requirements, as per IA criteria.}
 \begin{enumerate}
 \item Asymptotic behavior explained with an example of end behavior.
 \item Translation of an example point ($A \xrightarrow{} A'$)
 \item Translation's impact on asymptotes.
 \item General case of horizontal and vertical translation. Use $h$ and $k$, and vector notation: $\begin{pmatrix} h \\ k \end{pmatrix}$
 \item Define a function and its inverse as a composition resulting in the identity. $(f \circ f^{-1})(x)=x$. Include an example.
 \item Inverse explained with an example point under reflection across $y=x$
 \item Discussion of reversal of the position of $h$ and $k$ in the algebraic representation of the function and its inverse. (perhaps with just example values)
 \end{enumerate}
 }


 \frame
 {
   \frametitle{Aim and rationale}
   \framesubtitle{There are many important and amazing sequences and series in mathematics.}
   $\displaystyle \frac{1}{2}+\frac{1}{4}+\frac{1}{8}+\frac{1}{16}+\frac{1}{32}+\frac{1}{64} \dots$ Zeno's paradox (Greek)\\*[10pt]  $\displaystyle \frac{1}{2}+\frac{1}{3}+\frac{1}{4}+\frac{1}{5}+\frac{1}{6}+\frac{1}{7} \dots$ harmonic series\\*[10pt]
   $\displaystyle \frac{\pi}{4}=1-\frac{1}{3}+\frac{1}{5}-\frac{1}{7}+\frac{1}{9}-\frac{1}{11}+\frac{1}{13}$ \dots Madhava  (India)\\*[10pt]
   %1, 1, 2, 3, 5, 8, 13, 21, 34, 55, 89,... (Fibonacci)\\*[10pt]
   0.99999999...\\*[10pt]
 The aim of this exploration is to discover the patterns, formulas, and rules of geometric series by experimenting and investigating using a spreadsheet to total various example series.\\*[5pt]

 *Sequence: list of numbers. Series: sum of a sequence of numbers. Geometric sequence: consecutive terms have a constant ratio.
 }



 \end{document}
