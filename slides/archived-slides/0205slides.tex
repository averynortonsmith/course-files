\documentclass{beamer}
\usepackage{geometry}
\usepackage[english]{babel}
\usepackage[utf8]{inputenc}
\usepackage{amsmath}
\usepackage{amsfonts}
\usepackage{amssymb}
\usepackage{tikz}
\usepackage{graphicx}
\usepackage{venndiagram}

\setlength{\headheight}{26pt}%doesn't seem to fix warning

\usepackage{fancyhdr}
\pagestyle{fancy}
\fancyhf{}

\rhead{12.1 11.1 11.2}
\lhead{BECA / Dr. Huson / IB Math SL \\* 5 February 2018}

%\vspace{1cm}

\renewcommand{\headrulewidth}{0pt}


\title{Mathematics Class Slides}
\subtitle{Bronx Early College Academy}
\author{Chris Huson}
\date{February 2018}

\begin{document}

\frame{\titlepage}

%\section[Outline]{}
%\frame{\tableofcontents}

\section{12.1 Drui}
\frame
{
  \frametitle{GQ: How do we calculate area with integration?}
  \framesubtitle{CCSS: F.IF.B.6 Calculate \& interpret the rate of change of a function}

  \begin{block}{Do Now}
  \begin{enumerate}
      \item Find $\int{(4x^3-3x+1)}\mathrm{d}x$.
      \item Find $\int e^{5x}\mathrm{d}x$.
      \item Find $\displaystyle \int \frac{1}{3x+1} \mathrm{d}x$.
  \end{enumerate}
  Homework review \#1, 5, 6 p. 302
  \end{block}
  Lesson: Reimann sums and the definite integral\\*[5pt]
  Task: Example 8, page 304\\*[5pt]
  Assessment: Calculator integration \\*[5pt]
  Homework: Exercises 9H evens p. 308  
}

\section{11.1 Drui}
\frame
{
  \frametitle{GQ: How do we organize data using Venn diagrams?}
  \framesubtitle{CCSS: HSS.CP.B.6 Probabilities}

  \begin{block}{Do Now}
  \begin{enumerate}
      \item Find $_5C_1$.
      \item Write down the first 5 rows of Pascal's triangle, and circle the coefficient nCr for $n=5, r=1$.
  \end{enumerate}
  Homework review \#3 p. 67
  \end{block}
  Lesson: Tools for counting, Venn diagrams (p. 68)\\*[5pt]
  Task: unions, intersections, complements of sets\\*[5pt]
  Assessment:  \\*[5pt]
  Homework: p 71-2 3B \#1-6  
}

\begin{frame}{Combinatorics formulas}
    \alert{Combinations}, when order doesn't matter
	$$_nC_r = \frac{n!}{(n-r)! r!} \qquad \text{''n pick r"}$$
    \alert{Permutations}, when order does matter
	$$_nP_r = \frac{n!}{(n-r)!} $$
\end{frame}

\begin{frame}{Definition of theoretical probability}
    The \alert{theoretical probability} of an event $A$ is $\displaystyle \mathrm P(A) = \frac{n(A)}{n(U)}$\\*[10pt]
    \quad where $n(A)$ is the number of ways an event can occur\\*[5pt]
    \quad and $n(U)$ is the total number of possible outcomes (p. 65)\\*[10pt]
    Theoretically, in $n$ trials, one would expect the event to occur $n \times \mathrm P(A)$ times\\*[10pt]
    Probabilities are between 0 and 1, inclusive. $0 \leq \mathrm P(X) \leq 1$
\end{frame}

\begin{frame}{Empirical (experimental) probability}
    The \alert{relative frequency} of an event can be used as an estimate of its probability. $$\displaystyle \mathrm P(A) = \frac{\text{number of occurrences of event } A}{\text{total number of trials}}$$
    The larger the number of trials the more reliable the estimate of probability.
\end{frame}

\begin{frame}{Independence and mutual exclusivity}
    Two events are \alert{independent} if the occurrence of one does not affect the probability of the other. $$\displaystyle \mathrm P(\text{both }A \text{ and }B \text{ occur}) = \mathrm P(A) \times \mathrm P(B)$$
    Two events are \alert{mutually exclusive} if they never occur together. 
    $$\displaystyle \mathrm P(\text{both }A \text{ and }B \text{ occur}) = 0 \qquad \text{and}$$
    $$\mathrm P(\text{either }A \text{ or }B \text{ occur}) = \mathrm P(A) + \mathrm P(B)$$
\end{frame}

\begin{frame}{Venn diagrams}
    \framesubtitle{For organizing compound events}
    
    When two events can occur, and perhaps both, or neither.
    
    \begin{venndiagram2sets}[tikzoptions={scale=1.5}]
    \end{venndiagram2sets}

\end{frame}

\begin{frame}{The union of sets: $A \cup B$}
    That $A$ happens, or $B$ happens, or both
    \begin{venndiagram2sets}[tikzoptions={scale=1.5}]
    \fillA
    \fillB
    \end{venndiagram2sets}
\end{frame}

\begin{frame}{The intersection of sets: $A \cap B$}
    That both $A$ and $B$ happen
    \begin{venndiagram2sets}[tikzoptions={scale=1.5}]
    \fillACapB
    \end{venndiagram2sets}
\end{frame}

\begin{frame}{The addition rule}
    \framesubtitle{That $A$ or $B$ or both occur}
    
    When two events can occur, and perhaps both
    
    \begin{venndiagram2sets}%[labelA={primes}, labelB={evens}, shade =lightgray]%
    %\fillA
    %\fillB
    %\fillACapB
    \end{venndiagram2sets}

    $$P(\text{either }A \text{ or }B \text{ occur}) = P(A) + P(B) - P(\text{both }A \text{ and }B \text{ occur})$$
\end{frame}

\begin{frame}{Vocabulary for probability \& statistics}
    event, experiment, random\\*[5pt]
    probability, P(A), values [0,1]\\*[5pt]
    theoretical, empirical, subjective\\*[5pt]
    sample space, U; frequency, trials\\*[5pt]
    n(U) = number of possibilities\\*[5pt]
    P(A) = n(A)/n(U); expected = n * P
\end{frame}

\end{document}
independent events: P(A&B)=P(A)*P(B) 
dependent events, mutually exclusive
P(A or B)= P(A) + P(B) - P(A&B)
P(A) = n(A)/n(U)