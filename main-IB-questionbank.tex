\documentclass[12pt, oneside]{article}
\usepackage[letterpaper, margin=1in]{geometry}
\usepackage[english]{babel}
\usepackage[utf8]{inputenc}
\usepackage{amsmath}
\usepackage{amsfonts}
\usepackage{amssymb}
\usepackage{tikz}
%\usepackage{tkz-fct}
\usepackage{pgfplots}
\pgfplotsset{width=10cm,compat=1.9}
\usepgfplotslibrary{statistics}
\usepackage{pgfplotstable}
%\usepackage{venndiagram}

\usepackage{fancyhdr}
\pagestyle{fancy}
\fancyhf{}
\rhead{\thepage \\Name: \hspace{1.5in}}
\lhead{BECA / Dr. Huson / 11.1 IB Math SL\\* 7 May 2018 \\*\textbf{IB Questionbank: Sequences, logarithms\\*
}}

\renewcommand{\headrulewidth}{0pt}

\begin{document}


\subsection*{\\Sequences \& series}


\begin{enumerate}


\item In an arithmetic sequence, the first term is 3 and the second term is 7.
\begin{enumerate}
    \item Find the common difference.
        \begin{flushright}[2]\end{flushright}
    \item Find the tenth term.
        \begin{flushright}[2]\end{flushright}
    \item Find the sum of the first ten terms of the sequence. 
        \begin{flushright}[2]\end{flushright}
\end{enumerate}

\item The first three terms of a geometric sequence are $u_1=0.64$, $u_2=1.6$, and $u_3=4$.
\begin{enumerate}
    \item Find the value of $r$.
        \begin{flushright}[2]\end{flushright}
    \item Find the value of $S_6$.
        \begin{flushright}[2]\end{flushright}
    \item Find the least value of $n$
such that $S_n>75000$. 
        \begin{flushright}[3]\end{flushright}
\end{enumerate}

\item Consider a geometric sequence where the first term is 768 and the second term is 576.\\
Find the least value of $n$ such that the $n$th term of the sequence is less than 7.
    \begin{flushright}[6]\end{flushright}

\item In a geometric sequence, the fourth term is 8 times the first term. The sum of the first 10 terms is 2557.5. Find the 10th term of this sequence.
    \begin{flushright}[6]\end{flushright}

\item Three consecutive terms of a geometric sequence are $x-3$, 6, and $x+2$.\\
Find the possible values of $x$.
    \begin{flushright}[6]\end{flushright}

\item An arithmetic sequence has the first term $\ln a$ and a common difference $\ln 3$. The 13th term in the sequence is $8\ln 9$. Find the value of $a$.
    \begin{flushright}[6]\end{flushright}

\item The first three terms of a geometric sequence are $\ln{x^16}$, $\ln{x^8}$, $\ln{x^4}$, for $x>0$.\\
\begin{enumerate}
    \item Find the common ratio.
        \begin{flushright}[3]\end{flushright}
    \item Solve $\displaystyle \sum_{k=1}^{\infty} 2^{5-k} \ln x =64$.
        \begin{flushright}[5]\end{flushright}
    \end{enumerate}

\item The first two terms of an infinite geometric sequence, in order, are $2 \log_2{x}$, $\log_2{x}$, where $x>0$.\\
The first three terms of an arithmetic sequence, in order, are $\log_2{x}$, $\log_2{\left(\frac{x}{2}\right)}$, $\log_2{\left(\frac{x}{4}\right)}$, where $x>0$.\\
Let $S_{12}$ be the sum of the first 12 terms of the arithmetic sequence.
\begin{enumerate}
    \item Find $r$.
        \begin{flushright}[2]\end{flushright}
    \item Show that the sum of the infinite sequence is $4 \log_2{x}$
        \begin{flushright}[2]\end{flushright}
    \item Find $d$, giving your answer as an integer.
        \begin{flushright}[4]\end{flushright}
    \item Show that $S_{12}=12 \log_2 x -66$.
        \begin{flushright}[2]\end{flushright}
    \item Given that $S_{12}$ is equal to half the sum of the infinite geometric sequence, find $x$, giving your answer in the form $2^p$, where $p \in \mathbb{Q}$.
        \begin{flushright}[5]\end{flushright}
    \end{enumerate}

\subsection*{Logarithms (no calculator)}

\item Find the value of each of the following, giving your answer as an integer.
\begin{enumerate}
    \item $\log_6 36$.
        \begin{flushright}[2]\end{flushright}
    \item $\log_6 4 + \log_6 9$.
        \begin{flushright}[2]\end{flushright}
    \item $\log_6 2 - \log_6 12$.
        \begin{flushright}[3]\end{flushright}
\end{enumerate}

\item 
\begin{enumerate} \item Write down the value of
    \begin{enumerate}
    \item $\log_3 27$.
        \begin{flushright}[1]\end{flushright}
    \item $\log_8 \frac{1}{8}$.
        \begin{flushright}[1]\end{flushright}
    \item $\log_{16} 4$.
        \begin{flushright}[1]\end{flushright}
    \end{enumerate}
    \item Hence, solve $\log_3 27 + \log_8 \frac{1}{8} - \log_{16} 4 = \log_{4} x$
        \begin{flushright}[3]\end{flushright}
\end{enumerate}

\item Let $x=\ln 3$ and $y= \ln 5$. Write the following expressions in terms of $x$ and $y$.
\begin{enumerate}
    \item $\ln \left( \frac{5}{3} \right)$.
        \begin{flushright}[2]\end{flushright}
    \item $\ln 45$.
        \begin{flushright}[4]\end{flushright}
\end{enumerate}

\item Let $x=\ln 7$ and $y= \ln 3$. Write the following expressions in terms of $x$ and $y$.
\begin{enumerate}
    \item $\ln \left( \frac{3}{7} \right)$.
        \begin{flushright}[2]\end{flushright}
    \item $\ln 63$.
        \begin{flushright}[4]\end{flushright}
\end{enumerate}

\item 
\begin{enumerate}
    \item Given that $2^m=8$ and $2^n=16$, write down the value of $m$ and of $n$.
        \begin{flushright}[2]\end{flushright}
    \item Hence or otherwise solve $8^{2x+1}=16^{2x-3}$.
        \begin{flushright}[4]\end{flushright}
\end{enumerate}


\item Let $\log_3 p = 6$ and $\log_3 q =7$
\begin{enumerate}
    \item Find $\log_3 p^2$.
        \begin{flushright}[2]\end{flushright}
    \item Find $\log_3 \left( \frac{p}{q} \right)$.
        \begin{flushright}[2]\end{flushright}
    \item Find $\log_3 (9p)$
        \begin{flushright}[3]\end{flushright}
\end{enumerate}

\item 
\begin{enumerate}
    \item Write the expression $3 \ln 2 - \ln 4$ in the form $\ln k$, where $k \in \mathbb{Z}$.
        \begin{flushright}[3]\end{flushright}
    \item Hence or otherwise, solve $3 \ln 2 - \ln 4=-\ln x$.
        \begin{flushright}[3]\end{flushright}
\end{enumerate}

\item 
\begin{enumerate}
    \item Find the value of $\log_2 40 - \log_2 5$.
        \begin{flushright}[3]\end{flushright}
    \item Find the value of $8^{\log_2 5}$.
        \begin{flushright}[4]\end{flushright}
\end{enumerate}

\item 
\begin{enumerate}
    \item Find $\log_2 32$.
        \begin{flushright}[1]\end{flushright}
    \item Given that $\displaystyle \log_2 \left( \frac{32^x}{8^y} \right)$ can be written as $px+qy$, find the value of $p$ and of $q$.
        \begin{flushright}[4]\end{flushright}
\end{enumerate}

\item Solve $\log_2 x + \log_2 (x-2) = 3$, for $x>2$.
    \begin{flushright}[7]\end{flushright}

\item Let $f(x)= 3 \ln x$ and $g(x)= \ln 5x^3$.
\begin{enumerate}
    \item Express $g(x)$ in the form $f(x)+ \ln a$, where $a \in Z^+$.
        \begin{flushright}[4]\end{flushright}
    \item The graph of $g$ is a transformation of the graph of $f$. Give a full geometric description of this transformation.
        \begin{flushright}[3]\end{flushright}
\end{enumerate}

\item Let $f(x)= k \log_2 x$.
\begin{enumerate}
    \item Given that $f^{-1}(1) = 8$, find the value of $k$.
        \begin{flushright}[3]\end{flushright}
    \item Find $f^{-1}(\frac{2}{3})$
        \begin{flushright}[4]\end{flushright}
\end{enumerate}

\item Let $f(x)= \log_3 \sqrt{x}$, for $x>0$.
\begin{enumerate}
    \item Show that $f^{-1}(x) = 3^{2x}$.
        \begin{flushright}[2]\end{flushright}
    \item Write down the range of $f^{-1}$.
        \begin{flushright}[1]\end{flushright}
    \item Let $g(x)= \log_3 x$, for $x>0$.\\
        Find $\left( f^{-1} \circ g \right) (2)$, giving your answer as an integer.
        \begin{flushright}[4]\end{flushright}
\end{enumerate}

\item Let $f(x)= e^{x+3}$.
\begin{enumerate}
    \item
    \begin{enumerate}
    \item Show that $f^{-1}(x) = \ln x - 3$.
        \begin{flushright}[3]\end{flushright}
    \item Write down the domain of $f^{-1}$.
    \end{enumerate}
    \item Solve the equation $f^{-1}(x) = \ln \frac{1}{x}$.
        \begin{flushright}[4]\end{flushright}
\end{enumerate}

\item Let $f(x)= \log_3 \frac{x}{2}+ \log_3 16 - \log_3 4$, for $x>0$.
[calculator allowed]
\begin{enumerate}
    \item Show that $f(x)= \log_3 2x$.
        \begin{flushright}[2]\end{flushright}
    \item Find the value of $f(0.5)$ and $f(4.5)$.
        \begin{flushright}[3]\end{flushright}
    \item The function $f$ can also be written in the form $f(x)= \log_3 \frac{\ln ax}{\ln b}$
    \begin{enumerate}
    \item Write down the value of $a$ and $b$.
        %\begin{flushright}[3]\end{flushright}
    \item Hence on graph paper, sketch the graph of $f$, for $-5 \leq x \leq 5, -5 \leq y \leq 5$, using a scale of 1 cm to 1 unit on each axis.
        %\begin{flushright}[3]\end{flushright}
    \item Write down the equation of the asymptote.
        \begin{flushright}[6]\end{flushright}
    \end{enumerate}
    \item Write down the value of $f^{-1}(0)$.
        \begin{flushright}[1]\end{flushright}
    \item The point $A$ lies on the graph of $f$. At $A$, $x=4.5$.\\
        On your diagram, sketch the graph of $f^{-1}$, noting clearly the image of point $A$.
        \begin{flushright}[4]\end{flushright}
\end{enumerate}

\subsection*{Graphing calculator equation solving}

\item Solve the equation $e^x = 4 \sin x$, for $0 \leq x \leq 2 \pi$.
    \begin{flushright}[5]\end{flushright}

\item Let $f(x)=4x - e^{x-2} -3$, for $0 \leq x \leq 5$.\\
    Find the $x$-intercepts of the graph of $f$.
    \begin{flushright}[3]\end{flushright}

\end{enumerate}
\end{document}}