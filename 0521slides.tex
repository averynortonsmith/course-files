\documentclass{beamer}
\usepackage{geometry}
\usepackage[english]{babel}
\usepackage[utf8]{inputenc}
\usepackage{amsmath}
\usepackage{amsfonts}
\usepackage{amssymb}
\usepackage{tikz}
\usepackage{graphicx}
\usepackage{venndiagram}

%\usepackage{pgfplots}
%\pgfplotsset{width=10cm,compat=1.9}
%\usepackage{pgfplotstable}

\setlength{\headheight}{26pt}%doesn't seem to fix warning

\usepackage{fancyhdr}
\pagestyle{fancy}
\fancyhf{}

%\rhead{\small{21 May 2018}}
\lhead{\small{BECA / Dr. Huson / Mathematics}}

%\vspace{1cm}

\renewcommand{\headrulewidth}{0pt}


\title{Mathematics Class Slides}
\subtitle{Bronx Early College Academy}
\author{Chris Huson}
\date{21 May 2018}

\begin{document}

\frame{\titlepage}

%\section[Outline]{}
%\frame{\tableofcontents}

\section{11.2 Algebra II Drui}
\frame
{
  \frametitle{How do we use and interpret polynomial functions?}
  \framesubtitle{HSF.LB.B.5 Interpret the parameters in an exponential function in context \qquad \alert{11.2}}

  \begin{block}{Do Now: Exam review problems}
  \begin{enumerate}
    \item Handout practice of graphing and solving%Evaluate $f(x)=\sqrt{x^2}$ for $x=-2, -1, 0, 1, 2$
    %\item Sketch the function for $x \in \mathbb{R}$.
    %\item Write down another representation of $f(x)$.
    %\item Solve for $a$ where $a^x=(e^{0.03925})^x$
    \end{enumerate}
  \end{block}
  Lesson: polynomial modeling\\*
  Task: Regents practice - Complete mastery grid cells\\*
  Assessment: Accurate and efficient use of calculator graphing functions \\*
  Homework: Function graphing practice\\}


\section{11.1 IB Math SL Drui}
\frame
{
  \frametitle{How do we apply trig to solve problems?}
  \framesubtitle{HSG Define trigonometric ratios and solve problems involving right triangles \qquad \alert{11.1}}

\begin{block}{Do Now Quiz: Right triangle relationships\\
    \emph{Complete on lined paper to hand in. Work independently.}}
  \begin{enumerate}
    \item Sketch a $45^\circ, 45^\circ, 90^\circ$ triangle in standard position (with the right angle in the bottom right), label the legs of unit length.
    \item Find the length of the hypotenuse.
    \item Write down the values of $\sin{45^\circ}, \cos{45^\circ}$ 
    \item Sketch an equilateral triangle with sides of length 2. Divide it into two $30^\circ, 60^\circ, 90^\circ$ triangles.
    \item Find the altitude.
    \item Write down $\sin{30^\circ}, \sin{60^\circ}, \cos{30^\circ}, \cos{60^\circ}$
    %\item Given $\sin{x}=\frac{4}{5}$. Find $\cos{x}$
    %\item Draw a figure to explain why $\sin{x}=\cos{(90^\circ-x)}$
    \end{enumerate}
  \end{block}
  Lesson: Compass bearings \& elevations p. 369-373\\*
  %Task: Textbook notes p. 362-9 \\*
  Assessment: Exercise 11C \#1 p. 372\\*
  Homework: Exercises 11C p. 372-3\\
}

\frame
{
  \frametitle{Exercise 11C \#1}
  %\framesubtitle{}
  Isosceles triangle $ABC$ has base $AC = 10 \text{ cm}$ and sides $AB=CB=15 \text{ cm}$.
\begin{itemize}
      \item Find the height of the triangle
      \item Find the sizes of $B\hat{A}C$ and $A\hat{B}C$
\end{itemize}
\begin{center}
\begin{tikzpicture}
\draw (0,0) node[anchor=north]{$A$}
  -- (3,0) node[anchor=north]{$C$}
  -- (1.5,4.2) node[anchor=south]{$B$}
  -- cycle;
  \draw [dashed] (1.5,0) -- (1.5,4.2);
\end{tikzpicture}
\end{center}
 }

\frame
{
  \frametitle{Unit circle}
  %\framesubtitle{}
  Circle with radius of one centered on the origin.
%\begin{itemize}
      %\item Find the height of the triangle
      %\item Find the sizes of $B\hat{A}C$ and $A\hat{B}C$
%\end{itemize}
\begin{center}
\begin{tikzpicture}[scale=3]
  \draw[font=\scriptsize]
    (-1.2, 0) -- (1.2, 0)
    (0, -1.1) -- (0, 1.1)
    %(0, 0) -- (0.866, .5)
    (0, 0) circle[radius=1]
    (-1, 0) node[below left] {$(-1,0)$}
    (1, 0) node[above right] {$(1,0)$}
    (1.1, 0) node[below right] {$\large{x}$}
    ;
\end{tikzpicture}
\end{center}
 }

\frame
{
  \frametitle{Communication protocols}
  \framesubtitle{How to send information and common conventions to follow}
\begin{itemize}
      \item Mail (``post"), by messenger: formal, fancy, legally secure
      \item Text (cell phone): brief, informal, immediate, transitory
      \item Email: versatile, threaded, transitory or permanent (insecure)\\
      Subject line, salutations, handle@domain
      \item Attachments: .pdf is universally readable, can't be edited\\
      Microsoft Word, Excel, .ppt can be edited, commented
      \item Link: extended collaboration, commercial paywalls\\
      \emph{If you are not paying for it, you're not the customer; you're the product being sold.}
\end{itemize}
 }


\frame
{
  \frametitle{Extra help}
  %\framesubtitle{Practice writing mathematics according to IB requirements, as per IA criteria.}
\begin{itemize}
      \item Algebra 2 Regents prep\\
      7th period pullout to room 414 (sometimes 1st period)\\
      Twice a week\\
      Alesha, Elisabeth, Nicole, Emelyn, Stephen, Mivian, Joshua\\[20pt]
      \item IB Math, exploration paper, general help\\
      Thursday lunch (usually Mondays \& Fridays too)\\
      Thursday after school room 414 (Johnsen \& Guarnaccia)\\
      Saturday (Guarnaccia)
\end{itemize}
 }

\section{11.1 IB Math SL Drui}
\frame
{
  \frametitle{Graphing on a calculator to solve equations}

\begin{block}{To solve an equation, separate the equality into two functions, \[f(x)=g(x)\]The intersection of their graphs is the solution.}
  \begin{itemize}
      \item Solve for $x$: $|x-1|-3 = -2x^2+x+3$.
      \item As a check, first make a quick sketch of the functions.
      %\item Use a graphing calculator to compare two functions: $y=1500 \times e^{(10x)}$ and $y=3000$.
      %\item For what $x$ are the functions equal? What does this point represent?
  \end{itemize}
  \end{block}
  Notes: \\Learn to resize the calculator window efficiently\\
  Use the calculator's graph-solve function\\
  Even simple equations can be solved this way. e.g. $e^{0.12x} = 5.25$
}


\frame
{
  \frametitle{Steps for writing technical papers}
  \framesubtitle{Practice writing mathematics according to IB requirements, as per IA criteria.}
Proposal
\begin{enumerate}
    \item Define an ``aim," including success criteria. 
    \item Outline paper, especially Method including data collection, graphs, formulas; list references
    \item Draft introduction, including rationale and aim.
    \item Structure data tables, sketch graphs, begin formula and algebra (all handwritten, perhaps spreadsheets or Desmos)
    \item Draft Method section text
\end{enumerate}
Method
\begin{enumerate}
    \item Collect data (survey, search, simulation, etc.)
    \item Work interactively with spreadsheets, graphing software, math
    \item Refine Method section, draft results and discussion.
\end{enumerate}
Complete mathematics and paper. Proofread carefully. Rewrite. Receive peer feedback. Rewrite. Submit final draft.
}


\frame
{
  \frametitle{Standards for writing technical papers}
  \framesubtitle{Practice writing mathematics according to IB requirements, as per IA criteria.}
Criterion C: Personal engagement (0-4 points)
\begin{enumerate}
    \item Address a personal interest; ``make it your own"
    \item Think independently and/or creatively
    \item Present mathematical ideas in your own way 
\end{enumerate}
Criterion D: Reflection (0-3 points)
\begin{enumerate}
    \item Review, analyze, and evaluate the mathematics throughout the paper. Go beyond just describing results
    \item Link to the aims, comment on what has been learned, consider limitations, and compare different mathematical approaches
    \item Consider what's next, discuss the implications of results, strengths and weaknesses of approaches, and consider different perspectives
\end{enumerate}
}

\begin{frame}{Technical writing}
    \framesubtitle{Write a short paper answering the query: \\* "How many subsets can be picked from a group of four students?"}
    \begin{enumerate}
        \item Logical, step-by-step explanation, using an example
        \item Precise terminology, succinct: combination, permutation, order (matters), event, sample space, set, subset, with /without replacement, factorial
        \item Notation: algebra symbols, tables, trees, grids
        \item Summary, big-picture, conceptual idea
        \item Audience: student peers
    \end{enumerate}
\end{frame}

\frame
{
  \frametitle{Standard conventions for mathematical notation}
  \framesubtitle{Practice writing mathematics according to IB requirements, as per exam rubrics.}
\begin{enumerate}
    \item Use the formula sheet.
    \item Chose the appropriate formula (M1).\\*
    (you do not have to copy the formula)
    \item Substitute values correctly (A1). 
    \item Solve, showing key steps (A1).\\
    (skip routine algebra if you like)
    \item Write down the exact solution or copy the calculator display. An ellipsis (\ldots) indicates more digits (A1).
    \item Round to 3 significant digits (use $\approx$)(A1).
\end{enumerate}
}

\frame
{
  \frametitle{Standard conventions for mathematical notation}
  \framesubtitle{Practice writing mathematics according to IB requirements, as per exam rubrics.}
Examples of key algebraic techniques
\begin{enumerate}
    \item Setting a quadratic function $=0$
    \item Converting an exponent to a log
    \item Reading a value from a graph
    \item When writing lists, you may write only the first two and the last terms. For example,
\[\sum_{k=1}^5 3 \cdot 2.25^k =3 + 6.75+\ldots+76.8867\ldots\]
\[=135.99609\ldots \approx 136\]
\end{enumerate}
}


\frame
{
  \frametitle{Descriptive statistics terminology}
  \framesubtitle{Make a list of these terms, find their definitions in the textbook.}
  
  Univariate data, bivariate\\*
  Population, sample, random/biased sample, survey, census\\*
  Discrete/continuous data, quantitative/qualitative\\* 
  Central tendency, mean ($\overline{x}, \mu$), median, mode; quartiles, percentiles\\*
  5-figure summary, box \& whisker plots, range, interquartile range, outlier\\*
  Dispersion, standard deviation ($\sigma$), variance ($v=\sigma^2$)\\*
  Frequency distributions (tables/bar charts/histograms)\\*
  Grouped data, class, mid-interval value, boundaries, modal class\\*
  Cumulative frequency distributions

  
}

\frame
{
  \frametitle{Bias and fairness, random variation, \& combinations}
  \framesubtitle{When rolling two dice, why aren't all the possible totals equally likely?}
  Definition:\\*
  A \alert{fair} (p. 67) or \alert{unbiased} (p. 79) process \\*[15pt]
  In mathematics we usually simplify and assume a random process follows exact, idealized probabilities. For example, we assume heads and tails are equally likely results of a coin toss.
  
}

\frame
{
  \frametitle{Bias and fairness, random variation, \& combinations}
  \framesubtitle{When rolling two dice, why aren't all the possible totals equally likely?}
  Definition:\\*
  \alert{Experimental} or \alert{empirical} (p. 65) results \\*[15pt]
  In real life, the results of any experiment have a degree of \alert{random variation}. The observed relative frequencies are estimates of the underlying theoretical probabilities, which grow more accurate with additional trials.
  
}

\frame
{
  \frametitle{Bias and fairness, random variation, \& combinations}
  \framesubtitle{When rolling two dice, why aren't all the possible totals equally likely?}
  Counting events in a \alert{sample space} (p. 78) or calculating \alert{combinations} (p. 184) \\*[15pt]
  The six possible results of rolling a single die are equally likely, $\mathrm P(x)=\frac{1}{6}$, if we assume the die is fair. Similarly, the probability of any of the 36 $(6 \times 6)$ possible results of rolling two dice are equally likely, $\mathrm P(x)=(\frac{1}{6})^2$. However, the probability of a particular total varies according to how many combinations lead to that total. Thus, for example, 7 can be rolled six different ways, so $\mathrm P(7)=\frac{6}{36}$, while 2 can only result one way, $\mathrm P(2)=\frac{1}{36}$.
  
}

\frame
{
  \frametitle{Sets, subsets, \& proper subsets}
  
  Definitions:\\*
  A \alert{set} is an unordered collection of elements.\\ e.g. \{red, white, blue\} (do not repeat elements)\\*[5pt]
  \alert{Subset}: Set $A$ is a subset of set $B$ if and only if all of the elements of $A$ are elements of $B$.\\
  Written: $A \subseteq B$\\[5pt]
  \alert{Proper subset}: $A \subseteq B$ and $A$ is not equal to $B$. Written: $A \subset B$\\[5pt]
  The \alert{empty set} is a subset of all sets. $\{\} \text{ or } \emptyset$
  
}


\frame
{
  \frametitle{GQ: Combinatorics problem}
  \framesubtitle{CCSS: F.IF.B.6 Calculate \& interpret the rate of change of a function}

  \begin{block}{Show the formula and then use your calculator function}
  \begin{enumerate}
      \item You have a \$1 bill, a \$5 bill, a \$10 bill, a \$20 bill, a quarter, a dime, a nickel, and a penny. How many different total amounts can you make by choosing six bills and coins?
  \end{enumerate}
  \end{block}
  What is the number of the set you are choosing from?\\%*[5pt]
  How many are you picking?\\%*[5pt]
  Does their order matter?
}

\begin{frame}{Do Now \#1: Phone preferences by gender}
    \framesubtitle{Given the frequency table, make a Venn diagram}
    \begin{tabular}{l|c|r|}
        & Android & iPhone\\ 
        \hline 
        Boys & 15 & 5 \\ 
        \hline 
        Girls & 5 & 15 \\
        \hline 
    \end{tabular}\\*[10pt]
    \centering
    $A=\{ \text{prefers Android}\}$ and $B=\{ \text{is a boy}\}$
    \begin{venndiagram2sets}[tikzoptions={scale=1.0}]
    \end{venndiagram2sets}
\end{frame}

\begin{frame}{Do Now \#2: Independence}
    \framesubtitle{Given the situation, make a Venn diagram, frequency table, and tree representing}
    $\mathrm{P}(A)=0.6$, $\mathrm{P}(B)=0.5$, $\mathrm{P}(A \cap B)=0.3$
    \centering
    \begin{venndiagram2sets}[tikzoptions={scale=1.0}]
    \end{venndiagram2sets}\\*[10pt]
    \begin{tabular}{l|c|r|}
        & $A$ & $A^\prime$\\ 
        \hline 
        $B$ &  \qquad \qquad &  \qquad \qquad \\ 
        \hline 
        $B^\prime$ &  &  \\
        \hline 
    \end{tabular}\\*[10pt]
\end{frame}

\begin{frame}{$\mathrm P(A \cup B) = \mathrm P(A) + \mathrm P(B) - \mathrm P(A \cap B)$}
    \framesubtitle{The addition rule}
    \begin{venndiagram2sets}[tikzoptions={scale=2}]
    \end{venndiagram2sets}
\end{frame}

\begin{frame}{Distributions}
    \framesubtitle{Tables and charts used to summarize a problem situation}
    A \alert{frequency distribution} displays the number of times each event in the sample space occurs, either in tabular or graphical form.\\*[10pt]
    A \alert{probability distribution} shows the same data, normalizing the totals to one.
\end{frame}


\begin{frame}{Combinatorics formulas}
    \alert{Combinations}, when order doesn't matter
	$$_nC_r = \frac{n!}{(n-r)! r!} \qquad \text{''n pick r"}$$
    \alert{Permutations}, when order does matter
	$$_nP_r = \frac{n!}{(n-r)!} $$
\end{frame}

\begin{frame}{Definition of theoretical probability}
    The \alert{theoretical probability} of an event $A$ is $\displaystyle \mathrm P(A) = \frac{n(A)}{n(U)}$\\*[10pt]
    \quad where $n(A)$ is the number of ways an event can occur\\*[5pt]
    \quad and $n(U)$ is the total number of possible outcomes (p. 65)\\*[10pt]
    Theoretically, in $n$ trials, one would expect the event to occur $n \times \mathrm P(A)$ times\\*[10pt]
    Probabilities are between 0 and 1, inclusive. $0 \leq \mathrm P(X) \leq 1$
\end{frame}

\begin{frame}{Empirical (experimental) probability}
    The \alert{relative frequency} of an event can be used as an estimate of its probability. $$\displaystyle \mathrm P(A) = \frac{\text{number of occurrences of event } A}{\text{total number of trials}}$$
    The larger the number of trials the more reliable the estimate of probability.
\end{frame}

\begin{frame}{Independence and mutual exclusivity}
    Two events are \alert{independent} if the occurrence of one does not affect the probability of the other. $$\displaystyle \mathrm P(\text{both }A \text{ and }B \text{ occur}) = \mathrm P(A) \times \mathrm P(B)$$
    Two events are \alert{mutually exclusive} if they never occur together. 
    $$\displaystyle \mathrm P(\text{both }A \text{ and }B \text{ occur}) = 0 \qquad \text{and}$$
    $$\mathrm P(\text{either }A \text{ or }B \text{ occur}) = \mathrm P(A) + \mathrm P(B)$$
\end{frame}

\begin{frame}{Venn diagrams}
    \framesubtitle{For organizing compound events}
    When two events can occur, and perhaps both, or neither.
    \begin{venndiagram2sets}[tikzoptions={scale=1.5}]
    \end{venndiagram2sets}
\end{frame}

\begin{frame}{The union of sets: $A \cup B$}
    That $A$ happens, or $B$ happens, or both
    \begin{venndiagram2sets}[tikzoptions={scale=1.5}]
    \fillA
    \fillB
    \end{venndiagram2sets}
\end{frame}

\begin{frame}{The intersection of sets: $A \cap B$}
    That both $A$ and $B$ happen
    \begin{venndiagram2sets}[tikzoptions={scale=1.5}]
    \fillACapB
    \end{venndiagram2sets}
\end{frame}

\begin{frame}{The addition rule}
    \framesubtitle{That $A$ or $B$ or both occur}
    
    When two events can occur, and perhaps both
    
    \begin{venndiagram2sets}%[labelA={primes}, labelB={evens}, shade =lightgray]%
    %\fillA
    %\fillB
    %\fillACapB
    \end{venndiagram2sets}

    $$P(\text{either }A \text{ or }B \text{ occur}) = P(A) + P(B) - P(\text{both }A \text{ and }B \text{ occur})$$
\end{frame}

\begin{frame}{Vocabulary for probability \& statistics}
    event, experiment, random\\*[5pt]
    probability, P(A), values [0,1]\\*[5pt]
    theoretical, empirical, subjective\\*[5pt]
    sample space, U; frequency, trials\\*[5pt]
    n(U) = number of possibilities\\*[5pt]
    P(A) = n(A)/n(U); expected = n * P
\end{frame}


\end{document}
