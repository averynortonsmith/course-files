\documentclass{article}
\usepackage[english]{babel}
\usepackage[utf8x]{inputenc}
\usepackage{amsmath}
\usepackage{graphicx}


\title{12.1 Tests}
\author{chris }
\date{September 2017}

\begin{document}


\noindent BECA / Huson / 12.1 IB Math SL \qquad \qquad Name:\\
18 September 2017
\subsection*{Pre-test: Introduction to differential calculus}
Show working for all problems. State answers exactly or to three significant figures.

\begin{enumerate}

\item Write down the derivative of the function $f(x) = x^2 + 3x + 4$.

\item	A function is given as $y = ax^2 + bx + 6$.

\begin{itemize}
    \item[(a)] Find $\displaystyle \frac {dy}{dx}$.
	\item[(b)] If the gradient of this function is 2 when $x$ is 6, write an equation in terms of $a$ and $b$.
	\item[(c)] If the point $(3, -15)$ lies on the graph of the function find a second equation in terms of $a$ and $b$.
	\item[(d)] Solve for $x$ in terms of $a$ and $b$: 
	\[\frac {dy}{dx}=0\]
	Where have you seen this expression before?
\end{itemize}

\item Find $f'(x)$  for the following function. Express your final result without negative exponents:
	 \[	f(x) = \frac{x^3-4x-8}{x}\]
	 
\item Find the equation of the tangent to $\displaystyle f(x) = \frac{4}{x^2}$ when $x = 1$.
	 
\item Consider the function $f(x) = x^3 - 4x^2 - 3x + 18$.
\begin{itemize}
    \item[(a)] Find $\displaystyle \frac {dy}{dx}$.
	\item[(b)] Find the values of $f(x)$ for $a$ and $b$ in the table below:\\
	\begin{tabular}{|l|c|c|c|c|c|c|c|c|c|}
	\hline
	$x$ & -3 & -2 & -1 & 0 & 1 & 2 & 3 & 4 & 5\\
	\hline
    $f(x)$ & -36 & $a$ & 16 & $b$ & 12 & 4 & 0 & 6 & 28\\
	\hline
	\end{tabular}
	\item[(c)] Using a scale of 1 cm for each unit on the $x$-axis and 1 cm for each 5 units on the $y$-axis, draw the graph of $f(x)$ for $-3 \leq x \leq 5$. Label it clearly using IB conventions.
	\item[(d)] The gradient of the curve at any particular point varies. Within the interval $-3 \leq x \leq 5$, state all the intervals over which the gradient of the curve is
    \begin{itemize}
    \item[(i)] Negative
    \item[(ii)] Positive
    \end{itemize}

\end{itemize}

\end{enumerate}

\end{document}
