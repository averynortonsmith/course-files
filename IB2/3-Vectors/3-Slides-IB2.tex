\documentclass{beamer}
\usepackage{geometry}
\usepackage[english]{babel}
\usepackage[utf8]{inputenc}
\usepackage{amsmath}
\usepackage{amsfonts}
\usepackage{amssymb}
\usepackage{tikz}
\usepackage{graphicx}
\usepackage{venndiagram}

%\usepackage{pgfplots}
%\pgfplotsset{width=10cm,compat=1.9}
%\usepackage{pgfplotstable}

\setlength{\headheight}{26pt}%doesn't seem to fix warning

\usepackage{fancyhdr}
\pagestyle{fancy}
\fancyhf{}

%\rhead{\small{24 September 2018}}
\lhead{\small{BECA / Dr. Huson / 12.1 IB Math}}

%\vspace{1cm}

\renewcommand{\headrulewidth}{0pt}


\title{Mathematics Class Slides}
\subtitle{Bronx Early College Academy}
\author{Chris Huson}
\date{22 October - 2 November 2018}

\begin{document}

\frame{\titlepage}

\section[Outline]{}
\frame{\tableofcontents}


\section{3.1 Drui - Intro to Vectors, Monday Oct 22}
  \frame
  {
    \frametitle{GQ: What are the basic elements of vector algebra?}
    \framesubtitle{CCSS: HSF.IF.B.4 Interpret key features of functions and their graphs \qquad \alert{3.1}}

    \begin{block}{Do Now: Differentiate each function.}
    \begin{enumerate}
        \item $f(x)=x^3-4x$
        \item $g(x)=\ln x$
        \item $y=(x^3-4x)(\ln x)$
        \item Write down the value of $\displaystyle \cos \frac{\pi}{3}$. (sketch first. no calculator)
        \item Skills check \#1-3 p. 404-5
    \end{enumerate}
    \end{block}
    Lesson: Vector concepts and notation pp. 406-410\\ \bigskip
    Homework: Exercises 12A pp. 410-411
  }

\section{3.2 Drui - Deltamath differentiation practice, Tuesday Oct 23}
  \frame
  {
    \frametitle{GQ: How do we differentiate functions?}
    \framesubtitle{CCSS: HSF-IF.B.6 Interpret functions, and their rate of change \hspace{\stretch{1}} \alert{3.2}}

    \begin{block}{Do Now quiz, mixed review.}
      Complete the problem set without a calculator, then begin Deltamath.
    \end{block} \bigskip

    Lesson: Differentiation practice \\ \bigskip
    Homework: Complete Deltamath problem set at home
  }

\section{3.3 Drui - Resultant vectors, Wednesday Oct 24}
  \frame
  {
    \frametitle{GQ: How do we add vectors?}
    \framesubtitle{CCSS: HSF.IF.B.4 Interpret key features of functions and their graphs \qquad \alert{3.3}}

    \begin{block}{Do Now: Differentiate each function. $k \in \mathbb{R}$}
    \begin{enumerate}
        \item $f(x)=k x^2-4x^{-1}$
        \item $g(x)=\ln kx$
        \item $h(x)=e^{kx}$
        \item Write down the value of $\displaystyle \cos \frac{2 \pi}{3}$. (sketch first. no calculator)
    \end{enumerate}
    \end{block}
    Lesson: Parallelism, scalar multiplication, position, addition pp. 411-417\\ \bigskip
    Homework: Exercises 12B (odds), 12C (odds), pp. 410-416
  }

\section{3.4 Drui - Distance, Thursday Oct 25}
  \frame
  {
    \frametitle{GQ: How do we calculate distance in space?}
    \framesubtitle{CCSS: HSG.SRT.C.8 Use the Pythagorean theorem to solve applied problems \qquad \alert{3.4}}

    \begin{block}{Do Now: Differentiate each function.}
    \begin{enumerate}
        \item $f(x)=x^{-1}-4x^{-2}$
        \item $g(x)=\sin x^2$
        \item $y=(x^3-4x) \div (\ln x)$
        \item Write down the value of $\displaystyle \sin \frac{\pi}{4}$. (sketch first. no calculator)
    \end{enumerate}
    \end{block}
    Lesson: Length calculation using the Pythagorean formula, unit vectors pp. 418-419\\ \bigskip
    Homework: Exercises 12D p. 417, 12F odds p. 420.
  }

\section{3.5 Drui - Unit vectors, Friday Oct 26}
  \frame
  {
    \frametitle{GQ: How do we calculate distance in space?}
    \framesubtitle{CCSS: HSG.SRT.C.8 Use the Pythagorean theorem to solve applied problems \qquad \alert{3.5}}

    \begin{block}{Do Now: Let $\displaystyle g(x)=\frac{\ln x}{x^2}$ for $x>0$.}
    \begin{enumerate}
        \item Use the quotient rule to show that $\displaystyle g'(x)=\frac{1-2\ln x}{x^3}$
        \item The graph of $g$ has a maximum point at $A$. Find the $x$-coordinate of $A$.
        \item Given the point $P(4,5)$. State the position vector $\overrightarrow{OP}$ in unit vector form.
        \item Find the magnitude of $\overrightarrow{OP}$.
    \end{enumerate}
    \end{block}
    Lesson: Collinear points, unit vectors pp. 418-419\\ \bigskip
    Homework: Exercises 12E p. 418.
  }

\section{3.6 Drui - Unit vectors, Monday Oct 29}
  \frame
  {
    \frametitle{GQ: How do we add vectors?}
    \framesubtitle{CCSS: HSG.SRT.C.8 Use the Pythagorean theorem to solve applied problems \qquad \alert{3.6}}

    \begin{block}{Do Now: Let $f(x)=x e^x$.}
    \begin{enumerate}
        \item Find $f'(x)$
        \item The graph of $f$ has a minimum point at $A$. Find the exact values of the $x$- and $y$-coordinates of $A$.
        \item Given the point $A(5,12)$. State the position vector $\overrightarrow{OA}$ in unit vector form.
        \item Find the unit vector parallel to $\overrightarrow{OA}$.
    \end{enumerate}
    \end{block}
    Lesson: Adding vectors, the zero vector and equilibrium pp. 420-422\\ \bigskip
    Homework: Exercises 12F (evens) p. 420, 12G (a and c) p. 422-423.
  }

\end{document}
