\documentclass{beamer}
\usepackage{geometry}
\usepackage[english]{babel}
\usepackage[utf8]{inputenc}
\usepackage{amsmath}
\usepackage{amsfonts}
\usepackage{amssymb}
\usepackage{tikz}
\usepackage{graphicx}
\usepackage{venndiagram}

%\usepackage{pgfplots}
%\pgfplotsset{width=10cm,compat=1.9}
%\usepackage{pgfplotstable}

\setlength{\headheight}{26pt}%doesn't seem to fix warning

\usepackage{fancyhdr}
\pagestyle{fancy}
\fancyhf{}

%\rhead{\small{21 May 2018}}
\lhead{\small{BECA / Dr. Huson / 12.1 IB Math - Unit 6 Trig \& Circular Functions}}

%\vspace{1cm}

\renewcommand{\headrulewidth}{0pt}


\title{12.1 IB Math - Unit 9: Probability}
\subtitle{Bronx Early College Academy}
\author{Christopher J. Huson PhD}
\date{18 March 2019}

\begin{document}

\frame{\titlepage}

\section[Outline]{}
\frame{\tableofcontents}

\section{7.1 Venn diagrams, Monday 18 March}
  \frame
  {
    \frametitle{GQ: How do we notate sample spaces with Venn diagrams?}
    \framesubtitle{CCSS: HSS.CP.A.3 Understand conditional probability \hfill  \alert{7.1 Monday 18 March}}

    \begin{block}{Do Now: Draw a Venn diagram of these 110 students:}
      \begin{itemize}
        \item 25 students took physics
        \item 45 students took biology
        \item 48 students took mathematics
        \item 10 students took physics and mathematics
        \item 8 students took biology and mathematics
        \item 6 students took biology and physics
        \item 5 students took all three subjects
      \end{itemize}
      How many took biology, but neither physics nor mathematics?\\
      How many students did not take any of the three subjects?
    \end{block}
    Lesson: Sets, complements, union, intersection, empty set \\ \bigskip
    Homework: Problem set
  }

\section{7.2 Deltamath probability review. Tuesday 19 March}
  \frame
  {
    \frametitle{GQ: How do we notate sample spaces with Venn diagrams?}
    \framesubtitle{CCSS: HSS.CP.A.3 Understand conditional probability \hfill \alert{7.2 Tuesday 19 March}}

    \begin{block}{Do Now Quiz: Trig, calculus practice, with calculator}
    \begin{enumerate}
        \item \emph{Medium} Middling exam problems
        \item \emph{Spicy} Middling and extended exam problems
    \end{enumerate}
    \end{block}

    Lesson: Deltamath probability (trigonometry \& calculus) review\\
    Homework: Complete Deltamath problem set, review quiz answers
  }

  \section{7.3 Expected value, Wednesday 20 March}
    \frame
    {
      \frametitle{GQ: How do we calculate expected value?}
      \framesubtitle{CCSS: HSS.MD.A.3 Develop a probability distribution for a random variable \hfill \alert{7.3 Wednesday 20 March}}

      \begin{block}{Do Now: Algebra practice, with calculator}
      \end{block}

      Lesson: Expected value \\
      Homework: Problem set
    }

\section{7.4 Conditional probability, trees with \& without replacement, Thursday 21 March}
  \frame
  {
    \frametitle{GQ: How do we add the probabilities of multiple events?}
    \framesubtitle{CCSS: HSS.CP.A.3 Understand conditional probability \hfill \alert{7.4 Thursday 21 March}}

    \begin{block}{Do Now Quiz: Trig, calculus practice, with calculator}
    \begin{enumerate}
        \item \emph{Medium} Middling exam problems
        \item \emph{Spicy} Middling and extended exam problems
    \end{enumerate}
    \end{block}

    Lesson: Conditional probability, trees with \& without replacement \\
    Homework: Problem set
  }

\section{7.5 Binomial distribution, Friday 22 March}
  \frame
  {
    \frametitle{GQ: How do we model a series of events?}
    \framesubtitle{CCSS: HSS.MD.A.3 Develop a probability distribution for a random variable \hfill \alert{7.5 Friday 22 March}}

    \begin{block}{Do Now: Make a tree representing three coin flips}
    \begin{enumerate}
        \item What is the probability of each outcome?
        \item If order doesn't matter, how can the results be consolidated into a probability distribution of the total number of heads?
    \end{enumerate}
    \end{block}
    Lesson:  Binomial expansion p. 186-8\\%*[5pt]
    Homework: Problem set
  }

\section{7.6 Binomial distribution, Monday 25 March}
  \frame
  {
    \frametitle{GQ: How do we model a series of events?}
    \framesubtitle{CCSS: HSS.MD.A.3 Develop a probability distribution for a random variable \hfill \alert{7.6 Monday 25 March}}

    \begin{block}{Do Now: Sequences review, Exercise 6L \#1-4 p. 182-3}
    \begin{enumerate}
        \item Use the sequences formulas on the formula sheet
        \item The equation for compound interest (try to remember it first) is $P_n=P_0(1+\frac{i}{c})^{cn}$
    \end{enumerate}
    \end{block}
    Lesson:  Binomial expansion p. 186-8\\
    Assessment: Exercise 6N p. 187\\%*[5pt]
    Homework: Exercise 6O p. 188
  }
\end{document}
