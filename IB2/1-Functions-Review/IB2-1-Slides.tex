\documentclass{beamer}
\usepackage{geometry}
\usepackage[english]{babel}
\usepackage[utf8]{inputenc}
\usepackage{amsmath}
\usepackage{amsfonts}
\usepackage{amssymb}
\usepackage{tikz}
\usepackage{graphicx}
\usepackage{venndiagram}

%\usepackage{pgfplots}
%\pgfplotsset{width=10cm,compat=1.9}
%\usepackage{pgfplotstable}

\setlength{\headheight}{26pt}%doesn't seem to fix warning

\usepackage{fancyhdr}
\pagestyle{fancy}
\fancyhf{}

%\rhead{\small{5 September 2018}}
\lhead{\small{BECA / Dr. Huson / 12.1 IB Math Unit 1}}

%\vspace{1cm}

\renewcommand{\headrulewidth}{0pt}


\title{Mathematics Class Slides}
\subtitle{Bronx Early College Academy}
\author{Chris Huson}
\date{5-21 September 2018}

\begin{document}

\frame{\titlepage}

%\section[Outline]{}
%\frame{\tableofcontents}

  \section{1.1 Drui}
  \frame
  {
    \frametitle{GQ: How do we follow IB mathematics conventions?}
    \framesubtitle{CCSS: MP.6 Attend to precision  \hspace{\stretch{1}} \alert{1.1}}

    \begin{block}{Do Now Handout: precision, notation}
    \begin{enumerate}
        \item Welcome back to school!
        \item Take out notebooks, formula sheets, \& calculators\\*
    \end{enumerate}
    \end{block}
    Lesson: Identifying graphical features, average slope \\%*[5pt]
    Homework: Problem set: sig figs
  }
%Prepare copies of formula sheets

\section{1.2 Drui}
  \frame
  {
    \frametitle{GQ: What is the average rate of change of a function?}
    \framesubtitle{CCSS: HSF.IF.B.6 Calculate and interpret the rate of change of a function  \hspace{\stretch{1}} \alert{1.2}}

    \begin{block}{Do Now: Substitution notation, Point-slope}
    \begin{enumerate}
        \item Show that the point $(3, 4)$ is on the line $y=2x-2$
        \item $f(x)=x^2+7$ and $f(k)=11$. Solve for $k$.
        \item Find the slope of the line through the points $(-3,0)$ and $(5, -5)$ \\
        Find the slope of a line perpendicular to the given line.
    \end{enumerate}
    \end{block}
    Lesson: Applying the concept of slope in non-linear problems.
    Calculator deposits \$20
    \\%*[5pt]
    Homework: Average rate of change problems
  }

\section{1.3 Drui}
  \frame
  {
    \frametitle{GQ: What is asymptotic behavior?}
    \framesubtitle{CCSS: HSF.IF.B.4 Interpret key features of functions and their graphs  \hspace{\stretch{1}} \alert{1.3}}

    \begin{block}{Do Now Quiz: Precision}
    \begin{enumerate}
        \item Handout
    \end{enumerate}
    \end{block}
    Lesson: Rational functions, asymptotes, applications p. 147-156\\ [calculator graphing example]
    \\[10pt]
    Homework: Textbook Review exercises 1-5 page 155-6\\[10pt]
    Peer edit of exploration papers.
  }

\section{1.4 Drui}
  \frame
  {
    \frametitle{GQ: How do we graph quadratics?}
    \framesubtitle{CCSS: HSF.IF.B.4 Interpret key features of functions and their graphs  \hspace{\stretch{1}} \alert{1.4}}

    \begin{block}{Do Now: Factoring}
    \begin{enumerate}
        \item Find the intercepts, axis of symmetry, and minimum point of the graph of the function $f(x)=(x-1)(x-5)$?
        \item Factor the function $g(x)=x^2-x-12$ to determine the features of its graph.
        \item Convert the function $h(x)=x^2+4x+3$ to the vertex form, $h(x)=a(x-h)^2+k$. Write down its vertex.
    \end{enumerate}
    \end{block}
    Lesson: $x-$ and $y-$intercepts; completing the square, vertex, axis of symmetry; discriminant; odd, even functions
    \\%*[5pt]
    Homework: Cubic functions and their graphical characteristics
  }

  \frame
  {
    \frametitle{How do we graph quadratics?}
    \framesubtitle{CCSS: HSF.IF.B.4 Interpret key features of functions and their graphs  \hspace{\stretch{1}} \alert{1.4}}

    \begin{block}{Consider the function $f(x)=-x^2+2x+3$}
    \begin{enumerate}
        \item Factor $f$ and state its zeros.
        \item Restate $f$ in vertex form. Write down the vertex as an ordered pair.
        \item Over what intervals is the function increasing, decreasing, and neither?
        \item If $f(x)$ represents the height of a diver over the domain $0 \leq x \leq 3$, interpret $f(0)$, the vertex, and $f(3)$
        \item What does the "slope" of the curve represent?
    \end{enumerate}
    \end{block}
  }

  \frame
  {
    \frametitle{GQ: How do we graph quadratics?}
    \framesubtitle{CCSS: HSF.IF.B.4 Interpret key features of functions and their graphs  \hspace{\stretch{1}} \alert{1.4}}

    \begin{block}{Symmetry of functions, graphically and algebraically}
    \begin{enumerate}
        \item Definition: a function is \emph{even} if it can be reflected onto itself across the $y-$axix. Equivalently, iff $f(x)=f(-x)$
        \bigskip
        \item Definition: a function is \emph{odd} if it can be reflected onto itself across the origin. Equivalently, iff $f(x)=-f(-x)$
    \end{enumerate}
    \end{block}
    %graph examples
  }

\section{1.5 Drui}
  \frame
  {
    \frametitle{GQ: How do we calculate rates in context?}
    \framesubtitle{CCSS: HSF.IF.B.6 Calculate and interpret the rate of change of a function  \hspace{\stretch{1}} \alert{1.5}}

    \begin{block}{Do Now: Sketch the function $f(x)=x^3-9x$}
      \begin{enumerate}
      \item Either factor it by hand or use a graphing calculator .
      \item "Sketch" is an IB "command term" meaning roughly show the key relationships.
      \item Label the intersections, maximum, and minimum.
      \item Add an axis caption of increasing ("++++") and decreasing ("- - - -") intervals
      \end{enumerate}
   \end{block}
    Lesson: Quadratics applications, graphing pp. 53-56\\ \bigskip
    Homework: Review exercise \#1-5 pp. 57-58
  }

\section{1.6 Drui - exponents, Friday Sept 14}
  \frame
  {
    \frametitle{GQ: How do we simplify exponents?}
    \framesubtitle{CCSS: HSF.LB.B.5 Interpret the parameters in an exponential function \hspace{\stretch{1}} \alert{1.6}}

    \begin{block}{Do Now: Exponent warmup problems. Simplify. Try expanding the exponential expressions as multiplication, as an intermediate step.}
      \begin{enumerate}
      \item $2^3 \cdot 2^2$
      \item $3^6 \div 3^2$
      \item $(5^3)^2$
      \item $\displaystyle \frac{x^2 \cdot x^4}{x^3}$
      \end{enumerate}
   \end{block}
    Lesson: Exponent expressions pp. 100-107\\ \bigskip
    Homework: Select exercises from  4A, 4B, 4C pp. 104-107
  }

\section{1.7 Drui - exponents, Monday Sept 17}
  \frame
  {
    \frametitle{GQ: How do we work with logarthms?}
    \framesubtitle{CCSS: HSF.LB.B.5 Interpret the parameters in an exponential function \hspace{\stretch{1}} \alert{1.7}}

    \begin{block}{Do Now: Exponent warmup problems. Simplify.}
      \begin{enumerate}
      \item $x^3 \cdot x^2$
      \item $(ab)^6 \div a^2 b$
      \item $(2m^3)^2$
      \item $\displaystyle x^{-\frac{1}{2}}$
      \end{enumerate}
   \end{block}
    Lesson: Properties of logarithms pp. 115-117\\ \bigskip
    Homework: Select exercises from 4E p. 109: 1a, 1b, 2a, 2b, 3a, 3b; 4I pp. 117-118
  }

  \end{document}
