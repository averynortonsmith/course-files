\documentclass{beamer}
\usepackage{geometry}
\usepackage[english]{babel}
\usepackage[utf8]{inputenc}
\usepackage{amsmath}
\usepackage{amsfonts}
\usepackage{amssymb}
\usepackage{tikz}
\usepackage{graphicx}
\usepackage{venndiagram}

%\usepackage{pgfplots}
%\pgfplotsset{width=10cm,compat=1.9}
%\usepackage{pgfplotstable}

\setlength{\headheight}{26pt}%doesn't seem to fix warning

\usepackage{fancyhdr}
\pagestyle{fancy}
\fancyhf{}

\lhead{\small{BECA / Dr. Huson / 12.1 IB Math SL - Unit 4}}

\renewcommand{\headrulewidth}{0pt}

\title{Mathematics Class Slides}
\subtitle{Bronx Early College Academy}
\author{Chris Huson}
\date{5 December 2018}

\begin{document}

\frame{\titlepage}

%\section[Outline]{}
%\frame{\tableofcontents}

\section{4b.1 Drui - Graphing w 1st \& 2nd derivative tests, Wednesday Dec 5}
  \frame
  {\frametitle{GQ: How does a function's graph relate to its derivatives?}
    \framesubtitle{CCSS: HSF.IF.B.4 Interpret features of functions and their graphs \  \alert{4b.1 Wednesday Dec 5}}

    \begin{block}{Do Now: Differential calculus}
    \begin{enumerate}
        \item Take the 1st \& 2nd derivatives of $f(x)=x^3-6x^2+8x$.
        \item Sketch the function.\\*
        Challenge: Identify key features, graphically \& algebraically.
    \end{enumerate}
    \end{block}
    Lesson: Function graphs, extrema, the 1st \& 2nd derivative tests p. 233- 240\\ \bigskip
    Homework: Textbook exercises 7T p. 239
  }

\section{4b.2 Drui - Graphing w 1st \& 2nd derivative tests, Thursday Dec 6}
  \frame
  {\frametitle{GQ: How does a function's graph relate to its derivatives?}
    \framesubtitle{CCSS: HSF.IF.B.4 Interpret features of functions and their graphs \quad \alert{4b.2 Thursday Dec 6}}

    \begin{block}{Do Now: Vector review, handout}
    \end{block}
    Lesson: 7.7 Function graphs, extrema at endpoints p. 240-244\\
    Graphing exercises 7U p. 240\\ \bigskip
    Homework: Textbook exercises 7V \& 7W p. 242-4
  }

\section{4b.3 Drui - Optimization, Friday Dec 7}
  \frame
  {\frametitle{GQ: How do use calculus to optimize a situation?}
    \framesubtitle{CCSS: HSF.IF.B.4 Interpret features of functions and their graphs \quad \alert{4b.3  Friday Dec 7}}

    \begin{block}{Do Now: Vector \& derivatives review, handout}
    \end{block}
    Lesson: 7X Optimization problems p. 244-247\\ \bigskip
    Assessment: Pop quiz covering Do Now problems\\ \bigskip
    Homework: IB problem set
  }

\section{4b.4 Drui - Applications, Monday Dec 10}
  \frame
  {\frametitle{GQ: How does a function's graph relate to its derivatives?}
    \framesubtitle{CCSS: HSF.IF.B.4 Interpret features of functions and their graphs \quad \alert{4b.4 Monday Dec 10}}

    \begin{block}{Do Now: Graphing polynomial functions and their derivatives}
      \begin{enumerate}
        \item A cubic function $f(x)$ with positive leading coeffient has a local maximum at $x=-3$, local minimum at $x=5$ and constant term of 12. Sketch $f(x)$, marking the given attributes.
        \item Sketch $f'(x)$ and $f''(x)$ on the same axes as $f(x)$.
        \item On an $x$-axis below the sketch, sketch a $+/-$ number line to show where the function is increasing and decreasing.
      \end{enumerate}
    \end{block}
    Lesson: 7Y Optimization problems p. 244-247\\
    Application problems\\
    \alert{Test Thursday} \\
    Homework: Pretest packet, due Wednesday
  }

\section{4b.4 Drui - Applications, Friday Dec 14}
  \frame
  {\frametitle{GQ: How does a function's graph relate to its derivatives?}
    \framesubtitle{CCSS: HSF.IF.B.4 Interpret features of functions and their graphs \quad \alert{4b.4 Monday Dec 10}}

    \begin{block}{Do Now:  Vector \& derivatives review, handout}
    \end{block}
    Lesson: 7Y Optimization problems p. 244-247\\
    Application problems\\ \bigskip
    Homework: Textbook exercises 7Y p. 248
  }

\end{document}
