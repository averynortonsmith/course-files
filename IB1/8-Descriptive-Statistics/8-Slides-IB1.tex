\documentclass{beamer}
\usepackage{geometry}
\usepackage[english]{babel}
\usepackage[utf8]{inputenc}
\usepackage{amsmath}
\usepackage{amsfonts}
\usepackage{amssymb}
\usepackage{tikz}
\usepackage{graphicx}
\usepackage{venndiagram}

%\usepackage{pgfplots}
%\pgfplotsset{width=10cm,compat=1.9}
%\usepackage{pgfplotstable}

\setlength{\headheight}{26pt}%doesn't seem to fix warning

\usepackage{fancyhdr}
\pagestyle{fancy}
\renewcommand{\headrulewidth}{0pt}

\lhead{\small{BECA / Dr. Huson / 11.1 IB Math - Unit 8 Descriptive Statistics}}

\title{11.1 IB Math - Unit 8 Descriptive Statistics}
  \subtitle{Bronx Early College Academy}
  \author{Christopher J. Huson PhD}
  \date{6 May 2019}

\begin{document}
  \frame{\titlepage}
  \section[Outline]{}
  \frame{\tableofcontents}

\section{8.1 Introduction and definitions Monday 6 May}
  \frame
  {
    \frametitle{GQ: How do we determine the features of a population?}
    \framesubtitle{HSS.ID.A.1-4 Summarize, represent, and interpret data on a single measurement variable \hfill \alert{8.1 Monday 6 May}}

    \begin{block}{Do Now: Skills Check p. 254}
    \end{block}
    Lesson: Qualitative \& quantitative data, graphing, definitions p.255-9\\[1cm]
    Homework: Exercises 8B p. 259
  }

\section{8.2 Deltamath summary statistics, Tuesday 6 May}
  \frame
  {
    \frametitle{GQ: How do we determine the features of a population?}
    \framesubtitle{CCSS: HSS.ID.A.1-4 Summarize, represent, and interpret data on a single measurement variable \hfill \alert{8.2 Tuesday 7 May}}

    \begin{block}{Deltamath probability practice}
    \end{block}
    Homework: Complete Deltamath exercises
  }

\section{8.3 Central tendency Wednesday 8 May}
  \frame
  {
    \frametitle{GQ: How do we determine the ``center" of a population?}
    \framesubtitle{HSS.ID.A.1-4 Summarize, represent, and interpret data on a single measurement variable \hfill \alert{8.3 Wednesday 8 May}}

    \begin{block}{Do Now: Sequences review}
      \begin{enumerate}
        \item An arithmetic sequence begins $4, k, 10, ...$. Find $k$.
        \item Find the value of the 8th term of the sequence.
        \item The sum of the first $n$ terms in the sequence is 589. Find $n$.
    \end{enumerate}
    \end{block}
    Lesson: Measures of central tendency: mean, median, \& mode p. 260-7\\
    Using class interval midpoints for frequency table calculations.\\[0.5cm]
    Homework: Exercises 8C, 8D, 8E. Select an appropriate number of problems.
  }

\section{8.4 Central tendency Thursday 9 May}
  \frame
  {
    \frametitle{GQ: How do we determine the ``spread" of a population?}
    \framesubtitle{HSS.ID.A.1-4 Summarize, represent, and interpret data on a single measurement variable \hfill \alert{8.4 Thursday 9 May}}

    \begin{block}{Do Now: Enter the frequency table data shown in a calculator. Answer the questions both with the calculator and by hand.}
        \begin{tabular}{|l|r|r|r|r|r|}
          \hline
          Value & 0 & 1 & 2 & 3 & 4\\
          \hline
          Freq. & 2 & 6 & 4 & 2 & 1\\
          \hline
        \end{tabular}
        \begin{enumerate}
          \item How many data are there ($n=?$)? List them.
          \item Write down the mode. Find the median and mean.
          \item Sketch a histogram to represent the data.
      \end{enumerate}
    \end{block}

    Lesson: 8.4 Measures of dispersion: max, min, range, quartiles, IQR, \& 5-figure summary p. 267-271\\
    Using class interval midpoints for frequency table calculations.\\
    Cumulative distributions p. 271-2\\[0.25cm]
    Homework: Exercises 8F, 270-1 (8G).
  }

\section{8.5 Standard deviation Monday 13 May}
  \frame
  {
    \frametitle{GQ: How do we quantify the dispersion of a population?}
    \framesubtitle{HSS.ID.A.1-4 Summarize, represent, and interpret data on a single measurement variable \hfill \alert{8.5 Monday 13 May}}

    \begin{block}{Do Now: The frequency table represents the scores of an IB class out of a 90-point exam.}
        \begin{tabular}{|l|c|c|c|c|}
          \hline
          Score & $10 \leq x<30$ &
            $30 \leq x<50$ & $50 \leq x<70$ & $70 \leq x<90$\\
          \hline
          Freq. & 4 & 6 & 3 & 2\\
          \hline
        \end{tabular}
        \begin{enumerate}
          \item How many students are there?
          \item Write down the modal class.
          \item Estimate the median, quartiles, and the mean.
          \item Sketch a histogram to represent the data.
      \end{enumerate}
    \end{block}
    Cumulative distributions, \#6 p. 275\\
    Lesson: 8.6 Standard deviation p. 276-281\\[0.5cm]
    Homework: Exercises 8H, 279-280.
  }

\section{8.6 Standard deviation Tuesday 14 May}
  \frame
  {
    \frametitle{GQ: How do we ``rangle" a dataset?}
    \framesubtitle{HSS.ID.A.1-4 Summarize, represent, and interpret data on a single measurement variable \hfill \alert{8.6 Tuesday 14 May}}

    \begin{block}{Do Now: If you had access to the passenger roster of the Titanic, what interesting questions would you explore?}
        \begin{enumerate}
          \item Write down a question regarding the types of passengers on the Titanic's maiden voyage.
          \item Write a question regarding who survived versus died.
          \item Suggest calculations that answer the questions.
          \item What types of graphs might you make?
      \end{enumerate}
    \end{block}
    Cumulative distributions, \#6 p. 275\\
    Lesson: Working with datasets using modern technology\\[0.5cm]
    Homework: Review exercises 281-284.
  }

\section{8.7 Cumulative distributions Wednesday 15 May}
  \frame
  {
    \frametitle{GQ: How do we understand a dataset as a cumulative distribution?}
    \framesubtitle{HSS.ID.A.1-4 Summarize, represent, and interpret data on a single measurement variable \hfill \alert{8.7 Wednesday 15 May}}

    \begin{block}{Do Now Quiz}
        \begin{enumerate}
          \item Sequences review
          \item 5-figure summary
          \item Cumulative distributions
      \end{enumerate}
    \end{block}
    Lesson: Cumulative distributions, \#6 p. 275\\
    Effect on statistical measures of scaling data values\\[0.5cm]
    Homework: Pretest problem set
  }

\section{8.8 Cumulative distributions Thursday 16 May}
  \frame
  {
    \frametitle{GQ: How do we use a cumulative distribution graph?}
    \framesubtitle{HSS.ID.A.1-4 Summarize, represent, and interpret data on a single measurement variable \hfill \alert{8.8 Thursday 16 May}}

    \begin{block}{Do Now: Handout}
        \begin{enumerate}
          \item Mean \& standard deviation, scaling
          \item Interpreting a cumulative frequency graph
          \item Spicy sequence problems
      \end{enumerate}
    \end{block}
    Review pretest problems\\
    Lesson: Interpreting cumulative distribution graphs, \#6 p. 275\\
    Homework: Pretest problem set
  }

\section{9.1 Bivariate analysis Monday 20 May}
  \frame
  {
    \frametitle{GQ: How do we compare two variables?}
    \framesubtitle{HSS.ID.A.1-4 Summarize, represent, and interpret data of two measurement variables \hfill \alert{9.1 Monday 20 May}}

    \begin{block}{Do Now: Handout}
        \begin{enumerate}
          \item Plotting two variables
          \item Calculator use for two sets of data
          \item Quiz corrections
      \end{enumerate}
    \end{block}
    Review homework problems\\
    Lesson: Interpreting cumulative distribution graphs, \#6 p. 333-338\\[0.5cm]
    Homework: Textbook exercise 10A pp. 337-9 (use a calculator for \#4,5 instead of graphing by hand)
  }

\section{9.2 Line of best fit, using Desmos Tuesday 21 May}
  \frame
  {
    \frametitle{GQ: How do we compare two variables?}
    \framesubtitle{HSS.ID.A.1-4 Summarize, represent, and interpret data of two measurement variables \hfill \alert{9.2 Tuesday 21 May}}

    \begin{block}{Do Now Handout: Interpreting scatter plot data}
        \begin{enumerate}
          \item Plotting two variables
          \item Calculating the line of best fit's slope and $y$-intercept
          \item Interpreting the parameters
      \end{enumerate}
    \end{block}
    Introduction to Canvas: Word formatting practice\\
    Lesson: Using Desmos, 10B\&C p. 341, 343\\[0.5cm]
    Homework: Textbook exercise 10D pp. 344 (\alert{test Thursday})
  }

\section{9.3 Review logs, exponential functions, sequences for exam Wednesday 22 May}
  \frame
  {
    \frametitle{GQ: How do we compare two variables?}
    \framesubtitle{HSS.ID.A.1-4 Summarize, represent, and interpret data of two measurement variables \hfill \alert{9.3 Wednesday 22 May}}

    \begin{block}{Do Now Handout: Interpreting cumulative distribution plots}
        \begin{enumerate}
          \item Interpreting the parameters
      \end{enumerate}
    \end{block}
    Interpreting cumulative distribution graphs, \#6 p. 333-338\\
    Lesson: Review logs, exponential functions, sequences for exam\\[0.5cm]
    Homework: Study for (\alert{test tomorrow})
  }

\section{9.4 Statistics exam (sequences \& logs review) Thursday 23 May}
  \frame
  {
    \frametitle{GQ: How do we compare two variables?}
    \framesubtitle{HSS.ID.A.1-4 Summarize, represent, and interpret data of two measurement variables \hfill \alert{9.4 Thursday 23 May}}

    Assessment: Descriptive statistics exam\\[0.5cm]
    Homework: Weekend packet
  }

  \section{10.1 Scale \& applications of dilation Tuesday 28 May}
    \frame
    {
      \frametitle{GQ: How do we use scale factors?}
      \framesubtitle{CCSS: HSG.CO.D.12 Congruence, geometric constructions \hfill \alert{10.1 Tuesday 28 May}}

      \begin{block}{Do Now: Handout}
        \begin{enumerate}
          \item Using scale factors
          \item Real world situations
        \end{enumerate}
      \end{block}
      Guest teacher, Mr. Segal. Applications of scale factors in finance.\\[0.25cm]
      Homework: Problem set, test corrections due Thursday
    }

\end{document}
