\documentclass{beamer}
\usepackage{geometry}
\usepackage[english]{babel}
\usepackage[utf8]{inputenc}
\usepackage{amsmath}
\usepackage{amsfonts}
\usepackage{amssymb}
\usepackage{tikz}
\usepackage{graphicx}
\usepackage{venndiagram}

%\usepackage{pgfplots}
%\pgfplotsset{width=10cm,compat=1.9}
%\usepackage{pgfplotstable}

\setlength{\headheight}{26pt}%doesn't seem to fix warning

\usepackage{fancyhdr}
\pagestyle{fancy}
\renewcommand{\headrulewidth}{0pt}

\lhead{\small{BECA / Dr. Huson / 11.1 IB Math Unit 4}}



\title{Mathematics Class Slides}
\subtitle{Bronx Early College Academy}
\author{Chris Huson}
\date{2 January - 20 January 2019}

\begin{document}

\frame{\titlepage}

\section[Outline]{}
\frame{\tableofcontents}

  \section{5.2 Drui: Exponent rules, Monday Jan 7}
    \frame
    {
      \frametitle{GQ: How do we manipulate exponential expressions?}
      \framesubtitle{CCSS: HSF.IF.C.7 Analyze functions  \hspace{\stretch{1}}  \alert{5.2  Monday Jan 7}}

      \begin{block}{Do Now: Exponents handout (Regents formula sheet)}
      \end{block}
      Exponent operations, imaginary numbers, exponential function applictions \\ \bigskip
      Homework: Regents questions review
    }

  \section{5.3 Drui: Laptop, Deltamath, Desmos /Word. Tuesday Jan 8}
    \frame
    {
      \frametitle{How do we communicate mathematical results?}
      \framesubtitle{CCSS: MP.4 Model with mathematics \hspace{\stretch{1}} \alert{5.3 Tuesday Jan 8}}

      \begin{block}{Technical skills needed to communicate mathematics}
      \begin{enumerate}
          \item Word processing: Microsoft Word and equation editor
          \item Computer calculators: Desmos; domain restriction, labeling
          \item Cloud storage: Dropbox
          \item Technical writing standards: MLA format (Purdue OWL)
          \item Writing style: declarative
          \item Assessment criteria: IB exploration criterion \emph{B: Mathematics Presentation}
      \end{enumerate}
      \end{block}
      Deltamath exponential practice. Homework: complete Deltamath \\
      Makeup: Rewrite Quadratics paper, using model as guide
      }

  \section{5.4 Drui: Regents exponent \& exponential function problems, Wednesday Jan 9}
    \frame
    {
      \frametitle{GQ: How do we manipulate exponential expressions?}
      \framesubtitle{CCSS: HSF.IF.C.7 Analyze functions  \hspace{\stretch{1}}  \alert{5.4 Wednesday Jan 9}}

      \begin{block}{Do Now: Exponents handout (Regents problems)}
      \end{block}
      Exponent operations, imaginary numbers, exponential function applictions \\ \bigskip
      Homework: Regents questions review
    }

  \section{5.6 Drui: Regents exponent \& exponential function problems, Monday Jan 14}
    \frame
    {
      \frametitle{GQ: How do we manipulate exponential expressions?}
      \framesubtitle{CCSS: HSF.IF.C.7 Analyze functions  \hspace{\stretch{1}}  \alert{5.6 Monday Jan 14}}

      \begin{block}{Do Now: Exponents handout (Regents problems)}
      \end{block}
      Exponent operations, imaginary numbers, exponential function applictions \\ \bigskip
      Homework: Test corrections
    }

  \section{5.7 Drui: Regents exponent \& exponential function problems, Tuesday Jan 15}
    \frame
    {
      \frametitle{GQ: How do we manipulate exponential expressions?}
      \framesubtitle{CCSS: HSF.IF.C.7 Analyze functions  \hspace{\stretch{1}}  \alert{5.7 Tuesday Jan 15}}

      \begin{block}{Do Now: Regents problems}
        \begin{enumerate}
          \item Express $\sqrt[5]{x^3}$ as a single term with a rational exponent.
          \item Find $h$ and $k$: $3x^3+(2x-3)^2=hx^3+4x^2+kx+9$
          \item Explain how $\displaystyle 4^{-\frac{3}{2}}$ can be written equivalently as $\frac{1}{8}$
        \end{enumerate}
      \end{block}
      Review test corrections\\
      Polynomial functions, graphs, factoring, remainder theorem \\ \bigskip
      Homework: Complete classwork problem set
    }

\end{document}
