\documentclass{beamer}
\usepackage{geometry}
\usepackage[english]{babel}
\usepackage[utf8]{inputenc}
\usepackage{amsmath}
\usepackage{amsfonts}
\usepackage{amssymb}
\usepackage{tikz}
\usepackage{graphicx}
\usepackage{venndiagram}

%\usepackage{pgfplots}
%\pgfplotsset{width=10cm,compat=1.9}
%\usepackage{pgfplotstable}

\setlength{\headheight}{26pt}%doesn't seem to fix warning

\usepackage{fancyhdr}
\pagestyle{fancy}
\renewcommand{\headrulewidth}{0pt}

\lhead{\small{BECA / Dr. Huson / 11.1 IB Math - Unit 6 Probability}}



\title{Mathematics Class Slides}
\subtitle{Bronx Early College Academy}
\author{Chris Huson}
\date{30 January - 9 February 2019}

\begin{document}

\frame{\titlepage}

\section[Outline]{}
\frame{\tableofcontents}

\section{6.1 Probability definitions, Wednesday 30 January}
  \frame
  {
    \frametitle{GQ: How do we talk about probability?}
    \framesubtitle{CCSS: HSF.IF.C.7 Analyze functions \hfill \alert{6.1 Wednesday 30 January}}

    \begin{block}{Do Now: Skills check p. 62 \#1-2}
    \end{block}
    Lesson: History of the study of games of chance, sample space, frequency \\ \bigskip
    Homework: Exercises 3A p. 67-68
  }


\section{6.2 Venn diagrams, Thursday 31 January}
  \frame
  {
    \frametitle{GQ: How do we notate sample spaces with Venn diagrams?}
    \framesubtitle{CCSS: HSF.IF.C.7 Analyze functions \hfill  \alert{6.2 Thursday 31 January}}

    \begin{block}{Do Now: Probability handout}
    \end{block}
    Lesson: Sets, complements, union, intersection, empty set \\ \bigskip
    Homework: Exercises 3B p. 71-72
  }

\section{6.3 Addition rule of probabilities, Monday 4 February}
  \frame
  {
    \frametitle{GQ: How do we add the probabilities of multiple events?}
    \framesubtitle{CCSS: HSF.IF.C.7 Analyze functions \hfill \alert{6.3 Monday 4 February}}

    \begin{block}{Do Now: Draw a Venn diagram of these 110 students:}
      \begin{itemize}
        \item 25 students took physics
        \item 45 students took biology
        \item 48 students took mathematics
        \item 10 students took physics and mathematics
        \item 8 students took biology and mathematics
        \item 6 students took biology and physics
        \item 5 students took all three subjects
      \end{itemize}
      How many took biology, but neither physics nor mathematics?\\
      How many students did not take any of the three subjects?
    \end{block}
    Lesson: The addition rule, probability \\
    Homework: Exercises 3C p. 74-75 \alert{Unit test Thursday}
  }

  \section{6.4 Probability definitions, Wednesday 6 February}
    \frame
    {
      \frametitle{GQ: How do we calculate probability?}
      \framesubtitle{CCSS: HSF.IF.C.7 Analyze functions \hfill \alert{6.4 Wednesday 6 February}}

      \begin{block}{Do Now: Pretest review packet}
      \end{block}
      Lesson: Review for unit test \\ \bigskip
      Homework: study for \alert{test tomorrow}. Arrive at 8:00 sharp!
    }


\section{6.5 Probability definitions, Thursday 7 February}
  \frame
  {
    \frametitle{GQ: How do we calculate probability?}
    \framesubtitle{CCSS: HSF.IF.C.7 Analyze functions \hfill \alert{6.5 Thursday 7 February}}

    %\begin{block}{Do Now: Pretest review packet}
    %\end{block}
    Assessment: Probability unit test \\ \bigskip
    Homework: Handout practice problems
  }

\section{6.6 Sample space diagrams, Monday 11 February}
  \frame
  {
    \frametitle{GQ: How do we add the probabilities of multiple events?}
    \framesubtitle{CCSS: HSF.IF.C.7 Analyze functions \hfill \alert{6.6 Monday 11 February}}

    \begin{block}{Do Now: A six-sided, fair die is rolled 100 times, with $x$ representing each value rolled. Draw a Venn diagram to represent the 100 events.}
      \begin{itemize}
        \item $\{x$ is an even number \} was rolled 57 times
        \item $\{x: x <4\}$ occurred 44 times
        \item $\{x: x =2\}$ occurred 15 times
      \end{itemize}
      How many times was $x=5$ rolled?
    \end{block}
    Test question review\\
    Lesson: Mutually exclusive sets, sample space diagrams\\[0.5cm]
    Homework: Exercises 3D p. 76-77, 3E \#1-2 p. 79
  }


\section{6.7 Sample space diagrams, Tuesday 12 February}
  \frame
  {
    \frametitle{GQ: How do we add the probabilities of multiple events?}
    \framesubtitle{CCSS: HSF.IF.C.7 Analyze functions \hfill \alert{6.7 Tuesday 12 February}}

    \begin{block}{Deltamath probability practice}
    \end{block}
    Homework: Complete Deltamath exercises
  }

\section{6.8 Independence, Wednesday 13 February}
  \frame
  {
    \frametitle{GQ: How do we multiply the probabilities of multiple events?}
    \framesubtitle{CCSS: HSF.IF.C.7 Analyze functions \hfill \alert{6.8 Wednesday 13 February}}

    \begin{block}{Do Now: 3E p. 80}
      \begin{itemize}
        \item medium: exercise \#3
        \item spicy: exercise \#5
      \end{itemize}
    \end{block}
    Test question review\\
    Lesson: Independence and multiplying probabilities\\[0.5cm]
    Homework: Exercises 3F p. 82-84
  }


\section{6.9 Conditional probability, Thursday 14 February}
  \frame
  {
    \frametitle{GQ: How do we calculate probability given another condition?}
    \framesubtitle{CCSS: HSF.IF.C.7 Analyze functions \hfill \alert{6.9 Thursday 14 February}}

    \begin{block}{Do Now: Read the Monty Hall problem, p. 88. Be prepared to discuss}
    \end{block}
    Test question review\\
    Lesson: Conditional probability\\[0.5cm]
    Homework: Exercises 3G p. 86-888
  }




\end{document}
