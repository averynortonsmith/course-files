\documentclass{beamer}
\usepackage{geometry}
\usepackage[english]{babel}
\usepackage[utf8]{inputenc}
\usepackage{amsmath}
\usepackage{amsfonts}
\usepackage{amssymb}
\usepackage{tikz}
\usepackage{graphicx}
\usepackage{venndiagram}

%\usepackage{pgfplots}
%\pgfplotsset{width=10cm,compat=1.9}
%\usepackage{pgfplotstable}

\setlength{\headheight}{26pt}%doesn't seem to fix warning

\usepackage{fancyhdr}
\pagestyle{fancy}
\fancyhf{}

%\rhead{\small{5 September 2018}}
\lhead{\small{BECA / Dr. Huson / 11.1 IB Math Unit 3}}

%\vspace{1cm}

\renewcommand{\headrulewidth}{0pt}


\title{Mathematics Class Slides}
\subtitle{Bronx Early College Academy}
\author{Chris Huson}
\date{17 October - 1 November 2018}

\begin{document}

\frame{\titlepage}

\section[Outline]{}
\frame{\tableofcontents}

\section{3.1 Drui: Quadratic equations, Wednesday Oct 17}
  \frame
  {
    \frametitle{GQ: How do we solve quadratic equations?}
    \framesubtitle{CCSS: HSF.IF.C.7 Analyze functions  \hspace{\stretch{1}}  \alert{3.1}}

    \begin{block}{Do Now: Skills check \#1, 2a-c, p. 32}
    \end{block}
    %Review formula sheets\\
    Lesson: Quadratics review p 33-35, Exercises 2A, p. 35 \\
    Homework: Exercises 2B, p. 35
  }

\section{3.2 Drui: Completing the square, Thursday Oct 18}
  \frame
  {
    \frametitle{GQ: How do we solve quadratic equations?}
    \framesubtitle{CCSS: HSF.IF.C.7 Analyze functions  \hspace{\stretch{1}}  \alert{3.2}}

    \begin{block}{Do Now: Investigation \#1, 3, 5 p. 36}
    \end{block}
    Lesson: Completing the square p 36-40, Exercises 2C p. 37 \\
    Homework: Exercises 2D (all) p. 38, 2E (odds) p. 40, 2F pick two.
  }

\section{3.3 Drui: The quadratic formula, Monday Oct 22}
  \frame
  {
    \frametitle{GQ: How do we solve quadratic equations?}
    \framesubtitle{CCSS: HSF.IF.C.7 Analyze functions  \hspace{\stretch{1}}  \alert{3.3}}

    \begin{block}{Do Now: }
      \begin{enumerate}
          \item Factor the expression $x^2-25$
          \item Write down the domain and range of $y=(x-3)^2-4$.
          \item Find the asymptotes of $\displaystyle f(x)=\frac{1}{x^2-4}$.
          \item Pick one problem you have not done from 2F pp. 40-1
      \end{enumerate}
    \end{block}
    %Review formula sheets\\
    Lesson: The quadratic formula and the discriminant pp. 38-42 \\
    The powers of $i$, the solution to $x^2=-1$\\ \bigskip
    Homework: Exercises 2E (evens?) p. 40, 2G (a and c) p. 42-3
  }

\section{3.4 Drui: Laptop, Deltamath, Desmos /Word. Tuesday Oct 23}
  \frame
  {
    \frametitle{How do we communicate mathematical results?}
    \framesubtitle{CCSS: MP.4 Model with mathematics \hspace{\stretch{1}} \alert{3.4}}

    \begin{block}{Technical skills needed to communicate mathematics}
    \begin{enumerate}
        \item Word processing: Microsoft Word and equation editor
        \item Computer calculators: Desmos; domain restriction, labeling
        \item Cloud storage: Dropbox
        \item Technical writing standards: MLA format (Purdue OWL)
        \item Writing style: declarative
        \item Assessment criteria: IB exploration criterion \emph{B: Mathematics Presentation}
    \end{enumerate}
    \end{block}
    Lesson: Shared folder structure, graph copy/paste, MLA template\\ \bigskip
    Homework: Deltamath followup. Open textbook online
  }

\section{3.5 Drui: The quadratic formula, Wednesday Oct 24}
  \frame
  {
    \frametitle{GQ: How do we solve quadratic equations?}
    \framesubtitle{CCSS: HSF.IF.C.7 Analyze functions  \hspace{\stretch{1}}  \alert{3.5}}

    \begin{block}{Do Now: Simplifying radicals}
      \begin{enumerate}
          \item Write down a list of the first eight powers of $i$.
          \item Factor 18 as a perfect square times 2
          \item Simplify $\sqrt{-18}$ by separating it into three components: an integer, an irrational root, and $i$
          \item Simplify $\sqrt{-20}$, $\sqrt{-12}$, $\sqrt{-50}$
      \end{enumerate}
    \end{block}
    %Review formula sheets\\
    Lesson: Using the discriminant pp. 38-42 \\
    Features of parabolas pp. 43-46\\ \bigskip
    Homework: Exercises 2G (b and d) p. 42-3, 2H p. 46.
  }

\section{3.6 Drui: Equations from graphs, Thursday Oct 25}
  \frame
  {
    \frametitle{GQ: How do we derive a quadratic's equation from a graph?}
    \framesubtitle{CCSS: HSF.IF.C.7 Analyze functions  \hspace{\stretch{1}}  \alert{3.6}}

    \begin{block}{Do Now: Given the equation $f(x)=x^2-6x+5$}
      \begin{enumerate}
          \item Write the function in factored form.
          \item Complete the square and write the function in vertex form.
          \item Sketch the function, marking the intercepts, vertex, and axis of symmetry (labeled as an equation).
          \item Use a graphing calculator to check your sketch.
      \end{enumerate}
    \end{block}
    %Review formula sheets\\
    Lesson: Parabola features, deriving a function's equation pp. 49-52 \\
    Examples 14, 15, \& 16\\ \bigskip
    Homework: Exercises 2I (a and c) p. 48, 2J p. 52
  }

\section{3.7 Drui: Applications of quadratics, Monday Oct 29}
  \frame
  {
    \frametitle{GQ: How do we solve problems with quadratic equations?}
    \framesubtitle{CCSS: HSF.IF.C.7 Analyze functions  \hspace{\stretch{1}}  \alert{3.7}}

    \begin{block}{Do Now: Quadratic function practice}
      \begin{enumerate}
          \item Write the function $f(x)=x^2-10x-24$ in factored form.
          \item Complete the square and write the function $g(x)=x^2-10x+24$ in vertex form.
          \item The function $h(x)$ has $x$-intercepts of 1 and 5, and a $y$-intercept of 10. Express $h(x)$ in standard form.
      \end{enumerate}
    \end{block}
    %Review formula sheets\\
    Lesson: Solving problems involving quadratics pp. 53-4 \\
    Examples 17, 18\\ \bigskip
    Homework: Exercises 2K \#1-4 p. 55
  }

\section{3.8 Drui: Laptop, Deltamath, Desmos /Word. Tuesday Oct 30}
  \frame
  {
    \frametitle{How do we communicate mathematical results?}
    \framesubtitle{CCSS: MP.4 Model with mathematics \hspace{\stretch{1}} \alert{3.8}}

    \begin{block}{Technical skills needed to communicate mathematics}
    \begin{enumerate}
        \item Word processing: Microsoft Word and equation editor
        \item Computer calculators: Desmos; domain restriction, labeling
        \item Cloud storage: Dropbox
        \item Technical writing standards: MLA format (Purdue OWL)
        \item Writing style: declarative
        \item Assessment criteria: IB exploration criterion \emph{B: Mathematics Presentation}
    \end{enumerate}
    \end{block}
    Lesson: Deltamath individualized instruction on quadratics\\ \bigskip
    Homework: Deltamath followup, 10pm deadline. Open textbook online
  }

\section{3.9 Drui: Applications of quadratics, Wednesday Oct 31}
  \frame
  {
    \frametitle{GQ: How do we solve problems with quadratic equations?}
    \framesubtitle{CCSS: HSF.IF.C.7 Analyze functions  \hspace{\stretch{1}}  \alert{3.9}}

    \begin{block}{Do Now: Function operations and inverses, review}
      \begin{enumerate}
        \item Given $f(x)=2x-1$ and $g(x)=x^2+1$. Find $f+g$, $f \circ g$, and $(g \circ f)(-1)$.
        \item Graph the function $h=\{(-1,0),(1, 2),(3, 1), (4,5)\}$ and its inverse $h^{-1}$.
        \item Find the inverse of the function $h(x)=5x+2$.
      \end{enumerate}
    \end{block}
    %Review formula sheets\\
    Lesson: Solving problems involving quadratics pp. 53-4 \\
    Problem \#4 p. 55 \\ \bigskip
    Homework: Exercises 2K \#5-10 p. 55-56 (Deltamath)
  }

\section{3.10 Drui: Applications of quadratics, Monday Nov 5}
  \frame
  {
    \frametitle{GQ: How do we solve problems with quadratic equations?}
    \framesubtitle{CCSS: HSF.IF.C.7 Analyze functions  \hspace{\stretch{1}}  \alert{3.10}}

    \begin{block}{Do Now: Graphing practice, handout}
    \end{block}
    %Review formula sheets\\
    Review homework: Solving problems involving quadratics 2K \#1-10 p. 55-56 \\
    IB Exam problems, handout\\ \bigskip
    Homework: Complete handout problems
  }


  \section{3.11 Drui: Laptop, Deltamath, Desmos /Word. Tuesday Nov 13}
    \frame
    {
      \frametitle{How do we communicate mathematical results?}
      \framesubtitle{CCSS: MP.4 Model with mathematics \hspace{\stretch{1}} \alert{3.11 Tuesday Nov 13}}

      \begin{block}{Technical skills needed to communicate mathematics}
      \begin{enumerate}
          \item Word processing: Microsoft Word and equation editor
          \item Computer calculators: Desmos; domain restriction, labeling
          \item Cloud storage: Dropbox
          \item Technical writing standards: MLA format (Purdue OWL)
          \item Writing style: declarative
          \item Assessment criteria: IB exploration criterion \emph{B: Mathematics Presentation}
      \end{enumerate}
      \end{block}
      Lesson: Rewrite Quadratics paper, using model as guide\\ \bigskip
      Homework: New version due in Dropbox folder (print Thursday)\\
      \alert{Final exam Thursday}
    }


  \section{3.12 Drui: Applications of quadratics, Wedneday Nov 14}
    \frame
    {
      \frametitle{GQ: How do we solve problems with quadratic equations?}
      \framesubtitle{CCSS: HSF.IF.C.7 Analyze functions  \hspace{\stretch{1}}  \alert{3.10}}

      \begin{block}{Do Now: Graphing practice, handout}
      \end{block}
      %Review formula sheets\\
      Review homework: Solving problems involving quadratics 2K \#1-10 p. 55-56 \\
      IB Exam problems, handout\\ \bigskip
      Homework: Complete handout problems
    }
\end{document}
