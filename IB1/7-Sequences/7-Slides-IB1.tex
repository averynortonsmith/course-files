\documentclass{beamer}
\usepackage{geometry}
\usepackage[english]{babel}
\usepackage[utf8]{inputenc}
\usepackage{amsmath}
\usepackage{amsfonts}
\usepackage{amssymb}
\usepackage{tikz}
\usepackage{graphicx}
\usepackage{venndiagram}

%\usepackage{pgfplots}
%\pgfplotsset{width=10cm,compat=1.9}
%\usepackage{pgfplotstable}

\setlength{\headheight}{26pt}%doesn't seem to fix warning

\usepackage{fancyhdr}
\pagestyle{fancy}
\renewcommand{\headrulewidth}{0pt}

\lhead{\small{BECA / Dr. Huson / 11.1 IB Math - Unit 7 Sequences and Series}}

\title{11.1 IB Math - Unit 7 Sequences and Series}
\subtitle{Bronx Early College Academy}
\author{Christopher J. Huson PhD}
\date{18-28 March 2019}

\begin{document}

\frame{\titlepage}

\section[Outline]{}
\frame{\tableofcontents}

\section{7.1 Introduction and definitions Monday 18 March}
  \frame
  {
    \frametitle{GQ: How do we work with sequences?}
    \framesubtitle{CCSS: HSF.BF.A.2 Write arithmetic and geometric sequences, use them to model situations \hfill \alert{7.1 Monday 18 March}}

    \begin{block}{Do Now: Complete Investigation - \emph{Saving Money} p. 162}
    \end{block}
    Lesson: Arithmetic sequences, recursion, definitions p.161-6\\[1cm]
    Homework: Exercises 6A (a \& b only) p. 164, 6B p. 166
  }

  \section{7.2 Deltamath recursive notation practice, Tuesday 19 March}
    \frame
    {
      \frametitle{GQ: How do we use recursive notation?}
      \framesubtitle{CCSS: HSF.BF.A.2 Write arithmetic and geometric sequences, use them to model situations \hfill \alert{7.2 Tuesday 19 March}}

      \begin{block}{Deltamath probability practice}
      \end{block}
      Homework: Complete Deltamath exercises
    }

  \section{7.3 Geometric sequences, Wednesday 20 March}
    \frame
    {
      \frametitle{GQ: How do we model compound growth?}
      \framesubtitle{CCSS: HSF.BF.A.2 Write arithmetic and geometric sequences, use them to model situations \hfill \alert{7.3 Wednesday 20 March}}

      \begin{block}{Do Now: Exercise 6C p. 167}
      \end{block}
      Lesson: Geometric sequences, Sigma notation p.167-171\\[1cm]
      Homework: Exercises 6D, 6E, 6F (odds only) p. 168, 169, 171
    }

  \section{7.4 Arithmetic series, Thursday 21 March}
    \frame
    {
      \frametitle{GQ: How do we calculate the sum of a sequence?}
      \framesubtitle{CCSS: HSF.BF.A.2 Write arithmetic and geometric sequences, use them to model situations \hfill \alert{7.4 Thursday 21 March}}

      \begin{block}{Do Now: Exercise 6E \#4, \#6 p. 169-170}
      \end{block}
      Lesson: Arithmetic series, Sigma notation p.167-171\\[1cm]
      Homework: Exercises 6G, 6H (odds only) p.  173-5
    }

  \section{7.5 Geometric series, Monday 24 March}
    \frame
    {
      \frametitle{GQ: How do we calculate the sum of a sequence?}
      \framesubtitle{CCSS: HSF.BF.A.2 Write arithmetic and geometric sequences, use them to model situations \hfill \alert{7.5 Monday 24 March}}

      \begin{block}{Do Now: Review exercise \#1, \#2a, 2b, \#3 p. 189}
      \end{block}
      Lesson: Geometric series p. 175-7\\[1cm]
      Homework: Exercises 6I, 6J (a, c only) p.  176, 178
    }

\end{document}
