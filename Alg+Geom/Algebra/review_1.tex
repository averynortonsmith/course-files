\documentclass[12pt, oneside]{article}
\usepackage[letterpaper, margin=1in, headsep=0.5in]{geometry}
\usepackage[english]{babel}
\usepackage[utf8]{inputenc}
\usepackage{amsmath}
\usepackage{amsfonts}
\usepackage{amssymb}
\usepackage{multicol}
\usepackage{tikz}
\usetikzlibrary{quotes, angles}
\usepackage{graphicx}
%\usepackage{pgfplots}
%\pgfplotsset{width=10cm,compat=1.9}
%\usepgfplotslibrary{statistics}
%\usepackage{pgfplotstable}
%\usepackage{tkz-fct}
%\usepackage{venndiagram}

\usepackage{fancyhdr}
\pagestyle{fancy}
\fancyhf{}
\rhead{\thepage \\Name: \hspace{1.5in}.\\}
\lhead{BECA / Dr. Huson / 10.3 Geometry\\* 15 January 2019}

\renewcommand{\headrulewidth}{0pt}

\begin{document}

\section{Calculating slopes from tables}

\item Find the slope of the function from the ratio of the line differences.

  \begin{multicols}{2}
  \begin{enumerate}
    \item
      \begin{tabular}{|c|r|}
      \hline
      $x$ & $f(x)$\\
      \hline
      -2 & -5 \\
      \hline
      -1 & -1 \\
      \hline
      0 & 3 \\
      \hline
      1 & 7 \\
      \hline
      2 & 11 \\
      \hline
      \end{tabular}\\[0.85cm]

      Change in $y$ $= \rule{2cm}{0.15mm}$ \\[0.5cm]
      Change in $x$ $= \rule{2cm}{0.15mm}$ \\[0.5cm]
      Slope $= \rule{2cm}{0.15mm}$\\


    \item
      \begin{tabular}{|c|r|}
      \hline
      $x$ & $f(x)$\\
      \hline
      -4 & 7 \\
      \hline
      -2 & 1 \\
      \hline
      -1 & -2 \\
      \hline
      2 & -11 \\
      \hline
      4 & -17 \\
      \hline
      \end{tabular}\\[0.85cm]

      Change in $y$ $= \rule{2cm}{0.15mm}$ \\[0.5cm]
      Change in $x$ $= \rule{2cm}{0.15mm}$ \\[0.5cm]
      Slope $= \rule{2cm}{0.15mm}$\\

    \end{enumerate}
    \end{multicols}

  \item Find the slope of the function. If the rate of change is not constant, write, ``Non-linear. The rate of change is not constant."

    \begin{multicols}{2}
    \begin{enumerate}
      \item
        \begin{tabular}{|c|c|r|c|}
          \hline
          $\Delta x$ & $x$ & $f(x)$ & $\Delta y$\\
          \hline
          & -5 & -3 &\\
          \hline
          & -1 & -1 &\\
          \hline
          & 0 & 1 &\\
          \hline
          & 2 & 3 &\\
          \hline
          & 5 & 5 &\\
          \hline
        \end{tabular}\\[0.85cm]

        Slope $= \rule{2cm}{0.15mm}$\\

        \item
          \begin{tabular}{|c|c|r|c|}
            \hline
            $\Delta x$ & $x$ & $f(x)$ & $\Delta y$\\
            \hline
            & -3 & 0 &\\
            \hline
            & -1 & 2 &\\
            \hline
            & 0 & 3 &\\
            \hline
            & 2 & 5 &\\
            \hline
            & 5 & 8 &\\
            \hline
          \end{tabular}\\[0.85cm]

          Slope $= \rule{2cm}{0.15mm}$\\

        \end{enumerate}
      \end{multicols}

      \section{Solving quadratics}

      \begin{enumerate}
        \item Solve $x^2+7x+12=0$ by factoring. Then check with the quadratic formula. \vspace{4cm}
        \item Solve $2x^2+9x+7=0$ with the quadratic formula. \vspace{4cm}
        \item Solve $9x^2+8x-1=0$ with the quadratic formula. \vspace{4cm}
      \end{enumerate}
      \newpage
      \subsubsection*{Algebra Review}
      Solve for $x$.
      \begin{multicols}{2}
        \begin{enumerate}
          \item $2x+9=5x-6$ \vspace{4cm}
          \item $4(x - 3) = 6x + 18$ \vspace{4cm}
          \item $5 + 2(x + 3) = -7$ \vspace{4cm}
          \item $\dfrac{x}{8}=5$ \vspace{4cm}
          \item $\dfrac{x}{3}+17=24$ \vspace{4cm}
          \item $\dfrac{24}{x}+9=15$ \vspace{4cm}
          \item $\dfrac{1}{2}(x + 5)=7$ \vspace{4cm}
          \item $\dfrac{3}{x}(2x + 8)=18$ \vspace{4cm}
        \end{enumerate}
      \end{multicols}
      \pagebreak
      For early finishers:
      \begin{multicols}{2}
      \begin{enumerate}
        \item $\dfrac{2}{3}(5x+8)=12$ \vspace{7cm}
        \item $5 - \dfrac{2}{3}(7x+4)= -7$ \vspace{7cm}
        \item $3 - \dfrac{2}{x}(2x+18) = 17$ \vspace{7cm}
        \item $4 + \dfrac{4}{x}(-6x-8) = 12$ \vspace{7cm}
      \end{enumerate}
    \end{multicols}



\end{document}
