\documentclass{beamer}
\usepackage{geometry}
\usepackage[english]{babel}
\usepackage[utf8]{inputenc}
\usepackage{amsmath}
\usepackage{amsfonts}
\usepackage{amssymb}
\usepackage{tikz}
\usetikzlibrary{quotes, angles}
\usepackage{graphicx}

%\usepackage{pgfplots}
%\pgfplotsset{width=10cm,compat=1.9}
%\usepackage{pgfplotstable}

\usepackage{fancyhdr}
\pagestyle{fancy}
\setlength{\headheight}{12pt}%doesn't seem to fix warning
\fancyhf{}

%\rhead{\small{10 September 2018}}
\lhead{\small{BECA / Dr. Huson / Geometry Unit 4: Parallels and transversals}}

\renewcommand{\headrulewidth}{0pt}

\title{Mathematics Class Slides}
\subtitle{Bronx Early College Academy}
\author{Chris Huson}
\date{18 October 2018}

\begin{document}
\frame{\titlepage}
\section[Outline]{}
\frame{\tableofcontents}


\section{4.1 Transversals \& parallel lines, 18 October}
  \frame
  {
    \frametitle{GQ: How do we work with parallel lines?}
    \framesubtitle{CCSS: HSG.CO.A.1 Know precise geometric definitions \hfill \alert{4.1 Friday 18 Oct}}

    \begin{block}{On scrap paper, practice constructions}
    \begin{enumerate}
      \item A perpendicular through a point on a lines
      \item Bisect an obtuse angle
      \item Spicy: a hexagon (six adjacent equilateral triangles)
    \end{enumerate}
    \end{block}
    Review Khan Academy homework (worksheet homework makeup)\\
    Lesson: Parallel lines crossed by a transverse line \\
    Corresponding angles, alternate and same-side relationships \\*[5pt]
    Axiom: corresponding angles are congruent when a transverse line intersects two parallels\\*[5pt]
    Homework: Problem set 4-1 online Khan Academy
  }

\section{4.2 Triangle internal angle measures sum to 180, 21 October}
  \frame
  {
    \frametitle{GQ: How do we calculate the sum of a $\triangle$'s internal angle measures?}
    \framesubtitle{CCSS: HSG.CO.A.1 Know precise geometric definitions \hfill \alert{4.2 Monday 21 Oct}}

    \begin{block}{Exam followup}
    \begin{enumerate}
      \item $B$ bisects $\overline{AC}$ with $AB=3x$, $AC=24$. Diagram \& solve.
      \item A ray's end point is $T$. It extends through point $P$. Diagram \& name using proper notation.
      \item Dr. Huson commutes from 80th Street. At what street is he half way to BECA?
      \item Exam early finishers problems
    \end{enumerate}
    \end{block}
    Review exam results; Test corrections due \alert{Friday}\\
    Peer review of angle bisector papers \\
    Lesson: Sum of a triangle's internal angle measures is $180^\circ$ \\*[5pt]
    Homework: Problem set 4-2 Khan Academy complex angle situations
  }

\section{4.3 Laptops - Revision of angle bisector, 22 October}
  \frame
  {
    \frametitle{GQ: How do we construct an angle bisector?}
    \framesubtitle{CCSS: HSG.CO.A.1 Know precise geometric definitions \hfill \alert{4.3 Tuesday 22 Oct}}

    \begin{block}{Laptops: Construct an angle bisector}
    \begin{enumerate}
      \item Use Geogebra to construct an angle bisector
      \item Write a short (one page) paper presenting your work
      \begin{itemize}
        \item Use MS Word and follow MLA standards. (save as a template to the cloud)
        \item What is the first step in your construction? What is its center?
        \item How does Geogebra adjust the circles and rays as you move things around?
      \end{itemize}
    \end{enumerate}
    \end{block}
    Early finishers:  Khan Academy practice with parallel lines and triangles \\*[5pt]
    Homework: Pretest problem set 4-3
  }

  \section{4.4 Sum of a polygon's internal angle measures, 23 October}
  \frame
  {
    \frametitle{GQ: How do we calculate the sum of a polygon's internal angle measures?}
    \framesubtitle{CCSS: HSG.CO.A.1 Know precise geometric definitions \hfill \alert{4.4 Wednesday 23 Oct}}
  
    \begin{block}{Do Now: Area and perimeter, volume}
    \begin{itemize}
      \item Area of a rectangle, parallelogram, and triangle
      \item Volume of a rectangular prism
      \item Solving for a missing dimension given the area or volume
    \end{itemize}
    \end{block}
    Lesson: Polygons, the volume formula for a pyramid \\
    Sum of a polygon's internal angle measures is $(n-2) \times 180^\circ$ \\*[5pt]
    Homework: Problem set 4-4 Khan Academy polygon internal angles
  }

  \section{4.5 Triangle, polygon external angle measures, radicals, 24 October}
  \frame
  {
    \frametitle{GQ: How do we work with square roots?}
    \framesubtitle{CCSS: HSG.CO.A.1 Know precise geometric definitions \hfill \alert{4.5 Thursday 24 Oct}}
  
    \begin{block}{Do Now: angle measures in parallelogram and polygon situations}
    \begin{enumerate}
      \item Triangle external angles
      \item Consecutive internal angles of a parallelogram
      \item Polygon external angles
    \end{enumerate}
    \end{block}
    Types of quadrilaterals, area of a trapezoid \\
    Triangle inequality theorem \\
    Lesson: simplifying radicals, rounding \\*[5pt]
    Test corrections due \alert{tomorrow} \\
    Homework: 4-5 Khan Academy Triangle side length rules
  }

\section{4.6 Laptops - Deltamath setup and Exit quiz, 25 October}
\frame
{
  \frametitle{GQ: How do we communicate patterns polygons follow?}
  \framesubtitle{CCSS: HSG.CO.A.1 Know precise geometric definitions \hfill \alert{4.6 Friday Oct}}

  \begin{block}{Online Deltamath practice: Khan Academy assignment}
  \begin{enumerate}
    \item Complete assignments in order (1-5 problems each standard)
    \item Show work on lined paper, to be handed in
    \item Early finishers: make up Khan Academy assignments, projects
  \end{enumerate}
  \end{block}
  www.Deltamath.com Teacher ID \alert{546068}\\*[5pt]
  Exit \alert{Pop Quiz}: Deltamath (10 minutes) \\*[5pt]
  Homework: Problem set 4-6 Khan Academy review problems
}

\section{4.7 Triangle and polygon external angle measures, 28 October}
\frame
{
  \frametitle{GQ: How do we construct parallel lines?}
  \framesubtitle{CCSS: HSG.CO.A.1 Know precise geometric definitions \hfill \alert{4.7 Monday 28 Oct}}

  \begin{block}{Do Now: angle measures in parallelograms \& polygons}
  \begin{enumerate}
    \item Triangle external angles
    \item Consecutive internal angles of a parallelogram
    \item Polygon internal angles
  \end{enumerate}
  \end{block}
  Lesson: Construction handout: duplicate a segment, angle \\
  Construction of a parallel line through a point\\
  Inserting a table in Microsoft Word \\*[5pt]
  %scale factors
  Homework: 4-7 handout, Triangle angle sum situations
}

\frame
{
  \frametitle{Proportion and scale factors ($k$)}
  \framesubtitle{CCSS: HSG.CO.A.1 Know precise geometric definitions \hfill \alert{4.7 Monday 28 Oct}}

  \begin{block}{Classwork: Work these problems in your notebook using algebraic notation}
  \begin{enumerate}
    \item Dr. Huson's commute is 84 blocks. How long is 75\% of his ride?
    \item An ivy plant is 3 inches long. If it triples in length over the month of November, how long will it be?
    \item A water tank in the shape of a prism is 20 cm long, 10 cm deep, \& 15 cm tall. How much water does it hold if it is 80\% full? (1 milliliter = 1 cubic centimeter)
    \item The segment $\overline{AB}$ is doubled in length to make $\overline{ABC}$. If $AC = 6.4$, find $AB$.
  \end{enumerate}
  \end{block}
}

\section{Classwork self-assessment}
\frame
{
\frametitle{GQ: How do we work productively?}
\framesubtitle{CCSS: IB Trait - Reflection \hfill \alert{4.7 Monday 28 Oct}}
\begin{block}{Classwork engagement assessment criteria}
  \begin{enumerate}
    \item Respectful - Quietly and attentively listen to speaker. Speak loudly and clearly when called on.
    \item Notes - Take out paper \& notebook. Show work as algebra equations. Write definitions, formulas, theorems, and examples in your notebook.
    \item Work - Independently and quietly apply yourself to assignments. Start quickly, be ready to leave on time.
    \item Get help - Show grit, don't give up easily. Check your notes, then click for hint. Ask your partner. Raise your hand, but if the teacher is not immediately free, keep working.
  \end{enumerate}
  \end{block}
  Rate yourself 0-10 for each criterion\\
  Write it on the top right of your paper before you turn it in.
  }

\section{4.8 Laptop Project - Polygon angle sum table in Word, 29 October}
\frame
{
\frametitle{GQ: How do we communicate patterns polygons follow?}
\framesubtitle{CCSS: HSG.CO.A.1 Know precise geometric definitions \hfill \alert{4.8 Tuesday 29 Oct}}

\begin{block}{Project: Exploration of a polygon's internal angles measures}
\begin{enumerate}
  \item Use Geogebra \& MS Word to write a 1-2 page paper
  \item Include a polygon with dotted diagonals. \\
  Spicy: add color, marked angle measures
  \item In MS Word add a table and use the equation editor. \\
  Spicy: put equations in the table, add a caption to the table
  \item Follow MLA format. If not a single page, manage  page break
  \item Email pdf and MS Word .docx files \\
  Subject line: Polygon exploration
\end{enumerate}
\end{block}
Homework: Complete exploration paper (10:00 deadline)
}

\section{4.9 Isosceles triangle base theorem, angle duplication 30 October}
\frame
{
  \frametitle{GQ: How do we construct parallel lines?}
  \framesubtitle{CCSS: HSG.CO.A.1 Know precise geometric definitions \hfill \alert{4.9 Wednesday 30 Oct}}

  \begin{block}{Do Now: Constructions}
  \begin{enumerate}
    \item Segment bisector
    \item Perpendicular through a point
    \item Triangle circumcenter
  \end{enumerate}
  \end{block}
  Lesson: Construction handout: duplicate a segment, angle \\
  Construction of a parallel line through a point\\
  Isosceles triangle base theorem \\*[5pt]
  Homework: Problem set 4-9 (study and review for test \alert{Friday})
}

\section{4.x Review for unit exam, 31 October}
\frame
{
  \frametitle{GQ: What do you get when you divide the \\ \hspace{1cm} circumference of a pumpkin by its diameter?}
  \framesubtitle{\\ CCSS: HSG.CO.A.1 Know precise geometric definitions \hfill \alert{4.10 Thursday 31 Oct}}

  \begin{block}{Classwork: Pretest problem set}
  \begin{enumerate}
    \item Constructions
    \item Parallel lines intersecting a transversal
    \item Bisectors of angles and line segments
    \item Triangle angle sums
  \end{enumerate}
  \end{block}
  Lesson: Exam review \\*[5pt]
  Homework: Study and review for \alert{test tomorrow}
}

\section{4.x Unit exam, 1 November}

\end{document}

