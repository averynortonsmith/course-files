\documentclass{beamer}
\usepackage{geometry}
\usepackage[english]{babel}
\usepackage[utf8]{inputenc}
\usepackage{amsmath}
\usepackage{amsfonts}
\usepackage{amssymb}
\usepackage{tikz}
\usetikzlibrary{quotes, angles}
\usepackage{graphicx}

%\usepackage{pgfplots}
%\pgfplotsset{width=10cm,compat=1.9}
%\usepackage{pgfplotstable}

\usepackage{fancyhdr}
\pagestyle{fancy}
\setlength{\headheight}{12pt}%doesn't seem to fix warning
\fancyhf{}

%\rhead{\small{10 September 2018}}
\lhead{\small{BECA / Dr. Huson / Geometry Unit 4: Parallels and transversals}}

\renewcommand{\headrulewidth}{0pt}

\title{Mathematics Class Slides}
\subtitle{Bronx Early College Academy}
\author{Chris Huson}
\date{18 October 2018}

\begin{document}
\frame{\titlepage}
\section[Outline]{}
\frame{\tableofcontents}


\section{4.1 Transversals \& parallel lines, 18 October}
  \frame
  {
    \frametitle{GQ: How do we work with parallel lines?}
    \framesubtitle{CCSS: HSG.CO.A.1 Know precise geometric definitions \hfill \alert{4.1 Friday 18 Oct}}

    \begin{block}{On scrap paper, practice constructions}
    \begin{enumerate}
      \item A perpendicular through a point on a lines
      \item Bisect an obtuse angle
      \item Spicy: a hexagon (six adjacent equilateral triangles)
    \end{enumerate}
    \end{block}
    Review Khan Academy homework (worksheet homework makeup)\\
    Lesson: Parallel lines crossed by a transverse line \\
    Corresponding angles, alternate and same-side relationships \\*[5pt]
    Axiom: corresponding angles are congruent when a transverse line intersects two parallels\\*[5pt]
    Homework: Problem set 4-1 online Khan Academy
  }

\section{4.2 Sum of a triangle's internal angle measures is 180 degrees, 21 October}
  \frame
  {
    \frametitle{GQ: How do we calculate the sum of a $\triangle$'s internal angle measures?}
    \framesubtitle{CCSS: HSG.CO.A.1 Know precise geometric definitions \hfill \alert{4.2 Monday 21 Oct}}

    \begin{block}{Angle situations}
    \begin{enumerate}
      \item Model each situation. You do not need to solve the equation, but circle where it states what to solve for.
      \item Three angles are supplementary
      \item Model area leading to a quadratic equation
      \item Use the properties of radii of a circle
    \end{enumerate}
    \end{block}
    Lesson: Sum of a triangle's internal angle measures is $180^\circ$ \\*[5pt]
    Homework: Problem set 4-2
  }

\section{4.3 Laptops, 22 October}
  \frame
  {
    \frametitle{GQ: How do we construct an angle bisector?}
    \framesubtitle{CCSS: HSG.CO.A.1 Know precise geometric definitions \hfill \alert{4.3 Tuesday 22 Oct}}

    \begin{block}{Laptops: Construct an angle bisector}
    \begin{enumerate}
      \item Use Geogebra to construct an angle bisector
      \item Write a short (one page) paper presenting your work
      \begin{itemize}
        \item Use MS Word and follow MLA standards. (save as a template to the cloud)
        \item What is the first step in your construction? What is its center?
        \item How does Geogebra adjust the circles and rays as you move things around?
      \end{itemize}
    \end{enumerate}
    \end{block}
    Early finishers:  Khan Academy practice with parallel lines \\*[5pt]
    Homework: Pretest problem set 4-3
  }

\section{3.8 Review for unit exam, 16 October}

\section{4.x Unit exam, 1 November}

\end{document}