\documentclass{beamer}
\usepackage{geometry}
\usepackage[english]{babel}
\usepackage[utf8]{inputenc}
\usepackage{amsmath}
\usepackage{amsfonts}
\usepackage{amssymb}
\usepackage{tikz}
\usetikzlibrary{quotes, angles}
\usepackage{graphicx}

%\usepackage{pgfplots}
%\pgfplotsset{width=10cm,compat=1.9}
%\usepackage{pgfplotstable}

\setlength{\headheight}{26pt}%doesn't seem to fix warning

\usepackage{fancyhdr}
\pagestyle{fancy}
\fancyhf{}

%\rhead{\small{3 September 2019}}
\lhead{\small{BECA / Dr. Huson / Geometry Unit 1}}

\renewcommand{\headrulewidth}{0pt}

\title{Mathematics Class Slides}
\subtitle{Bronx Early College Academy}
\author{Christopher J. Huson PhD}
\date{5-21 September 2019}

\begin{document}
\frame{\titlepage}
\section[Outline]{}
\frame{\tableofcontents}

\section{1.1 First day of Geometry 5 Sept}
\frame
{
  \frametitle{GQ: How do we define the basic elements of geometry?}
  \framesubtitle{CCSS: HSG.CO.A.1 Know precise geometric definitions \hfill \alert{1.1 Thursday 5 Sept}}

  Welcome back to school
  \begin{block}{Do Now: Algebra skills check}
  \begin{enumerate}
      \item Assigned seating
      \item Take out notebooks (or blank paper) \& calculator (or \$20)
      \item Complete Do Now handout
  \end{enumerate}
  \end{block}
  Supply list: Notebook, folder, looseleaf, pencils \& pens, \\*
  compass and ruler, calculator \alert{(due Monday)} \\
  Lesson: Measurement, drawing, geometric calculations \\
  Homework: Algebra skills practice
}

\section{1.2 Drawing and construction tools 6 Sept}
\frame
{
  \frametitle{GQ: How do we construct geometric figures?}
  \framesubtitle{CCSS: HSG.CO.D.12 Congruence, Make geometric constructions  \hfill \alert{1.2 Friday 6 Sept}}

  \begin{block}{Do Now: Copy definitions into notebook}
  %\begin{enumerate}
      %\item
  %\end{enumerate}
  \end{block}
  Lesson: Definitions: point, line, plane, ray, segment, end point, colinear, coplanar, congruent, distance or length, angle, vertex \\*[5pt]
  Homework review\\
  Practice compass use: "flower of life"\\
  Calculator deposits \$20 \\*[5pt]
  Homework: Problem set 1-2 Vocabulary and terminology
}

\section{1.3 Equilateral triangle construction}
\frame
{
  \frametitle{GQ: How do we construct an equilateral triangle?}
  \framesubtitle{CCSS: HSG.CO.D.13 Construct an equilateral triangle \hfill \alert{1.3 Monday 9 Sept}}

  \begin{block}{Do Now Quiz}
  \begin{enumerate}
      \item Notation and terminology
  \end{enumerate}
  \end{block}
  Lesson: Circle notation; ``Sketch", ``draw", ``construct"; ``Given"\\[5pt]
  Euclid's first construction
  \begin{enumerate}
      \item Steps in the construction
      \item Logic: Why does it work?
      \item Assessment criteria: precision, correct \& complete, elegance or beauty
  \end{enumerate}
  \vspace{0.5cm}
  Lesson: Opposite rays, intersection, p. 13-15\\
  Segment addition postulate (classwork handout)\\
  Homework: Geometry and algebra practice\\
  Due: notebook, folder, compass, ruler, protractor, calculator
}

\section{1.4 Drui}
\frame
{
  \frametitle{GQ: How do we work in three dimensions?}
  \framesubtitle{CCSS: HSG.CO.A.1 Know precise geometric definitions  \hfill \alert{1.4 Tuesday 10 Sept}}

  \begin{block}{Do Now: I have a compass, notebook, and calculator}
  \begin{enumerate}
      \item Tools checklist
      \item Notation
      \item Segment addition, absolute value
      \item Equilateral triangle construction\\
      What is the actual measured length of $DE$? (in centimeters)
  \end{enumerate}
  \end{block}
  Plane geometry pp. 14-17\\ Problems \#1-20 p.16\\
  \vspace{0.5cm}
  Homework: Pre-quiz geometry and algebra practice\\
  Construction project due Friday (hexagon challenge)
}

\frame
{
  \frametitle{A step-by-step guide to solving geometry problems}
  \framesubtitle{Segment addition postulate \hfill \alert{1.2 Friday 6 Sept}}
  Given $\overline{ABC}$, $AB=3x-7$, $BC=x+5$, $AC=14$. Find ${AB}$.\\[0.5in]
     \begin{tikzpicture}
      \draw [-, thick] (0,0)--(7,0);
      \draw [fill] (0,0) circle [radius=0.05] node[below]{$A$};
      \draw [fill] (3,0) circle [radius=0.05] node[below]{$B$};
      \draw [fill] (7,0) circle [radius=0.05] node[below]{$C$};
    \end{tikzpicture} \vspace{1cm}
\begin{enumerate}
    \item<2-> Sketch and label the situation\\
    \item<2-> Write a geometric equation\\
    \item<2-> Substitute algebraic values\\
    \item<2-> Solve for the unknown\\
    \item<2-> Answer the question\\
    \item<2-> Check your answer
  \end{enumerate}
}


\section{1.5 Drui}
  \frame
  {
    \frametitle{GQ: How do we measure distance?}
    \framesubtitle{CCSS: HSG.CO.D.12 Congruence, Make geometric constructions \hfill \alert{1.2 Friday 6 Sept}}

    \begin{block}{Do Now: I have a ruler and protractor}
    \begin{enumerate}
        \item Plane geometry
        \item Measure lengths
    \end{enumerate}
    \end{block}
    Midpoint, bisector, step-by-step solving distance algebra\\
    Problems pp. 20-24, \#1-5, 8-20\\
    \vspace{0.5cm}
    Homework: Distance algebra problems, challenge p. 26\\
    Construction project due tomorrow (hexagon challenge)
  }

\section{1.6 Drui}
  \frame
  {
    \frametitle{GQ: How do we measure angles?}
    \framesubtitle{CCSS: HSG.CO.A.1 Know precise geometric definitions \hfill \alert{1.2 Friday 6 Sept}}

    \begin{block}{Do Now quiz - \alert{Project due}}
    \begin{enumerate}
        \item Notation
        \item Segment addition, absolute value
        \item Equilateral triangle construction
    \end{enumerate}
    \end{block}
    Angle terminology, types; protractor use pp. 27-29\\
    Problems \#1-20 (odds) p.31\\
    \vspace{1cm}
    Homework: Angle notation
  }

\section{1.7 Drui}
  \frame
  {
    \frametitle{GQ: How do we add angle measures?}
    \framesubtitle{CCSS: HSG.CO.A.1 Know precise geometric definitions \hfill \alert{1.2 Friday 6 Sept}}

    \begin{block}{Do Now: Practice and review}
    \begin{enumerate}
        \item Segment addition problem
        \item Equilateral triangle construction
        \item Segment replication
    \end{enumerate}
    \end{block}
    Review common errors of Friday's Do Now Quiz\\
    Angle addition postulate pp. 30-32\\
    Classwork problems 7-23 odds pp. 31-32\\
    \vspace{1cm}
    Homework: Angle measure algebra problems\\
    Parent-teacher meeting tomorrow 4:30-7:30 \alert{Participation credit!}
  }


\section{1.8 Drui Tuesday Sept 18}
  \frame
  {
    \frametitle{GQ: How do we solve bisector problems?}
    \framesubtitle{CCSS: HSG.GPE.B.6 Find the midpoint of a segment \hfill \alert{1.2 Friday 6 Sept}}

    \begin{block}{Do Now: Practice, construction}
    \begin{enumerate}
        \item Use your protractor and ruler to measure and draw.
        \item Use your compass and straightedge to construct an equilateral triangle.
    \end{enumerate}
    \end{block}
    Congruence situations versus addition situations\\
    Angle bisectors\\
    \vspace{1cm}
    Homework: Pre-test\\
    Parent-teacher meeting tomorrow 4:30-7:30 \alert{Participation credit!}
  }

\section{1.9 Drui Thursday Sept 20}
  \frame
  {
    \frametitle{GQ: How do we use the tools of geometry?}
    \framesubtitle{CCSS: HSG.CO.A.1 Know precise geometric definitions \hfill \alert{1.2 Friday 6 Sept}}

    \begin{block}{Do Now: Angle measure practice, construction}
    \begin{enumerate}
        \item Use your protractor, ruler, and compass
    \end{enumerate}
    \end{block}
    Test review\\
    \vspace{1cm}
    Homework: Study for test
  }

  \section{1.10 Drui Friday Sept 21}
    \frame
    {
      \frametitle{GQ: How do we use the tools of geometry?}
      \framesubtitle{CCSS: HSG.CO.A.1 Know precise geometric definitions \hfill \alert{1.2 Friday 6 Sept}}

      \begin{block}{Do Now: (Test)}
      %\begin{enumerate}
          %\item Use your assigned laptop number
      %\end{enumerate}
      \end{block}
      Test\\
      \vspace{1cm}
      Homework: Angle measure algebra problems
    }

\end{document}

\section{2.2 Drui}
  \frame
  {
    \frametitle{GQ: How do we use geometric notation?}
    \framesubtitle{CCSS: HSG.CO.A.1 Know precise geometric definitions \hfill \alert{1.2 Friday 6 Sept}}

    \begin{block}{Do Now: Laptop setup}
    \begin{enumerate}
        \item Use your assigned laptop number
        \item ``Lids down" for group focus
        \item Return laptops to proper slot number, charging cable
    \end{enumerate}
    \end{block}
    Deltamath geometry notation; \\Challenge Geogebra construction\\
    \vspace{1cm}
    Homework: Angle measure, algebra problems
  }
