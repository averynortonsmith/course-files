\documentclass{beamer}
\usepackage{geometry}
\usepackage[english]{babel}
\usepackage[utf8]{inputenc}
\usepackage{amsmath}
\usepackage{amsfonts}
\usepackage{amssymb}
\usepackage{tikz}
\usetikzlibrary{quotes, angles}
\usepackage{graphicx}

%\usepackage{pgfplots}
%\pgfplotsset{width=10cm,compat=1.9}
%\usepackage{pgfplotstable}

\setlength{\headheight}{26pt}%doesn't seem to fix warning

\usepackage{fancyhdr}
\pagestyle{fancy}
\fancyhf{}

%\rhead{\small{10 September 2018}}
\lhead{\small{BECA / Dr. Huson / Geometry Unit 1}}

\renewcommand{\headrulewidth}{0pt}

\title{Mathematics Class Slides}
\subtitle{Bronx Early College Academy}
\author{Chris Huson}
\date{9 October 2018}

\begin{document}
\frame{\titlepage}
\section[Outline]{}
\frame{\tableofcontents}

\section{1c.0 Project criteria}
  \frame
  {
    \frametitle{GQ: How do we present mathematical work?}
    \framesubtitle{CCSS: HSG.CO.D.12 Congruence, Make geometric constructions \hspace{\stretch{1}} \alert{1b.0}}

    Complete binder: \alert{Due Friday}\\
    Exam 1 + corrections; exam 2 (optional corrections); 5 best construction:\\
    Equilateral triangle, Congruent segment \& angles, bisected segment \& angle
      \begin{block}{Criteria for construction projects}
      \begin{enumerate}
          \item Complete and correct construction
          \item Steps written with proper notation
          \item Layout: GQ title, date on left; first \& last name on right
          \item Precise, elegant, mathematical aesthetic
      \end{enumerate}
      \end{block}
    Grading policy: full credit 20, minus 2 points for each missing\\[5pt]
  }

\section{1c.0 Notetaking criteria}
  \frame
  {
    \frametitle{GQ: How do we organize our mathematical notes?}
    \framesubtitle{CCSS: HSG.CO.A.1 Know precise geometric definitions \hspace{\stretch{1}} \alert{1b.0}}

    \begin{block}{Criteria for notebook project grade (20 points)}
    \begin{enumerate}
      \item \alert{Your name and "Geometry" on cover}
      \item \alert{Toward front: math.huson.com, husonbeca@gmail.com, 917-648-5632, Deltamath teacher ID: 546068}
      \item Labeled composition book out during class; GQ, date each day
      \item Definitions, postulates, constructions, \& theorems
      \item Combination of symbols, diagrams, text (best: your own words)
      \item Examples, but not practice problem sets
    \end{enumerate}
    \end{block}
    Grading policy: daily tracker, pop notebook checks
  }

\section{1c.1 Drui: Deltamath. Tuesday 9 October}
  \frame
  {
    \frametitle{GQ: How do we use geometric notation?}
    \framesubtitle{CCSS: HSG.CO.D.12 Congruence, Make geometric constructions \hspace{\stretch{1}} \alert{1c-1}}

    \begin{block}{Do Now: Write down Geogebra assignment steps}
    \begin{enumerate}
      \item geogebra.org $>$ New Math Apps $>$ Geometry
      \item Use segment, circle, intersection (point), polygon, \\ \& text (click ``More") tools. Optional: Construction steps
      \item Print preview; print hardcopy after my approval
      \item Login to Geogebra account and save work (title something like ``1c-1\_Construction...")
      \item Reminder: Use your assigned laptop number\\
          Return laptops to proper slot number, charging cable
    \end{enumerate}
    \end{block}
    Geogebra construction, Deltamath practice\\
    Test review \\ \bigskip
    Homework: Complete deltamath (10pm deadline)
  }

\section{1c.2 Drui: Distance formula. Wednesday 10 October}
  \frame
  {
    \frametitle{GQ: How do we calculate distance using the Pythagorean theorem?}
    \framesubtitle{CCSS: HSG.GPE.B.7 Compute areas and perimeters using the distance formula \hspace{\stretch{1}} \alert{1c.2}}

    \begin{block}{Do Now: Midpoint and segment partition practice. Given $A(3,0), B(15,0)$}
    \begin{enumerate}
        \item Find the distance between $A$ and $B$
        \item Find the midpoint of $\overline{AB}$
        \item Find the point one-third of the way from $A$ to $B$.
        \item Find the point three-quarters of the way from $A$ to $B$.
    \end{enumerate}
    \end{block}
    1-7 Length of a segment p. 52\\
    Classwork problems 22-44 odds p. 54\\
    \vspace{0.5cm}
    Homework: Distance formula practice
  }

\section{1c.3 Drui: Distance formula. Thursday 11 October}
  \frame
  {
    \frametitle{GQ: How do we calculate perimeters and areas?}
    \framesubtitle{CCSS: HSG.GPE.B.7 Compute areas and perimeters using the distance formula \hspace{\stretch{1}} \alert{1c.3}}

    \begin{block}{Do Now: Distance practice. Given $D(3,0), E(15,0), F(15,5), G(3,5)$}
    \begin{enumerate}
        \item Sketch $DEFG$
        \item Find $DE$, $EF$, and $DF$
        \item Spicy: Find the area and perimeter of $DEFG$
    \end{enumerate}
    \end{block}
    1-8 Perimeter, area, circumference pp. 59-63; Polygons\\
    Classwork problems 7-26 odds p. 64\\
    \vspace{0.5cm}
    Homework: Perimeter \& area practice
  }

\section{1c.4 Drui: Distance formula. Friday 12 October}
  \frame
  {
    \frametitle{GQ: How do we calculate perimeters and areas?}
    \framesubtitle{CCSS: HSG.GPE.B.7 Compute areas and perimeters using the distance formula \hspace{\stretch{1}} \alert{1c.4}}

    \begin{block}{Do Now: Angle review}
    \begin{enumerate}
        \item Exercises \#20-25 p. 73. (on loose leaf paper)
    \end{enumerate}
    \end{block}
    Partitioning a line segment p. 57\\
    Perimeter, area, circumference, exercises 7-26 odds p. 64\\
    \vspace{0.5cm}
    Homework: Perimeter \& area practice
  }

\end{document}
