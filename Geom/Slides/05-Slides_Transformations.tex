\documentclass{beamer}
\usepackage{geometry}
\usepackage[english]{babel}
\usepackage[utf8]{inputenc}
\usepackage{amsmath}
\usepackage{amsfonts}
\usepackage{amssymb}
\usepackage{tikz}
\usetikzlibrary{quotes, angles}
\usepackage{graphicx}

%\usepackage{pgfplots}
%\pgfplotsset{width=10cm,compat=1.9}
%\usepackage{pgfplotstable}

\usepackage{fancyhdr}
\pagestyle{fancy}
\setlength{\headheight}{12pt}%doesn't seem to fix warning
\fancyhf{}

%\rhead{\small{10 September 2018}}
\lhead{\small{BECA / Dr. Huson / Geometry Unit 5: Transformations, dilation, scale}}

\renewcommand{\headrulewidth}{0pt}

\title{Mathematics Class Slides}
\subtitle{Bronx Early College Academy}
\author{Chris Huson}
\date{4 November 2019}

\begin{document}
\frame{\titlepage}
\section[Outline]{}
\frame{\tableofcontents}


\section{5.1 Transformations intro, dilation constructions, 4 November}
  \frame
  {
    \frametitle{GQ: How do we construct a triangle with double the side lengths?}
    \framesubtitle{CCSS: HSG.CO.A.1 Know precise geometric definitions \hfill \alert{5.1 Monday 4 Nov}}

    \begin{block}{Do Now: Exam early finishers problems}
    \begin{enumerate}
      \item Modeling geometric situations with an algebraic equation
      \item Complex angle combinations
      \item Constructions with a purpose
    \end{enumerate}
    \end{block}
    Review exam results; Test corrections due \alert{Friday}\\
    Dilation constructions \\
    Lesson: Translation, dilation, reflection \\*[5pt]
    Homework: Problem set 5-1 Khan Academy transformations (due Tuesday 10:00PM)
  }

  \section{5.1 Transformations intro, dilation constructions, 4 November}
  \frame
  {
    \frametitle{GQ: How do we notate transformations?}
    \framesubtitle{CCSS: HSG.CO.A.1 Know precise geometric definitions \hfill \alert{5.1 Monday 4 Nov}}

    \begin{block}{Terminology and notation for transformations}
    \begin{enumerate}
      \item A preimage is mapped to the image, $A \rightarrow A'$
      \item Translation or slide: $T_{+1,-3}$ or $(x,y) \rightarrow (x+1,y-3)$ \\ 
      (or as a vector or arrow)
      \item Rotation around a point by an angle measure, $R_{30^\circ, (0,0)}$
      \item Reflection over a line, $r_{x-axis}$
      \item Dilation by a factor $k$ centered at a point, $D_{\times 2, (0,0)}$
    \end{enumerate}
    \end{block}
    Rigid motions or isometries are transformations that maintain lengths and angles (translation, reflection, rotation, but not dilation)
  }


  \section{5.2 Dilation calculations of triangle, 6 November}
  \frame
  {
    \frametitle{GQ: How do we calculate the lengths of $\triangle$s under dilation?}
    \framesubtitle{CCSS: HSG.CO.A.1 Know precise geometric definitions \hfill \alert{5.2 Wednesday 6 Nov}}
  
    \begin{block}{Do Now: Area and perimeter, volume}
    \begin{itemize}
      \item Dilate a given triangle with scale factor $k=3$
      \item Volume of a rectangular prism
      \item Solving for a missing dimension given the area or volume
    \end{itemize}
    \end{block}
    Portfolio binder checklist, due Wednesday (parent conferences)
    Lesson: Triangle in standard position, side length notation \\
    Modeling with $A'B'=k \times AB$ \\*[5pt]
    Homework: 5.2 Khan Academy dilation practice
  }


\section{5.2 Project rubric - Polygon angle sum table in Word}
\frame
{
\frametitle{GQ: How do we communicate patterns polygons follow?}
\framesubtitle{CCSS: HSG.CO.A.1 Know precise geometric definitions \hfill \alert{4.8 Tuesday 29 Oct}}

\begin{block}{Project rubric: Polygon paper, 29 October\\ Use Geogebra \& MS Word to write a 1-2 page paper}
\begin{enumerate} 
  \item Include a polygon (20)
  \item Dotted diagonals (5)\\
  Spicy: add color, marked angle measures (+5, +5)
  \item In MS Word table (20) 
  \item Use the equation editor (20) \\
  Spicy: Caption to the table (+5)
  \item Follow MLA format. (20) \\
  If not a single page, manage page break (-5)
  \item Email pdf and MS Word .docx files\\
  Subject line: Polygon exploration (5) 
\end{enumerate}
\end{block}
}


\section{5.x Unit exam, 1 November}

\end{document}

