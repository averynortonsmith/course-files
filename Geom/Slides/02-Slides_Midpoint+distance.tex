\documentclass{beamer}
\usepackage{geometry}
\usepackage[english]{babel}
\usepackage[utf8]{inputenc}
\usepackage{amsmath}
\usepackage{amsfonts}
\usepackage{amssymb}
\usepackage{tikz}
\usetikzlibrary{quotes, angles}
\usepackage{graphicx}

%\usepackage{pgfplots}
%\pgfplotsset{width=10cm,compat=1.9}
%\usepackage{pgfplotstable}

\setlength{\headheight}{26pt}%doesn't seem to fix warning

\usepackage{fancyhdr}
\pagestyle{fancy}
\fancyhf{}

%\rhead{\small{10 September 2018}}
\lhead{\small{BECA / Dr. Huson / Geometry Unit 1}}

\renewcommand{\headrulewidth}{0pt}

\title{Mathematics Class Slides}
\subtitle{Bronx Early College Academy}
\author{Chris Huson}
\date{24 September 2018}

\begin{document}
\frame{\titlepage}
\section[Outline]{}
\frame{\tableofcontents}

\section{2.1 Midpoint definition \& calculations, 16 Sept}
  \frame
  {
    \frametitle{GQ: How do we bisect a line segment?}
    \framesubtitle{CCSS: HSG.CO.A.1 Know precise geometric definitions \hfill \alert{2.1 Monday 16 Sept}}

    \begin{block}{Segment addition and measurement practice}
    \begin{enumerate}
      \item Equilateral triangle construction
      \item Measuring and calculating length
      \item Segment addition situations
    \end{enumerate}
    \end{block}
    Lesson: Definitions: midpoint, bisect, trisect, perpendicular \\*[5pt]
    Test corrections / analysis\\
    Construction: Perpendicular bisector\\
    Demonstration: Geogebra equilateral triangle\\ \smallskip
    Homework: Problem set 2-1 
  }

  \section{2.2 Laptops: Pupilpath \& Geogebra equilateral triangle, 17 Sept}
  \frame
  {
    \frametitle{GQ: How do we use computer technology?}
    \framesubtitle{CCSS: MP4 Use technology appropriately \hfill \alert{2.2 Tuesday 17 Sept}}
  
    \begin{block}{Do Now: Boot up laptop, Pupilpath, geogebra.org $>$ geometry}
    \begin{enumerate}
        \item Always use the laptop with your number
        \item Write down your Geometry grade (from Pupilpath) in your notebook
        \item Log into your personal email
        \item Open geogebra.org $>$ geometry
        \item Explore \& play!\\
    \end{enumerate}
    \end{block}
    Lesson: Geogebra equilateral triangle construction\\
    ``Lids down" for group focus \\
    Return laptops to proper slot number, charging cable \\ \vspace{0.25cm}
    Homework: Parent Pupilpath checklist
  }

  \frame
  {
    \frametitle{GQ: How do we present mathematical work?}
    \framesubtitle{CCSS: HSG.CO.D.12 Congruence, Make geometric constructions  \hfill \alert{2.2 Monday 17 Sept}}

    \begin{block}{Criteria for construction projects}
    \begin{enumerate}
        \item Complete and correct construction
        \item MLA layout: First \& last name / Dr. Huson / 10.x Geometry / 17 September 2019 \\
        Title centered (no underlining)
        \item Precise, elegant, mathematical aesthetic
        \item Spicy: Steps written with proper notation
    \end{enumerate}
    \end{block}
    Grading policy: full credit or redo\\[5pt]
    (collect exams and projects in your personal classroom binder)
  }

\section{2.3 Triangle area formula, 18 Sept}
    \frame
    {
      \frametitle{GQ: How do we calculate the area of a triangle?}
      \framesubtitle{CCSS: HSG.CO.A.1 Know precise geometric definitions \hfill \alert{2.3 Wednesday 18 Sept}}
  
      \begin{block}{Bisector and measurement practice}
      \begin{enumerate}
        \item Midpoint calculations
        \item Measuring an obtuse angle
        \item Drawing a rectangle
        \item Half rectangle areas
      \end{enumerate}
      \end{block}
      Lesson: Triangle area $A_{\triangle} = \frac{1}{2}bh$ \\*[5pt]
      Midpoint average method, vector method\\ \smallskip
      Homework: Problem set 2-3 
    }

    \section{2.4 How do we solve for a missing value, 19 Sept}
    \frame
    {
      \frametitle{GQ: How do we solve for a missing value?}
      \framesubtitle{CCSS: HSG.CO.A.1 Know precise geometric definitions \hfill \alert{2.4 Thursday 19 Sept}}
  
      \begin{block}{Bisector and area practice}
      \begin{enumerate}
        \item Construct a perpendicular bisector
        \item Midpoint calculations
        \item Triangle area
        \item Segment addition
      \end{enumerate}
      \end{block}
      Lesson: Construct a perpendicular through a point on a line \\*[5pt]
      Solving for an input parameter value\\ \smallskip
      Homework: Problem set 2-4 
    }

    \section{2.5 How do we solve for a missing value, 20 Sept}
    \frame
    {
      \frametitle{GQ: How do we solve for a missing value?}
      \framesubtitle{CCSS: HSG.CO.A.1 Know precise geometric definitions \hfill \alert{2.5 Friday 20 Sept}}
  
      \begin{block}{Do Now Quiz}
      \begin{enumerate}
        \item Construct a perpendicular bisector
        \item Midpoint calculations
        \item Triangle area
        \item Segment addition
      \end{enumerate}
      \end{block}
      Lesson: Construct a perpendicular through a point on a line \\*[5pt]
      Solving for an input parameter value\\ \smallskip
      Homework: Problem set 2-4 
    }

\section{1b.3 Drui: Vertical angles. Wednesday Sept 26}
  \frame
  {
    \frametitle{GQ: How do we classify angle pairs?}
    \framesubtitle{CCSS: HSG.CO.A.1 Know precise geometric definitions \hspace{\stretch{1}} \alert{1b.3}}

    \begin{block}{Do Now: Construction practice and review}
    \begin{enumerate}
        \item Given $\overline{AB}$, construct a congruent line segment
        \item Given $\overline{DE}$, construct an equilateral triangle
    \end{enumerate}
    \end{block}
    1-5 Exploring Angle Pairs pp. 34-37\\
    Classwork problems 7-26 odds pp. 38\\
    Construct a perpendicular bisector \\
    \vspace{0.5cm}
    Homework: Angle pair practice
  }

\section{1b.4 Drui: Construct perpendicular bisector. Thursday Sept 27}
  \frame
  {
    \frametitle{GQ: How do we classify angle pairs?}
    \framesubtitle{CCSS: HSG.CO.A.1 Know precise geometric definitions \hspace{\stretch{1}} \alert{1b.4}}

    \begin{block}{Do Now: Angle pair practice. Show steps, including the check.}
    \begin{enumerate}
        \item Given two supplementary angles: $m \angle 1 = 50$, $m \angle 2 = x$.\\ Find $x$.
        \item Given two complementary angles: $m \angle 1 = x+10$, $m \angle 2 = x+20$. Find $m \angle 1$.
        \item Given two vertical angles: $m \angle 1 = 3x+10$, $m \angle 2 = 55$.\\ Find $x$.
    \end{enumerate}
    \end{block}
    1-5 Exploring Angle Pairs pp. 34-37\\
    Classwork problems 8-30 evens pp. 38-39\\
    Construct a perpendicular bisector \\
    \vspace{0.2cm}
    Homework: Angle pair practice
  }

\section{1b.5 Drui: Constuct angle bisector. Friday Sept 28}
  \frame
  {
    \frametitle{GQ: How do we do classical constructions?}
    \framesubtitle{CCSS: HSG.CO.D.12 Congruence, Make geometric constructions \hspace{\stretch{1}} \alert{1b.5}}
    Do Now: Angle pair practice
    \begin{block}{Constructions due today: Complete, correct, precise, elegant}
      Standard header. (you may combine constructions on same page)
      \begin{enumerate}
          \item Equilateral triangle (you may combine on same page)
          \item Congruent segments
          \item Perpendicular bisector
          \item New: Angle bisector
          \item Spicy: Flower design p. 42
      \end{enumerate}
    \end{block}
    Classwork problems 3-25 odds pp. 41\\
    \vspace{0.5cm}
    Homework: Angle pair practice
  }


\end{document}


  \frame
  {
    \frametitle{A step-by-step guide to solving geometry problems}
    \framesubtitle{Segment addition postulate \hspace{\stretch{1}} \alert{1.4}}
    Given $\overline{ABC}$, $AB=3x-7$, $BC=x+5$, $AC=14$. Find ${AB}$.\\[0.5in]
       \begin{tikzpicture}
        \draw [-, thick] (0,0)--(7,0);
        \draw [fill] (0,0) circle [radius=0.05] node[below]{$A$};
        \draw [fill] (3,0) circle [radius=0.05] node[below]{$B$};
        \draw [fill] (7,0) circle [radius=0.05] node[below]{$C$};
      \end{tikzpicture} \vspace{1cm}
  \begin{enumerate}
      \item<2-> Sketch and label the situation\\
      \item<2-> Write a geometric equation\\
      \item<2-> Substitute algebraic values\\
      \item<2-> Solve for the unknown\\
      \item<2-> Answer the question\\
      \item<2-> Check your answer
    \end{enumerate}
  }
