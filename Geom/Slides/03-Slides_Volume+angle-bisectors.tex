\documentclass{beamer}
\usepackage{geometry}
\usepackage[english]{babel}
\usepackage[utf8]{inputenc}
\usepackage{amsmath}
\usepackage{amsfonts}
\usepackage{amssymb}
\usepackage{tikz}
\usetikzlibrary{quotes, angles}
\usepackage{graphicx}

%\usepackage{pgfplots}
%\pgfplotsset{width=10cm,compat=1.9}
%\usepackage{pgfplotstable}

\usepackage{fancyhdr}
\pagestyle{fancy}
\setlength{\headheight}{12pt}%doesn't seem to fix warning
\fancyhf{}

%\rhead{\small{10 September 2018}}
\lhead{\small{BECA / Dr. Huson / Geometry Unit 3}}

\renewcommand{\headrulewidth}{0pt}

\title{Mathematics Class Slides}
\subtitle{Bronx Early College Academy}
\author{Chris Huson}
\date{2 October 2018}

\begin{document}
\frame{\titlepage}
\section[Outline]{}
\frame{\tableofcontents}

\section{3.1 Volume formula, vertical angle congruence, 2 October}
  \frame
  {
    \frametitle{GQ: How do we calculate the volume of a prism?}
    \framesubtitle{CCSS: HSG.CO.A.1 Know precise geometric definitions \hfill \alert{3.1 Wednesday 2 Oct}}

    \begin{block}{Do Now: Angle terminology}
    \begin{enumerate}
      \item Adjacent \& non-adjacent angles
      \item Complementary \& supplementary angles (perpendicular)
      \item Vertical angles
      \item Organizing the work for a solution
    \end{enumerate}
    \end{block}
    Lesson: Definitions: prism, base, face, edge  \\*[5pt]
    Test corrections / analysis\\
    Theorem: Vertical angles are congruent; classwork practice\\*[5pt]
    Homework: Problem set 3-1 
  }

  \section{3.2 Angle bisectors, construction, 3 October}
    \frame
    {
      \frametitle{GQ: How do we construct the bisector of a given angle?}
      \framesubtitle{CCSS: HSG.CO.A.1 Know precise geometric definitions \hfill \alert{3.2 Thursday 3 Oct}}

      \begin{block}{Do Now: Area and volume practice}
      \begin{enumerate}
        \item Area of a rectangle, square, and triangle
        \item Solving for a unknown parameter
        \item Prism volume calculation, solving for unknown
        \item Vertical angle situations
      \end{enumerate}
      \end{block}
      Lesson: Angle bisectors  \\*[5pt]
      Construct the bisector of an angle\\*[5pt]
      Homework: Problem set 3-2
    }


\section{High School MAP math testing, 4 October}

\section{3.3 Area of a parallelogram, 7 October}
  \frame
  {
    \frametitle{GQ: How do we calculate the area of a parallelogram?}
    \framesubtitle{CCSS: HSG.CO.A.1 Know precise geometric definitions \hfill \alert{3.3 Monday 7 Oct}}

    \begin{block}{Do Now: Area and volume practice}
    \begin{enumerate}
      \item Construct a perpendicular through a point
      \item Volume and the area of a compound shape
      \item Vertical, supplementary, and complementary angles
    \end{enumerate}
    \end{block}
    Lesson: The area of a parallelogram $A=bh$\\*[5pt]
    Solving angle bisector problems\\*[5pt]
    Homework: Problem set 3-3
  }

\section{3.4 Laptop construction paper makeup, 8 October}
  \frame
  {
    \frametitle{GQ: How do we calculate the area of a parallelogram?}
    \framesubtitle{CCSS: HSG.CO.A.1 Know precise geometric definitions \hfill \alert{3.4 Tuesday 8 Oct}}

    \begin{block}{Check Pupilpath and complete missing assignments}
    \begin{enumerate}
      \item Projects: hand construction of equilateral triangle
      \item Geogebra construction of equilateral triangle
      \item MS Word paper with four constructions
      \item email to husonbeca@gmail.com
    \end{enumerate}
    \end{block}
    Lesson: Using KhanAcademy to practice Geometry \\*[5pt]
    Teacher: DrHuson, Class code: MPFQ83\\*[5pt]
    Homework: Online Khan assignment
  }

\section{3.5 Sum of a triangle's internal angle measures is 180 degrees, 10 October}
  \frame
  {
    \frametitle{GQ: How do we calculate the sum of a $\triangle$'s internal angle measures?}
    \framesubtitle{CCSS: HSG.CO.A.1 Know precise geometric definitions \hfill \alert{3.5 Thursday 10 Oct}}

    \begin{block}{Angle situations}
    \begin{enumerate}
      \item Model each situation. You do not need to solve the equation, but circle where it states what to solve for.
      \item Three angles are supplementary
      \item Model area leading to a quadratic equation
      \item Use the properties of radii of a circle
    \end{enumerate}
    \end{block}
    Lesson: Sum of a triangle's internal angle measures is $180^\circ$ \\*[5pt]
    Homework: Problem set 3-5
  }

\section{3.6 Transversals \& parallel lines, 11 October}
  \frame
  {
    \frametitle{GQ: How do we work with parallel lines?}
    \framesubtitle{CCSS: HSG.CO.A.1 Know precise geometric definitions \hfill \alert{3.6 Friday 11 Oct}}

    \begin{block}{On scrap paper, practice constructions}
    \begin{enumerate}
      \item A perpendicular through a point on a lines
      \item Bisect an obtuse angle
      \item Spicy: a hexagon (six adjacent equilateral triangles)
    \end{enumerate}
    \end{block}
    Lesson: Parallel lines crossed by a transverse line \\
    Corresponding angles, alternate and same-side relationships \\*[5pt]
    Axiom: corresponding angles are congruent when a transverse line intersects two parallels\\*[5pt]
    Homework: Problem set 3-6 online Khan Academy
  }

\section{3.7 Laptops, 15 October}
  \frame
  {
    \frametitle{GQ: How do we construct an angle bisector?}
    \framesubtitle{CCSS: HSG.CO.A.1 Know precise geometric definitions \hfill \alert{3.7 Tuesday 15 Oct}}

    \begin{block}{Laptops: Construct an angle bisector}
    \begin{enumerate}
      \item Use Geogebra to construct an angle bisector
      \item Write a short (one page) paper presenting your work
      \begin{itemize}
        \item Use MS Word and follow MLA standards. (save as a template to the cloud)
        \item What is the first step in your construction? What is its center?
        \item How does Geogebra adjust the circles and rays as you move things around?
      \end{itemize}
    \end{enumerate}
    \end{block}
    Early finishers:  Khan Academy practice with parallel lines \\*[5pt]
    Homework: Pretest problem set 3-7
  }

\section{3.8 Review for unit exam, 16 October}
  \frame
  {
    \frametitle{GQ: How do we work with angle bisectors and volumes?}
    \framesubtitle{CCSS: HSG.CO.A.1 Know precise geometric definitions \hfill \alert{3.8 Wednesday 16 Oct}}

    \begin{block}{Do Now: Modeling practice}
    \begin{enumerate}
      \item Segment addition and segment bisectors
      \item Angle bisectors, complementary and supplementary situations
      \item Early finishers: constructions (equilateral triangle, segment bisector, angle bisector, perpendicular through a point)
    \end{enumerate}
    \end{block}
    Roundtable: Review for \alert{test tomorrow} \\*[5pt]
    Note: Any three points are coplanar  \\*[5pt]
    Homework: Problem set 3-8
  }

\section{3.9 Unit exam, 17 October}
    \frame
    {
      \frametitle{GQ: How do we work with angle bisectors and volumes?}
      \framesubtitle{CCSS: HSG.CO.A.1 Know precise geometric definitions \hfill \alert{3.9 Thursday 17 Oct}}

      Assessment: Unit exam, Volumes and angle bisectors\\*[5pt]
      Homework: Problem set 3-9, parallel lines crossed by a transversal
    }

\end{document}