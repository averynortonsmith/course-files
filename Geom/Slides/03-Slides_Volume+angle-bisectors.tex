\documentclass{beamer}
\usepackage{geometry}
\usepackage[english]{babel}
\usepackage[utf8]{inputenc}
\usepackage{amsmath}
\usepackage{amsfonts}
\usepackage{amssymb}
\usepackage{tikz}
\usetikzlibrary{quotes, angles}
\usepackage{graphicx}

%\usepackage{pgfplots}
%\pgfplotsset{width=10cm,compat=1.9}
%\usepackage{pgfplotstable}

\setlength{\headheight}{26pt}%doesn't seem to fix warning

\usepackage{fancyhdr}
\pagestyle{fancy}
\fancyhf{}

%\rhead{\small{10 September 2018}}
\lhead{\small{BECA / Dr. Huson / Geometry Unit 3}}

\renewcommand{\headrulewidth}{0pt}

\title{Mathematics Class Slides}
\subtitle{Bronx Early College Academy}
\author{Chris Huson}
\date{2 October 2018}

\begin{document}
\frame{\titlepage}
\section[Outline]{}
\frame{\tableofcontents}

\section{3.1 Volume formula, vertical angle congruence, 2 October}
  \frame
  {
    \frametitle{GQ: How do we calculate the volume of a prism?}
    \framesubtitle{CCSS: HSG.CO.A.1 Know precise geometric definitions \hfill \alert{3.1 Wednesday 2 Oct}}

    \begin{block}{Do Now: Angle terminology}
    \begin{enumerate}
      \item Adjacent \& non-adjacent angles
      \item Complementary \& supplementary angles (perpendicular)
      \item Vertical angles
      \item Organizing the work for a solution
    \end{enumerate}
    \end{block}
    Lesson: Definitions: prism, base, face, edge  \\*[5pt]
    Test corrections / analysis\\
    Theorem: Vertical angles are congruent; classwork practice\\*[5pt]
    Homework: Problem set 3-1 
  }

  \section{3.2 Angle bisectors, construction, 3 October}
    \frame
    {
      \frametitle{GQ: How do we construct the bisector of a given angle?}
      \framesubtitle{CCSS: HSG.CO.A.1 Know precise geometric definitions \hfill \alert{3.2 Thursday 3 Oct}}

      \begin{block}{Do Now: Area and volume practice}
      \begin{enumerate}
        \item Area of a rectangle, square, and triangle
        \item Solving for a unknown parameter
        \item Prism volume calculation, solving for unknown
        \item Vertical angle situations
      \end{enumerate}
      \end{block}
      Lesson: Angle bisectors  \\*[5pt]
      Construct the bisector of an angle\\*[5pt]
      Homework: Problem set 3-2
    }


  \section{High School MAP math testing, 4 October}

  \section{3.3 Area of a parallelogram, 7 October}
    \frame
    {
      \frametitle{GQ: How do we calculate the area of a parallelogram?}
      \framesubtitle{CCSS: HSG.CO.A.1 Know precise geometric definitions \hfill \alert{3.3 Monday 7 Oct}}

      \begin{block}{Do Now: Area and volume practice}
      \begin{enumerate}
        \item Construct a perpendicular through a point
        \item Volume and the area of a compound shape
        \item Vertical, supplementary, and complementary angles
      \end{enumerate}
      \end{block}
      Lesson: The area of a parallelogram $A=bh$\\*[5pt]
      Solving angle bisector problems\\*[5pt]
      Homework: Problem set 3-3
    }

  \section{3.4 Quiz, Laptop construction paper makeup, 8 October}
    \frame
    {
      \frametitle{GQ: How do we calculate the area of a parallelogram?}
      \framesubtitle{CCSS: HSG.CO.A.1 Know precise geometric definitions \hfill \alert{3.4 Tuesday 8 Oct}}

      \begin{block}{Area and volume practice}
      \begin{enumerate}
        \item --
      \end{enumerate}
      \end{block}
      Lesson:   \\*[5pt]
      Construct\\*[5pt]
      Homework: Problem set 3-4
    }

  \section{3.5 Sum of a triangle's internal angle measures is $180^\circ$, 10 October}
    \frame
    {
      \frametitle{GQ: How do we calculate the sum of a $\triangle$'s internal angle measures?}
      \framesubtitle{CCSS: HSG.CO.A.1 Know precise geometric definitions \hfill \alert{3.5 Thursday 10 Oct}}

      \begin{block}{Angle situations}
      \begin{enumerate}
        \item --
      \end{enumerate}
      \end{block}
      Lesson:   \\*[5pt]
      Construct\\*[5pt]
      Homework: Problem set 3-5
    }

  \section{3.6 Isoceles triangle base theorem, 11 October}
    \frame
    {
      \frametitle{GQ: How do we know the base angles of an isosceles triangle?}
      \framesubtitle{CCSS: HSG.CO.A.1 Know precise geometric definitions \hfill \alert{3.6 Friday 10 Oct}}

      \begin{block}{Line segment practice}
      \begin{enumerate}
        \item --
      \end{enumerate}
      \end{block}
      Lesson:   \\*[5pt]
      Theorem: the base angles of an isosceles triangle are congruent\\*[5pt]
      Homework: Problem set 3-6
    }

  \section{3.7 Angle and volume practice, 14 October}
    \frame
    {
      \frametitle{GQ: How do we work with angle situations?}
      \framesubtitle{CCSS: HSG.CO.A.1 Know precise geometric definitions \hfill \alert{3.7 Monday 14 Oct}}

      \begin{block}{Line segment practice}
      \begin{enumerate}
        \item --
      \end{enumerate}
      \end{block}
      Lesson:   \\*[5pt]
      Homework: Problem set 3-7
    }

  \section{3.8 Laptops, 15 October}
    \frame
    {
      \frametitle{GQ: How do we construct an angle bisector?}
      \framesubtitle{CCSS: HSG.CO.A.1 Know precise geometric definitions \hfill \alert{3.8 Tuesday 15 Oct}}

      \begin{block}{Laptops: Construct an angle bisector}
      \begin{enumerate}
        \item Use Geogebra to construct an angle bisector
        \item Write a short (one page) paper presenting your work
      \end{enumerate}
      \end{block}
      Lesson:   \\*[5pt]
      Homework: Pretest problem set 3-8
    }

\section{3.9 Review for unit exam, 16 October}
  \frame
  {
    \frametitle{GQ: How do we work with angle bisectors and volumes?}
    \framesubtitle{CCSS: HSG.CO.A.1 Know precise geometric definitions \hfill \alert{3.9 Wednesday 16 Oct}}

    \begin{block}{Line segment practice}
    \begin{enumerate}
      \item --
    \end{enumerate}
    \end{block}
    Lesson:   \\*[5pt]
    Construct\\*[5pt]
    Homework: Problem set 3-9
  }

\section{3.10 Unit exam, 17 October}
    \frame
    {
      \frametitle{GQ: How do we work with angle bisectors and volumes?}
      \framesubtitle{CCSS: HSG.CO.A.1 Know precise geometric definitions \hfill \alert{3.10 Thursday 17 Oct}}

      Assessment: Unit exam, Volumes and angle bisectors\\*[5pt]
      Homework: Problem set 3-10
    }

\end{document}