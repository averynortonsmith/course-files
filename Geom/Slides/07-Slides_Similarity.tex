\documentclass{beamer}
\usepackage{geometry}
\usepackage[english]{babel}
\usepackage[utf8]{inputenc}
\usepackage{amsmath}
\usepackage{amsfonts}
\usepackage{amssymb}
\usepackage{tikz}
\usetikzlibrary{quotes, angles}
\usepackage{graphicx}

%\usepackage{pgfplots}
%\pgfplotsset{width=10cm,compat=1.9}
%\usepackage{pgfplotstable}

\usepackage{fancyhdr}
\pagestyle{fancy}
\setlength{\headheight}{12pt}%doesn't seem to fix warning
\fancyhf{}

%\rhead{\small{2 January 2020}}
\lhead{\small{BECA / Dr. Huson / Geometry Unit 7: Similarity}}

\renewcommand{\headrulewidth}{0pt}

\title{Mathematics Class Slides}
\subtitle{Bronx Early College Academy}
\author{Chris Huson}
\date{2 January 2020}

\begin{document}
\frame{\titlepage}
\section[Outline]{}
\frame{\tableofcontents}


\section{7.1 Dilation calculations of triangle, Thursday 2 January}
\frame
{
  \frametitle{GQ: How do we calculate the lengths of $\triangle$s under dilation?}
  \framesubtitle{CCSS: HSG.SRT.B5 Use similarity criteria to solve problems \hfill \alert{7.1 Thursday 2 January}}

  \begin{block}{Do Now: Exam review}
  \begin{itemize}
    \item Dilate a given triangle with scale factor
    \item Applying dilations on the coordinate plane
    \item The parameter $m$ in a function $f(x)=mx+b$
    \item Isosceles triangle review
    \item Graph peer grading
  \end{itemize}
  \end{block}
  Lesson: Dilation and the properties of similar figures, notation\\*[5pt]
  Homework: Complete problem set (Portfolio binder extra credit Monday)
}

\section{7.2 Dilation calculations of triangle, Friday 3 January}
\frame
{
  \frametitle{GQ: How do we use equations to solve geometry problems?}
  \framesubtitle{CCSS: HSG.SRT.B5 Use similarity criteria to solve problems \hfill \alert{7.2 Friday 3 January}}

  \begin{block}{Do Now: Applying the tangent function}
  \begin{enumerate}
    \item Calculate the tangent of an angle using a calculator
    \item Calculate the tangent of an angle given a slope, or $\triangle$ side lengths
    \item Solving for the a triangle's sides given a vertex angle measure
    \item Inverse function on the calculator $tan^{-1}(x)$
  \end{enumerate}
  \end{block}
  Lesson: Review of problems using coordinate geometry \\
  Homework: Complete problem set (Portfolio binder extra credit Monday)
}

\section{7.3 Dilation calculations of triangle, Monday 6 January}
\frame
{
  \frametitle{GQ: How do we use equations to solve geometry problems?}
  \framesubtitle{CCSS: HSG.SRT.B5 Use similarity criteria to solve problems \hfill \alert{7.3 Monday 6 January}}

  \begin{block}{Do Now: Applying the tangent function}
  \begin{enumerate}
    \item Calculate the tangent of an angle using a calculator
    \item Calculate the tangent of an angle given a slope, or $\triangle$ side lengths
    \item Solving for the a triangle's sides given a vertex angle measure
    \item Inverse function on the calculator $tan^{-1}(x)$
  \end{enumerate}
  \end{block}
  Test corrections due. Portfolio binder review for extra credit \\
  Lesson: Angle-angle similarity theorem, the reflexive property \\
  Homework: Complete problem set 
}

\section{7.4 Laptop: Geogebra triangle reflection+dilation, Tuesday 7 January}
\frame
{
  \frametitle{GQ: How do we communicate examples of compositions?}
  \framesubtitle{CCSS: MP5 Use appropriate tools strategically \hfill \alert{7.4 Tuesday 7 January}}

  \begin{block}{Project: Reflection and dilation composition of a $\triangle$}
  \begin{enumerate}
    \item Use Geogebra \& MS Word to write a 1+ page paper
    \item Perform the following operations:
    \begin{enumerate}
      \item Bisect the angle of one vertex of a triangle, $\triangle ABC$
      \item Reflect $\triangle ABC$ across the bisector, creating image $\triangle A'B'C'$
      \item Dilate the image, $\triangle A'B'C' \rightarrow A''B''C''$
    \end{enumerate}
    \item In the text, describe your steps, the mappings and congruences. 
    \item Use proper notation and the equation editor. Follow MLA.
    \item Email a pdf file, subject line: Dilation composition assignment
  \end{enumerate}
  \end{block}
  Homework: Complete exploration paper (10:00 deadline)
}

\section{7.4 Laptop: Composition project assessment criteria, Tuesday 7 January}
\frame
{
  \frametitle{GQ: How do we assess project papers?}
  \framesubtitle{CCSS: MP5 Use appropriate tools strategically \hfill \alert{7.4 Tuesday 7 January}}

  \begin{block}{Project Criteria: Reflection and dilation composition of a $\triangle$}
  \begin{enumerate}
    \item Perform the complete construction in Geogebra. \hfill (30 points)
    \item Describe the steps, mappings, \& congruences.  \hfill (20 points)
    \item Use proper notation, the equation editor, color.  \hfill (15 points)
    \item Follow MLA. \hfill (20 points)
    \item Submit a pdf file \hfill (10 points)
    \item Email subject line: Dilation composition assignment \hfill (5 points)
  \end{enumerate}
  \end{block}
}

\section{7.5 Transformational symmetries, Wednesday 8 January}
\frame
{
  \frametitle{GQ: How do we transform a figure onto itself?}
  \framesubtitle{CCSS: HSG.SRT.B5 Use similarity criteria to solve problems \hfill \alert{7.5 Wednesday 8 January}}

  \begin{block}{Do Now: Dilation situations}
  \begin{enumerate}
    \item Ratio calculations
    \item Corresponding angles and polygon sides
    \item Transformation composition
  \end{enumerate}
  \end{block}
  Lesson: Symmetry as transformations ``onto itself'' \\[0.5cm]
  Homework: Transformations problem set (\alert{Test Friday})
}

\section{7.6 Transformational symmetries, Thursday 9 January}
\frame
{
  \frametitle{GQ: How do we transform a figure onto itself?}
  \framesubtitle{CCSS: HSG.SRT.B5 Use similarity criteria to solve problems \hfill \alert{7.6 Thursday 9 January}}

  \begin{block}{Do Now: Dilation situations}
  \begin{enumerate}
    \item Ratio calculations
    \item Corresponding angles and polygon sides
    \item Transformation composition
  \end{enumerate}
  \end{block}
  Lesson: Symmetry as transformations ``onto itself'' \\[0.5cm]
  Homework: Transformations problem set (\alert{Test tomorrow})
}

\section{7.7 Unit exam: Similarity, Friday 10 January}
\frame
{
  \frametitle{GQ: How do we apply transformations to solve problems?}
  \framesubtitle{CCSS: HSG.SRT.B5 Use similarity criteria to solve problems \hfill \alert{7.7 Friday 10 January}}

  \begin{block}{Similarity Unit Exam}
  \begin{enumerate}
    \item Similarity ratio calculations
    \item Applications of slope and linear equations
    \item Transformations
    \item Symmetry
  \end{enumerate}
  \end{block}
}

\section{7.8 Constructions review, Monday 13 January}
\frame
{
  \frametitle{GQ: How do we transform a figure onto itself?}
  \framesubtitle{CCSS: HSG.SRT.B5 Use similarity criteria to solve problems \hfill \alert{7.8 Monday 13 January}}

  \begin{block}{Do Now: Exam followup}
  \begin{enumerate}
    \item Reflection situations
    \item Using algebraic language to justify answers
    \item Analytic proof using the distance formula
  \end{enumerate}
  \end{block}
  Lesson: Constructions review \\[0.25cm]
  Right triangle similarity situations, cross multiplying to show ratios as equal products \\[0.5cm]
  Homework: Right triangle situations problem set \\ \alert{Test corrections due Wednesday}
}

\section{7.9 Constructions review, Tuesday 14 January}
\frame
{
  \frametitle{GQ: How do we perform simple classical constructions?}
  \framesubtitle{CCSS: HSG.SRT.B5 Use similarity criteria to solve problems \hfill \alert{7.9 Tuesday 14 January}}
  Do Now: Analytic geometry proofs
  \begin{block}{Lesson: Constructions review}
  \begin{enumerate}
    \item Equilateral triangle
    \item Segment and angle bisectors
    \item Duplicate a segment and an angle
    \item Perpendicular to a line through a point
    \item Special: hexagon, square, diameter, line of reflection
  \end{enumerate}
  \end{block}
  Homework: Constructions \\[0.25cm]
  \alert{Report card: final day tomorrow, test corrections due}
}

\section{7.10 Constructions review, Wednesday 15 January}
\frame
{
  \frametitle{GQ: How do we perform simple classical constructions?}
  \framesubtitle{CCSS: HSG.SRT.B5 Use similarity criteria to solve problems \hfill \alert{7.10 Wednesday 15 January}}

  \begin{block}{Do Now: Transformations review}
  \begin{enumerate}
    \item Translation
    \item Reflection
    \item Rotation
    \item Dilation
    \item Composition
  \end{enumerate}
  \end{block}
  Homework: Transformations \\[0.25cm]
  \alert{Report card: final day today, test corrections due}
}

\end{document}

